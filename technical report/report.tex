\documentclass[a4paper]{article}

\usepackage[utf8]{inputenc}
\usepackage[T1]{fontenc}

\usepackage[english]{babel}

\usepackage{macros}

\usepackage{listings}
\usepackage{amsmath}
\usepackage{graphicx}
\usepackage[colorinlistoftodos]{todonotes}
\usepackage{placeins}
\usepackage{mathpartir}
\usepackage{mathtools,array}
\usepackage{amssymb}
\usepackage{enumitem}
\usepackage{amsthm}
\usepackage{float}
\usepackage{stmaryrd}


\newtheorem{theorem}{Theorem}
\newtheorem{lemma}{Lemma}

\begin{document}

\title{Gradual Intersection Types}

\author{Pedro Ângelo}

\maketitle
\pagenumbering{arabic}

\section{Language Definition}
\begin{figure}[H]
Syntax
\begin{align*}
&Types\ \begin{aligned}[t] T &::=\ \Int\ |\ \Bool\ |\ \Dyn\ |\ T \rightarrow T'\ |\ T \cap \ldots \cap T\\
                          T' &::=\ \Int\ |\ \Bool\ |\ \Dyn\ |\ T' \rightarrow T' \end{aligned}\\
&Expressions\ \begin{aligned}[t] e\ ::=&\ x\ |\ \lambda x : T\ .\ e\ |\ e\ e\ |\ n\ |\ \true\ |\ \false\ |\ e + e\\
                                      |&\ e : T' \Rightarrow^l T'\ |\ e : c \cap \ldots \cap c\ |\ \blame{T}{l} \end{aligned}\\
&Ground\ Types\ G\ ::=\ \Int\ |\ \Bool\ |\ \Dyn \rightarrow \Dyn\\
&Casts\ \begin{aligned}[t] c\ ::=&\ c : T' \Rightarrow^l T'\ ^n\ |\ \blamecast{T'}{T'}{l}{n}\ |\ \emptycast{T'}{n}\ \iffalse |\ \stuckcast{T'}{T'}{n}\fi \end{aligned}\\
&Values\ \begin{aligned}[t] v\ ::=&\ x\ |\ \lambda x : T\ .\ e\ |\ n\ |\ \true\ |\ \false\ |\ \blame{T}{l}\\
                                 |&\ v : G \Rightarrow^l \Dyn\\
                                 |&\ v : T'_1 \rightarrow T'_2 \Rightarrow^l T'_3 \rightarrow T'_4\\
                                 |&\ v : cv_1 \cap \ldots \cap cv_n\ such\ that\\
                                 &\neg(\forall_{i\in 1..n}\ .\ cv_i = \blamecast{T'}{T'}{l}{m})\ \land\\
                                 &\neg(\forall_{i\in 1..n}\ .\ cv_i = \emptycast{T'}{m}) \end{aligned}\\
&Cast\ Values\ \begin{aligned}[t] cv\ ::=&\ cv1\ |\ cv2\\
                                cv1\ ::=&\ \emptycast{T'}{n} : G \Rightarrow^l \Dyn\ ^n\\
                                       |&\ \emptycast{T'}{n} : T'_1 \rightarrow T'_2 \Rightarrow^l T'_3 \rightarrow T'_4\\
                                       |&\ cv1 : G \Rightarrow^l \Dyn\ ^n\\
                                       |&\ cv1 : T'_1 \rightarrow T'_2 \Rightarrow^l T'_3 \rightarrow T'_4\\
                                cv2\ ::=&\ \blamecast{T'}{T'}{l}{n}\\
                                       |&\ \emptycast{T'}{n} \end{aligned}
\end{align*}

\hrulefill
\caption{Gradual Intersection System}
\label{intersection_syntax}
\end{figure}

\begin{figure}[H]
\framebox[1.2\width]{$\Gamma \vdash_{\cap G} e : T$}\ Typing
\begin{mathpar}
\inferrule* [right=${\rightarrow} I$]
{\Gamma, x : T_1 \cap \ldots \cap T_n \vdash_{\cap G}\ e : T}
{\Gamma \vdash_{\cap G} \lambda x : T_1 \cap \ldots \cap T_n\ .\ e : T_1 \cap \ldots \cap T_n \rightarrow T}

\inferrule* [right=${\rightarrow} I'$]
{\Gamma, x : T_i \vdash_{\cap G}\ e : T}
{\Gamma \vdash_{\cap G} \lambda x : T_1 \cap \ldots \cap T_n\ .\ e : T_i \rightarrow T}

\inferrule* [right=${\rightarrow} E$]
{\Gamma \vdash_{\cap G} e_1 : PM \\
 PM \rhd T_1 \cap \ldots \cap T_n \rightarrow T \\\\
 \Gamma \vdash_{\cap G} e_2 : T_1' \cap \ldots \cap T_n' \\
 T_1' \cap \ldots \cap T_n' \sim T_1 \cap \ldots \cap T_n}
{\Gamma \vdash_{\cap G} e_1\ e_2 : T}

\inferrule* [right=${\cap} I$]
{\Gamma \vdash_{\cap G} e : T_1\ \ldots\ \Gamma \vdash_{\cap G} e : T_n}
{\Gamma \vdash_{\cap G} e : T_1 \cap \ldots \cap T_n}

\inferrule* [right=${\cap} E$]
{\Gamma \vdash_{\cap G} e : T_1 \cap \ldots \cap T_n}
{\Gamma \vdash_{\cap G} e : T_i}

\inferrule* [right=$Var$]
{x : T \in \Gamma}
{\Gamma \vdash_{\cap G} x : T}
\end{mathpar}

\framebox[1.2\width]{$T \sim T$}\ Consistency
\begin{mathpar}
\inferrule* []
{ }
{B \sim B}

\inferrule* []
{ }
{T \sim \Dyn}

\inferrule* []
{ }
{\Dyn \sim T}

\inferrule* []
{T_1 \sim T_3 \\ T_2 \sim T_4}
{T_1 \rightarrow T_2 \sim T_3 \rightarrow T_4}

\inferrule* []
{T_1 \sim T'_1\ \ldots\ T_n \sim T'_n}
{T_1 \cap \ldots \cap T_n \sim T'_1 \cap \ldots \cap T'_n}

\inferrule* []
{T \sim T_1\ \ldots\ T \sim T_n}
{T \sim T_1 \cap \ldots \cap T_n}

\inferrule* []
{T_1 \sim T\ \ldots\ T_n \sim T}
{T_1 \cap \ldots \cap T_n \sim T}
\end{mathpar}

\framebox[1.2\width]{$T \rhd T$}\ Pattern Matching
\begin{mathpar}
\inferrule* []
{ }
{(T_1 \rightarrow T_2) \rhd T_1 \rightarrow T_2}

\inferrule* []
{ }
{\Dyn \rhd \Dyn \rightarrow \Dyn}
\end{mathpar}

\framebox[1.2\width]{$T \sqsubseteq T$}\ Type Precision
\begin{mathpar}
\inferrule* []
{ }
{\Dyn \sqsubseteq T}

\inferrule* []
{ }
{B \sqsubseteq B}

\inferrule* []
{T_1 \sqsubseteq T_3 \\ T_2 \sqsubseteq T_4}
{T_1 \rightarrow T_2 \sqsubseteq T_3 \rightarrow T_4}

\inferrule* []
{T_1 \sqsubseteq T'_1\ \ldots\ T_n \sqsubseteq T'_n}
{T_1 \cap \ldots \cap T_n \sqsubseteq T'_1 \cap \ldots \cap T'_n}

\inferrule* []
{T \sqsubseteq T_1\ \ldots\ T \sqsubseteq T_n}
{T \sqsubseteq T_1 \cap \ldots \cap T_n}

\inferrule* []
{T_1 \sqsubseteq T\ \ldots\ T_n \sqsubseteq T}
{T_1 \cap \ldots \cap T_n \sqsubseteq T}
\end{mathpar}

\framebox[1.2\width]{$e \sqsubseteq e$}\ Term Precision
\begin{mathpar}
\inferrule* []
{ }
{x \sqsubseteq e}

\inferrule* []
{T \sqsubseteq T' \\ e \sqsubseteq e'}
{\lambda x : T\ .\ e \sqsubseteq \lambda x : T'\ .\ e'}

\inferrule* []
{e_1 \sqsubseteq e_1' \\e_2 \sqsubseteq e_2'}
{e_1\ e_2 \sqsubseteq e_1'\ e_2'}
\end{mathpar}
\hrulefill
\caption{Gradual Intersection Type System ($\vdash_{\cap G}$)}
\label{intersection_type_system}
\end{figure}

\begin{figure}[H]
\framebox[1.2\width]{$\Gamma \vdash_{\cap CC} e : T$}\ Typing
\begin{mathpar}
\inferrule* []
{}
{rules\ in\ Figure\ \ref{intersection_type_system}\ and}\\

\inferrule* [right=T-App]
{\Gamma \vdash_{\cap CC} e_1 : T_{11} \rightarrow T_{12} \cap \ldots \cap T_{n1} \rightarrow T_{n2} \\
 \Gamma \vdash_{\cap CC} e_2 : T_1' \cap \ldots \cap T_n' \\
 T_{11} \sim T_1' \ldots T_{n1} \sim T_n'}
{\Gamma \vdash_{\cap CC} e_1\ e_2 : T_{12} \cap \ldots \cap T_{n2}}

\inferrule* [right=T-Cast]
{\Gamma \vdash_{\cap CC} e : T_1 \\ T_1 \sim T_2}
{\Gamma \vdash_{\cap CC} (e : T_1 \Rightarrow^l T_2) : T_2}

\inferrule* [right=T-Blame]
{ }
{\Gamma \vdash_{\cap CC} \blame{T}{l} : T}

\inferrule* [right=T-IntersectionCast]
{\Gamma \vdash_{\cap CC} e : T \\ \vdash_{\cap IC} c_1 : T_1\ \ldots\ \vdash_{\cap IC} c_n : T_n \\\\ initialType(c_1) \cap \ldots \cap initialType(c_n) =_{\cap} T}
{\Gamma \vdash_{\cap CC} (e : c_1 \cap \ldots \cap c_n) : T_1 \cap \ldots \cap T_n}
\end{mathpar}

\framebox[1.2\width]{initialType(c) = T}
\begin{mathpar}
\inferrule* []
{}
{initialType(c : T_1 \Rightarrow^l T_2\ ^n) = initialType(c)}\\

\inferrule* []
{}
{initialType(\emptycast{T}{n}) = T}\\

\inferrule* []
{}
{initialType(\blamecast{T_I}{T_F}{l}{n}) = T_I}
\end{mathpar}

\framebox[1.2\width]{finalType(c) = T}
\begin{mathpar}
\inferrule* []
{}
{finalType(c : T_1 \Rightarrow^l T_2\ ^n) = T_2}\\

\inferrule* []
{}
{finalType(\emptycast{T}{n}) = T}\\

\inferrule* []
{}
{finalType(\blamecast{T_I}{T_F}{l}{n}) = T_F}
\end{mathpar}
\hrulefill
\caption{Intersection Cast Calculus ($\vdash_{\cap CC}$)}
\label{intersection_cast_calculus}
\end{figure}

\begin{figure}[H]
\framebox[1.2\width]{$\Gamma \vdash_{\cap CC} e \leadsto e : T$}\ Compilation
\begin{mathpar}
\inferrule* []
{x : T_1 \cap \ldots \cap T_n \in \Gamma}
{\Gamma \vdash_{\cap CC} x \leadsto x : T_i}

\inferrule* []
{\Gamma, x : T_1 \cap \ldots \cap T_n \vdash_{\cap CC} e \leadsto e' : T}
{\Gamma \vdash_{\cap CC} (\lambda x : T_1 \cap \ldots \cap T_n\ .\ e) \leadsto (\lambda x : T_1 \cap \ldots \cap T_n\ .\ e') : T_1 \cap \ldots \cap T_n \rightarrow T}

\inferrule* []
{\Gamma \vdash_{CC} e_1 \leadsto e_1' : PM \\ PM \rhd T_1 \cap \ldots \cap T_n \rightarrow T \\\\ \Gamma \vdash_{CC} e_2 \leadsto e_2' : T'_1 \cap \ldots \cap T'_n \\ T'_1 \cap \ldots \cap T'_n \sim T_1 \cap \ldots \cap T_n\\\\ e_1'' = addCasts(getInstances(PM),\ getInstances(T_1 \cap \ldots \cap T_n \rightarrow T),\ e_1') \\\\ e_2'' = addCasts(getInstances(T'_1 \cap \ldots \cap T'_n),\ getInstances(T_1 \cap \ldots \cap T_n),\ e_2')}
{\Gamma \vdash_{CC} e_1\ e_2 \leadsto e_1''\ e_2'' : T}
\end{mathpar}

\framebox[1.2\width]{getInstances(T) = \{T\}}
\begin{mathpar}
\inferrule* []
{}
{getInstances(\Int) = \{\Int\}}\\

\inferrule* []
{}
{getInstances(\Bool) = \{\Bool\}}\\

\inferrule* []
{}
{getInstances(\Dyn) = \{\Dyn\}}\\

\inferrule* []
{}
{getInstances(T_1 \rightarrow T_2) =\\\\ let\ \{T_{11}, \ldots, T_{1n}\} = getInstances(T_1)\ in\ \{T_{11} \rightarrow T_2, \ldots, T_{1n} \rightarrow T_2\}}\\

\inferrule* []
{}
{getInstances(T_1 \cap \ldots \cap T_n) =\\\\ let\ \{T_{11}, \ldots, T_{1m}\} = getInstances(T_1)\\\\ \ldots\\\\ let\ \{T_{n1}, \ldots, T_{nj}\} = getInstances(T_n)\\\\ in\ \{T_{11}, \ldots, T_{1m}, \ldots, T_{n1}, \ldots, T_{nj}\}}
\end{mathpar}

\framebox[1.2\width]{addCasts(\{T\}, \{T\}, e) = e}
\begin{mathpar}
\inferrule* []
{}
{addCasts(\{T_1\}, \{T_2\}, e) = e : T_1 \Rightarrow^l T_2}\\

\inferrule* []
{}
{addCasts(\{T_{11}, \ldots, T_{1n}\}, \{T_{21}, \ldots, T_{2n}\}, e) =\\\\e : (\emptycast{T_{11}}{0} : T_{11} \Rightarrow^l T_{21}\ ^{0}) \cap \ldots \cap (\emptycast{T_{1n}}{0} : T_{1n} \Rightarrow^l T_{2n}\ ^{0})}\\

\inferrule* []
{}
{addCasts(\{T_{11}, \ldots, T_{1n}\}, \{T_2\}, e) =\\\\e : (\emptycast{T_{11}}{0} : T_{11} \Rightarrow^l T_2\ ^{0}) \cap \ldots \cap (\emptycast{T_{1n}}{0} : T_{1n} \Rightarrow^l T_2\ ^{0})}\\

\inferrule* []
{}
{addCasts(\{T_1\}, \{T_{21}, \ldots, T_{2n}\}, e) =\\\\e : (\emptycast{T_1}{0} : T_1 \Rightarrow^l T_{21}\ ^{0}) \cap \ldots \cap (\emptycast{T_1}{0} : T_1 \Rightarrow^l T_{2n}\ ^{0})}
\end{mathpar}
\hrulefill
\caption{Compilation to the Cast Calculus}
\label{intersection_cast_insertion}
\end{figure}

\begin{figure}[H]
\framebox[1.2\width]{$e \longrightarrow_{\cap CC} e$}\ Evaluation
\begin{mathpar}
\inferrule* []
{}
{Simulate\ casts\ on\ data\ types}

\inferrule* [right=Simulate$\cap$]
{isValue\ v_1 : cv_1 \cap \ldots \cap cv_n \\ \exists i \in 1..n\ .\ isArrowCompatible\ cv_i \\\\ (cv_1', \ldots, cv_m') = filter\ isArrowCompatible\ (cv_1, \ldots, cv_n) \\\\ ((c_{11}, c_{12}, r_1), \ldots, (c_{m1}, c_{m2}, r_m)) = map\ simulateArrow\ (cv_1', \ldots, cv_m')}
{(v_1 : cv_1 \cap \ldots \cap cv_n)\ v_2 \longrightarrow_{\cap CC}\\ (v_1 : r_1 \cap \ldots \cap r_m)\ (v_2 : c_{11} \cap \ldots \cap c_{m1}) : c_{12} \cap \ldots \cap c_{m2}}

\inferrule* []
{}
{Merge\ casts}\\

\inferrule* [right=MergeIC$\cap$]
{isValue\ v : cv_1 \cap \ldots \cap cv_n \\ v : c_1' \cap \ldots \cap c_m' = mergeIC (v : cv_1 \cap \ldots \cap cv_n : T_1 \Rightarrow^l T_2)}
{v : cv_1 \cap \ldots \cap cv_n : T_1 \Rightarrow^l T_2 \longrightarrow_{\cap CC} v : c_1' \cap \ldots \cap c_m'}

\inferrule* [right=MergeCI$\cap$]
{isValue\ v : T_1 \Rightarrow^l T_2 \\ v : c_1' \cap \ldots \cap c_m' = mergeCI (v : T_1 \Rightarrow^l T_2 : c_1 \cap \ldots \cap c_n)}
{v : T_1 \Rightarrow^l T_2 : c_1 \cap \ldots \cap c_n \longrightarrow_{\cap CC} v : c_1' \cap \ldots \cap c_m'}\\

\inferrule* [right=MergeII$\cap$]
{isValue\ v : cv_1 \cap \ldots \cap cv_n \\ v : c_1'' \cap \ldots \cap c_j'' = mergeII (v : cv_1 \cap \ldots \cap cv_n : c_1' \cap \ldots \cap c_m')}
{v : cv_1 \cap \ldots \cap cv_n : c_1' \cap \ldots \cap c_m' \longrightarrow_{\cap CC} v : c_1'' \cap \ldots \cap c_j''}\\

\inferrule* []
{}
{Evaluate\ intersection\ casts}\\

\inferrule* [right=Evaluate$\cap$]
{\neg(\forall i \in 1..n\ .\ isCastValue\ c_i) \\ c_1 \longrightarrow_{\cap IC} cv_1\ \ldots\ c_n \longrightarrow_{\cap IC} cv_n}
{v : c_1 \cap \ldots \cap c_n \longrightarrow_{\cap CC} v : cv_1 \cap \ldots \cap cv_n}\\

\inferrule* []
{}
{Transition\ from\ cast\ values\ to\ values}

\inferrule* [right=PropagateBlame$\cap$]
{ }
{v : \blamecast{T_1'}{T_1}{l_1}{m_1} \cap \ldots \cap \blamecast{T_n'}{T_n}{l_n}{m_n}\\ \longrightarrow_{\cap CC} \blame{(T_1 \cap \ldots \cap T_n)}{l_1}}

\inferrule* [right=RemoveEmpty$\cap$]
{ }
{v : \emptycast{T_1}{m_1} \cap \ldots \cap \emptycast{T_n}{m_n} \longrightarrow_{\cap CC} v}
\begin{comment}

\inferrule* [right=RemoveStuck$\cap$]
{\neg(\forall i \in 1..n\ .\ isStuckCast\ cv_i) \\ \exists i \in 1..n\ .\ isStuckCast\ cv_i  \\ (cv_1', \ldots, cv_m') = filter\ (\neg isStuckCast)\ (cv_1, \ldots, cv_n)}
{v : cv_1 \cap \ldots \cap cv_n \longrightarrow_{\cap CC} v : cv_1' \cap \ldots \cap cv_m'}
\end{comment}
\end{mathpar}
\hrulefill
\caption{Cast Calculus Semantics ($\longrightarrow_{\cap CC}$)}
\label{casts_calculus_semantics}
\end{figure}

\begin{figure}[H]
\framebox[1.2\width]{$\vdash_{\cap IC} c : T$}\ Typing
\begin{mathpar}

\inferrule* [right=T-SingleC]
{\vdash_{\cap IG} c : T_1 \\ T_1 \sim T_2}
{\vdash_{\cap IG} (c : T_1 \Rightarrow^l T_2\ ^n) : T_1}

\inferrule* [right=T-EmptyC]
{ }
{\vdash_{\cap IG} \emptycast{T}{n} : T}

\inferrule* [right=T-BlameC]
{ }
{\vdash_{\cap IG} \blamecast{T_I}{T_F}{l}{n} : T_F}
\begin{comment}

\inferrule* [right=T-StuckC]
{ }
{\vdash_{\cap IG} \stuckcast{T_I}{T_F}{n} : T_F}
\end{comment}
\end{mathpar}
\hrulefill
\caption{Intersection Casts Type System ($\vdash_{\cap IC}$)}
\label{intersection_casts_type_system}
\end{figure}

\begin{figure}[H]
\framebox[1.2\width]{$c \longrightarrow_{\cap IC} c$}\ Evaluation
\begin{mathpar}
\inferrule* []
{}
{Push\ blame\ to\ top\ level}\\

\inferrule* [right=PushBlameC]
{ }
{\blamecast{T_I}{T_F}{l_1}{n_1} : T_1 \Rightarrow^{l_2} T_2\ ^{n_2} \longrightarrow_{\cap IC} \blamecast{T_I}{T_2}{l_1}{n_1}}

\begin{comment}
\inferrule* [right=PushStuckC]
{ }
{\stuckcast{T_I}{T_F}{n_1} : T_1 \Rightarrow^l T_2\ ^{n_2} \longrightarrow_{\cap IC} \stuckcast{T_I}{T_2}{n_1}}\\
\end{comment}

\inferrule* []
{}
{Evaluate\ inside\ casts}\\

\inferrule* [right=EvaluateC]
{\neg(isCastValue\ c) \\ c \longrightarrow_{\cap IC} c'}
{c : T_1 \Rightarrow^l T_2\ ^n \longrightarrow_{\cap IC} c' : T_1 \Rightarrow^l T_2\ ^n}

\inferrule* []
{}
{Detect\ success\ or\ failure\ of\ casts}\\

\inferrule* [right=IdentityC]
{isCastValue1\ c \lor isEmptyCast\ c}
{c : T \Rightarrow^l T\ ^n \longrightarrow_{\cap IC} c}

\inferrule* [right=SucceedC]
{isCastValue1\ c \lor isEmptyCast\ c}
{c : G \Rightarrow^{l_1} \Dyn\ ^{n_1} : \Dyn \Rightarrow^{l_2} G\ ^{n_2} \longrightarrow_{\cap IC} c}

\inferrule* [right=FailC]
{isCastValue1\ c \lor isEmptyCast\ c \\ \neg(same\ ground\ G_1\ G_2) \\ initialType(c) = T_I}
{c : G_1 \Rightarrow^{l_1} \Dyn\ ^{n_1} : \Dyn \Rightarrow^{l_2} G_2\ ^{n_2} \longrightarrow_{\cap IC} \blamecast{T_I}{G_2}{l_2}{n_1}}\\

\inferrule* []
{}
{Mediate\ the\ transition\ between\ the\ two\ disciplines}\\

\inferrule* [right=GroundC]
{isCastValue1\ c \lor isEmptyCast\ c \\ G\ is\ ground\ type\ of\ T \\ \neg(ground\ T)}
{c : T \Rightarrow^l \Dyn\ ^n \longrightarrow_{\cap IC} c : T \Rightarrow^l G\ ^n : G \Rightarrow^l \Dyn\ ^n}

\inferrule* [right=ExpandC]
{isCastValue1\ c \lor isEmptyCast\ c \\ G\ is\ ground\ type\ of\ T \\ \neg(ground\ T)}
{c : \Dyn \Rightarrow^l T\ ^n \longrightarrow_{\cap IC} c : \Dyn \Rightarrow^l G\ ^n : G \Rightarrow^l T\ ^n}
\begin{comment}

\inferrule* []
{}
{Trigger\ stuck}\\

\inferrule* [right=TriggerStuckC]
{isCastValue1\ c \lor isEmptyCast\ c \\ initialType(c) = T_I}
{c : T_1 \Rightarrow^l T_2\ ^n \longrightarrow_{\cap IC} \stuckcast{T_I}{T_2}{n}}
\end{comment}
\end{mathpar}
\hrulefill
\caption{Intersection Casts Semantics ($\longrightarrow_{\cap IC}$)}
\label{intersection_casts_semantics}
\end{figure}

\begin{figure}[H]
\framebox[1.2\width]{\erasee{e} = e} Erase identity casts
\begin{mathpar}
\inferrule* []
{}
{\erasee{x} = x}\\

\inferrule* []
{}
{\erasee{\lambda x : T\ .\ e} = \lambda x : T\ .\ \erasee{e}}\\

\inferrule* []
{}
{\erasee{e_1\ e_2} = \erasee{e_1}\ \erasee{e_2}}\\

\inferrule* []
{}
{\erasee{n} = n}\\

\inferrule* []
{}
{\erasee{\true} = \true}\\

\inferrule* []
{}
{\erasee{\false} = \false}\\

\inferrule* []
{}
{\erasee{e_1 + e_2} = \erasee{e_1} + \erasee{e_2}}\\

\inferrule* []
{}
{\erasee{e : T \Rightarrow^l T} = \erasee{e}}\\

\inferrule* []
{}
{\erasee{e : T_1 \Rightarrow^l T_2} = \erasee{e} : T_1 \Rightarrow^l T_2}\\

\inferrule* []
{\erasec{c_1} = \emptycast{T_1}{n_1}\ \ldots\ \erasec{c_n} = \emptycast{T_n}{n_n}}
{\erasee{e : c_1 \cap \ldots \cap c_n} = \erasee{e}}\\

\inferrule* []
{\erasec{c_1} = c_1'\ \ldots\ \erasec{c_n} = c_n'}
{\erasee{e : c_1 \cap \ldots \cap c_n} = \erasee{e} : c_1' \cap \ldots \cap c_n'}\\

\end{mathpar}

\framebox[1.2\width]{\erasec{c} = c} Erase identity casts
\begin{mathpar}
\inferrule* []
{}
{\erasec{c : T \Rightarrow^l T\ ^n} = \erasec{c}}\\

\inferrule* []
{}
{\erasec{c : T_1 \Rightarrow^l T_2\ ^n} = \erasec{c} : T_1 \Rightarrow^l T_2\ ^n}\\

\inferrule* []
{}
{\erasec{\blamecast{T_I}{T_F}{l}{n}} = \blamecast{T_I}{T_F}{l}{n}}\\

\inferrule* []
{}
{\erasec{\emptycast{T}{n}} = \emptycast{T}{n}}
\begin{comment}
\inferrule* []
{}
{\erasec{\stuckcast{T_I}{T_F}{n}} = \stuckcast{T_I}{T_F}{n}}\\
\end{comment}
\end{mathpar}
\hrulefill
\caption{Identity Cast Erasure}
\label{identity_cast_erasure}
\end{figure}

\newpage

\section{Proofs}
\begin{theorem}[Conservative Extension]
\label{equivalence}
Depends on Lemma \ref{consistencytoequality}.
If e is fully static and T is a static type, then $\Gamma \vdash_{\cap S} e : T \iff \Gamma \vdash_{\cap G} e : T$.
\end{theorem}
\begin{proof}
First we will prove that if $\vdash_{\cap S} e : T$ then $\vdash_{\cap G} e : T$.
We proceed by induction on the length of the derivation tree of $\vdash_{\cap S}$.\\\\
Base case:
\begin{itemize}
    \item e = $x$.
    If $\Gamma \vdash_{\cap S} x : T_i$, then $x : T_1 \cap \ldots \cap T_n \in \Gamma$ such that $T_i \in \{T_1, \ldots, T_n\}$.
    Therefore, by rule $\cap E$ of $\vdash_{\cap G}$, $\Gamma \vdash_{\cap G} e : T_i$.\\
\end{itemize}
Induction step:
\begin{itemize}
    \item e = $\lambda x\ .\ T_1 \cap \ldots \cap T_n\ .\ e'$.
    There are two possibilities:
    \begin{itemize}
        \item Using the rule ${\rightarrow} I$.
        If $\Gamma \vdash_{\cap S} \lambda x\ .\ T_1 \cap \ldots \cap T_n\ .\ e' : T_1 \cap \ldots \cap T_n \rightarrow T$, then $\Gamma, x : T_1 \cap \ldots \cap T_n \vdash_{\cap S} e' : T$.
        By the induction hypothesis, $\Gamma, x : T_1 \cap \ldots \cap T_n \vdash_{\cap G} e' : T$.
        Therefore, by rule ${\rightarrow} I$, $\Gamma \vdash_{\cap G} \lambda x\ .\ T_1 \cap \ldots \cap T_n\ .\ e' : T_1 \cap \ldots \cap T_n \rightarrow T$.
        \item Using the rule ${\rightarrow} I'$.
        If $\Gamma \vdash_{\cap S} \lambda x\ .\ T_1 \cap \ldots \cap T_n\ .\ e' : T_i \rightarrow T$, then $\Gamma, x : T_i \vdash_{\cap S} e' : T$.
        By the induction hypothesis, $\Gamma, x : T_i \vdash_{\cap G} e' : T$.
        Therefore, by rule ${\rightarrow} I'$, $\Gamma \vdash_{\cap G} \lambda x\ .\ T_1 \cap \ldots \cap T_n\ .\ e' : T_i \rightarrow T$.
    \end{itemize}
    \item e = $e_1\ e_2$.
    If $\Gamma \vdash_{\cap S} e_1\ e_2 : T$ then $\Gamma \vdash_{\cap S} e_1 : T_1 \cap \ldots \cap T_n \rightarrow T$ and $\Gamma \vdash_{\cap S} e_2 : T_1 \cap \ldots \cap T_n$.
    By the induction hypothesis, $\Gamma \vdash_{\cap G} e_1\ : T_1 \cap \ldots \cap T_n \rightarrow T$ and $\Gamma \vdash_{\cap G} e_2 : T_1 \cap \ldots \cap T_n$.
    By the definition of $\rhd$, $T_1 \cap \ldots \cap T_n \rightarrow T \rhd T_1 \cap \ldots \cap T_n \rightarrow T$.
    By the definition of consistency ($T \sim T$), $T_1 \cap \ldots \cap T_n \sim T_1 \cap \ldots \cap T_n$.
    Therefore, by rule ${\rightarrow} E$, $\Gamma \vdash_{\cap G} e_1\ e_2 : T$.
    \item e = $e$.
    If $\Gamma \vdash_{\cap S} e : T_1 \cap \ldots \cap T_n$ then $\Gamma \vdash_{\cap S} e : T_1$ and ... and $\Gamma \vdash_{\cap S} e : T_n$.
    By the induction hypothesis, $\Gamma \vdash_{\cap G} e : T_1$ and ... and $\Gamma \vdash_{\cap G} e : T_n$.
    Therefore, by rule ${\cap} E$, $\Gamma \vdash_{\cap G} e : T_1 \cap \ldots \cap T_n$.\\
\end{itemize}
Now we will prove that if $\vdash_{\cap G} e : T$ then $\vdash_{\cap S} e : T$.
We proceed by induction on the length of the derivation tree of $\vdash_{\cap G}$.\\\\
Base case:
\begin{itemize}
    \item e = $x$.
    If $\Gamma \vdash_{\cap G} x : T_i$, then $x : T_1 \cap \ldots \cap T_n \in \Gamma$ such that $T_i \in \{T_1, \ldots, T_n\}$.
    Therefore, by rule $\cap E$ of $\vdash_{\cap S}$, $\Gamma \vdash_{\cap S} e : T_i$.\\
\end{itemize}
Induction step:
\begin{itemize}
    \item e = $\lambda x\ .\ T_1 \cap \ldots \cap T_n\ .\ e'$.
    There are two possibilities:
    \begin{itemize}
        \item Using the rule ${\rightarrow} I$.
        If $\Gamma \vdash_{\cap G} \lambda x\ .\ T_1 \cap \ldots \cap T_n\ .\ e' : T_1 \cap \ldots \cap T_n \rightarrow T$, then $\Gamma, x : T_1 \cap \ldots \cap T_n \vdash_{\cap G} e' : T$.
        By the induction hypothesis, $\Gamma, x : T_1 \cap \ldots \cap T_n \vdash_{\cap S} e' : T$.
        Therefore, by rule ${\rightarrow} I$, $\Gamma \vdash_{\cap S} \lambda x\ .\ T_1 \cap \ldots \cap T_n\ .\ e' : T_1 \cap \ldots \cap T_n \rightarrow T$.
        \item Using the rule ${\rightarrow} I'$.
        If $\Gamma \vdash_{\cap G} \lambda x\ .\ T_1 \cap \ldots \cap T_n\ .\ e' : T_i \rightarrow T$, then $\Gamma, x : T_i \vdash_{\cap G} e' : T$.
        By the induction hypothesis, $\Gamma, x : T_i \vdash_{\cap S} e' : T$.
        Therefore, by rule ${\rightarrow} I'$, $\Gamma \vdash_{\cap S} \lambda x\ .\ T_1 \cap \ldots \cap T_n\ .\ e' : T_i \rightarrow T$.
    \end{itemize}
    \item e = $e_1\ e_2$.
    If $\Gamma \vdash_{\cap G} e_1\ e_2 : T$ then $\Gamma \vdash_{\cap G} e_1 : PM$, $PM \rhd T_1 \cap \ldots \cap T_n \rightarrow T$, $\Gamma \vdash_{\cap G} e_2 : T'_1 \cap \ldots \cap T'_n$ and $T'_1 \cap \ldots \cap T'_n \sim T_1 \cap \ldots \cap T_n$.
    By the definition of $\rhd$, $PM = T_1 \cap \ldots \cap T_n \rightarrow T$, therefore $\Gamma \vdash_{\cap G} e_1 : T_1 \cap \ldots \cap T_n \rightarrow T$.
    By Lemma \ref{consistencytoequality}, $T'_1 \cap \ldots \cap T'_n = T_1 \cap \ldots \cap T_n$, and therefore $\Gamma \vdash_{\cap G} e_2 : T_1 \cap \ldots \cap T_n$.
    By the induction hypothesis, $\Gamma \vdash_{\cap S} e_1\ : T_1 \cap \ldots \cap T_n \rightarrow T$ and $\Gamma \vdash_{\cap S} e_2 : T_1 \cap \ldots \cap T_n$.
    Therefore, by rule ${\rightarrow} E$, $\Gamma \vdash_{\cap S} e_1\ e_2 : T$.
    \item e = $e$.
    If $\Gamma \vdash_{\cap G} e : T_1 \cap \ldots \cap T_n$ then $\Gamma \vdash_{\cap G} e : T_1$ and ... and $\Gamma \vdash_{\cap G} e : T_n$.
    By the induction hypothesis, $\Gamma \vdash_{\cap S} e : T_1$ and ... and $\Gamma \vdash_{\cap S} e : T_n$.
    Therefore, by rule ${\cap} E$, $\Gamma \vdash_{\cap S} e : T_1 \cap \ldots \cap T_n$.\\
\end{itemize}
\end{proof}

\begin{theorem}[Conservative Extension]
\label{equivalence}
Depends on Lemmas \ref{staticcompilation} and \ref{eliminationidentitycastse}.
If e is fully static, T is a static type and $\Gamma \vdash_{\cap CC} e \leadsto e' : T$, then $e \longrightarrow_{\cap S} v \iff e' \longrightarrow_{\cap CC} v$.
\end{theorem}
\begin{proof}
Since $\Gamma \vdash_{\cap CC} e \leadsto e' : T$ and $e$ is fully static, then by Lemma \ref{staticcompilation} and by the definition of $\Gamma \vdash_{\cap CC} e \leadsto e' : T$, the expression $e$ equals $e'$, except that $e'$ contains identity casts.
Therefore, $\erasee{e'} = e$.
Then, by Lemma \ref{eliminationidentitycastse}, if $e \longrightarrow v$ and $e' \longrightarrow_{\cap CC} v'$, then v = v'.
\end{proof}

\begin{theorem}[Monotonicity w.r.t. precision]
\label{monotonicity}
If $\Gamma \vdash_{\cap G} e : T$ and $e' \sqsubseteq e$ then $\exists T'\ .\ \Gamma \vdash_{\cap G} e' : T'$ and $T' \sqsubseteq T$.
\end{theorem}
\begin{proof}
We proceed by induction on the length of the derivation tree of $\Gamma \vdash_{\cap G} e : T$.\\\\
Base case:
\begin{itemize}
    \item e = $x$.
    If $\Gamma \vdash_{\cap G} x : T_i$ and $x \sqsubseteq x$, then $\Gamma \vdash_{\cap G} x : T_i$ and $T_i \sqsubseteq T_i$.
    \item e = $n$.
    If $\Gamma \vdash_{\cap G} n : \Int$ and $n \sqsubseteq n$, then $\Gamma \vdash_{\cap G} n : \Int$ and $\Int \sqsubseteq \Int$.
    \item e = $\true$.
    If $\Gamma \vdash_{\cap G} \true : \Bool$ and $\true \sqsubseteq \true$, then $\Gamma \vdash_{\cap G} \true : \Bool$ and $\Bool \sqsubseteq \Bool$.
    \item e = $\false$.
    If $\Gamma \vdash_{\cap G} \false : \Bool$ and $\false \sqsubseteq \false$, then $\Gamma \vdash_{\cap G} \false : \Bool$ and $\Bool \sqsubseteq \Bool$.
\end{itemize}
Induction step:
\begin{itemize}
    \item e = $\lambda x : T_1 \cap \ldots \cap T_n\ .\ e_1$.
    If $\Gamma \vdash_{\cap G} \lambda x : T_1\ .\ e_1 : T_1 \rightarrow T_2$ and $\lambda x : T_1'\ .\ e_1' \sqsubseteq \lambda x : T_1\ .\ e_1$, then $\Gamma \vdash_{\cap G} e_1 : T_2$, $T_1' \sqsubseteq T_1$ and $e_1' \sqsubseteq e_1$.
    By the induction hypothesis, $\exists T_2'\ .\ \Gamma \vdash_{\cap G} e_1' : T_2'\ and\ T_2' \sqsubseteq T_2$.
    As $\Gamma \vdash_{\cap G} \lambda x : T_1'\ .\ e_1' : T_1' \rightarrow T_2'$, and by the definition of $\sqsubseteq$, $T_1' \rightarrow T_2' \sqsubseteq T_1 \rightarrow T_2$, then it is proved.
    \item e = $e_1\ e_2$.
    If $\Gamma \vdash_{\cap G} e_1\ e_2 : T$ and $e_1'\ e_2' \sqsubseteq e_1\ e_2$ then $\Gamma \vdash_{\cap G} e_1 : PM$, $PM \rhd T_{11} \cap \ldots \cap T_{1n} \rightarrow T$, $\Gamma \vdash_{\cap G} e_2 : T_{21} \cap \ldots \cap T_{2n}$, and $T_{21} \cap \ldots \cap T_{2n} \sim T_{11} \cap \ldots \cap T_{1n}$, $e_1' \sqsubseteq e_1$ and $e_2' \sqsubseteq e_2$.
    By the induction hypothesis, $\exists PM'\ .\ \Gamma \vdash_{\cap G} e_1' : PM'\ and\ PM' \sqsubseteq PM\ and\ PM' \rhd T_{11}' \cap \ldots \cap T_{1n}' \rightarrow T'$ and $\exists T_{21}', \ldots, T_{2n}'\ .\ \Gamma \vdash_{\cap G} e_2' : T_{21}' \cap \ldots \cap T_{2n}'\ and\ T_{21}' \cap \ldots \cap T_{2n}' \sqsubseteq T_{21} \cap \ldots \cap T_{2n}\ and\ T_{21}' \cap \ldots \cap T_{2n}' \sim T_{11}' \cap \ldots \cap T_{1n}'$.
    By the definition of $\sqsubseteq$ and $\rhd$, $T_{11}' \cap \ldots \cap T_{1n}' \rightarrow T' \sqsubseteq T_{11} \cap \ldots \cap T_{1n} \rightarrow T$, and therefore, $T' \sqsubseteq T$.
    As $\Gamma \vdash_{\cap G} e_1'\ e_2' : T'$, it is proved.
    \item e = $e$.
    If $\Gamma \vdash_{\cap G} e : T_1 \cap \ldots \cap T_n$ and $e' \sqsubseteq e$, then $\Gamma \vdash_{\cap G} e : T_1$ and ... and $\Gamma \vdash_{\cap G} e : T_n$.
    By the induction hypothesis, $\exists T_1'\ .\ \Gamma \vdash_{\cap G} e' : T_1'\ and\ T_1' \sqsubseteq T_1$ and ... and $\exists T_n'\ .\ \Gamma \vdash_{\cap G} e' : T_n'\ and\ T_n' \sqsubseteq T_n$.
    Then, $\Gamma \vdash_{\cap G} e' : T_1' \cap \ldots \cap T_n'$ and by the definition of $\sqsubseteq$, $T_1' \cap \ldots \cap T_n' \sqsubseteq T_1 \cap \ldots \cap T_n$, then it is proved.
\end{itemize}
\end{proof}

\begin{theorem}[Subject reduction of $\longrightarrow_{\cap CC}$]
\label{subjectreduction}
Depends on Lemmas \ref{subjectreductionIC} and \ref{initialtypepreservationIC}.
If $\Gamma \vdash_{\cap CC} e : T$ and $e \longrightarrow_{\cap CC} e'$ then $\Gamma \vdash_{\cap CC} e' : T$.
\end{theorem}
\begin{proof}
We proceed by induction on the length of the derivation tree of $\longrightarrow_{\cap CC}$.\\\\
Base case:
\begin{itemize}
    \item $e = (v_1 : cv_1 \cap \ldots \cap cv_n)\ v_2$ and $isValue\ v_1 : cv1 \cap \ldots \cap cv_n$ and $\exists i \in 1 .. n\ .\ isArrowCompatible(cv_i)$ and $(cv_1', \ldots, cv_m') = filter\ isArrowCompatible (cv_1, \ldots, cv_n)$ and $((c_{11}, c_{12}, r_1), \ldots, (c_{m1}, c_{m2}, r_m)) = map\ simulateArrow\ (cv_1', \ldots, cv_m')$.
    If $\Gamma \vdash_{\cap CC} (v_1 : cv_1 \cap \ldots \cap cv_n)\ v_2 : T_{12} \cap \ldots \cap T_{n2}$, then $\Gamma \vdash_{\cap CC} v_1 : cv_1 \cap \ldots \cap cv_n : T_1 \cap \ldots \cap T_n$ with $\vdash_{\cap IC} cv_1 : T_1$ and ... and $\vdash_{\cap IC} cv_n : T_n$, such that $\exists i \in 1 .. n\ .\ T_i = T_{i1} \rightarrow T_{i2}$ and $\Gamma \vdash_{\cap CC} v_1 : T_1' \cap \ldots \cap T_l'$ and $I_1 = initialType(cv_1)$ and ... and $I_n = initialType(cv_n)$ such that either $T_1' \cap \ldots \cap T_l' =_{\cap} I_1 \cap \ldots \cap I_n$ or $unique\ types\ in\ [I_1, \ldots, I_n] \subset unique\ types\ in\ [T_1', \ldots, T_l']$ and $\Gamma \vdash_{\cap CC} v_2 : T_{11} \cap \ldots \cap T_{n1}$.
    For the sake of simplicity lets elide cast labels and blame labels.
    As $\vdash_{\cap IC} cv_1' : T_{11} \rightarrow T_{12}$ and ... and $\vdash_{\cap IC} cv_m' : T_{m1} \rightarrow T_{m2}$ then $cv_1' = cv_1'' : T_{11}' \rightarrow T_{12}' \Rightarrow T_{11} \rightarrow T_{12}$ and ... and $cv_m' = cv_m'' : T_{m1}' \rightarrow T_{m2}' \Rightarrow T_{m1} \rightarrow T_{m2}$.
    By the definition of simulateArrow, $c_{11} : \emptycast{T_{11}}{} : T_{11} \Rightarrow T_{11}'$ and ... and $c_{m1} = \emptycast{T_{m1}}{} : T_{m1} \Rightarrow T_{m1}'$ and $c_{12} : \emptycast{T_{12}'}{} : T_{12}' \Rightarrow T_{12}$ and ... and $c_{m2} = \emptycast{T_{m2}'}{} : T_{m2}' \Rightarrow T_{m2}$ and $initialType(r_1) = I_1$ and ... and $initialType(r_m) = I_m$ and $\vdash_{\cap IC} r_1 : T_{11}' \rightarrow T_{12}'$ and ... and $\vdash_{\cap IC} r_m : T_{m1}' \rightarrow T_{m2}'$.
    Therefore $\Gamma \vdash_{\cap CC} v_1 : r_1 \cap \ldots \cap r_m : T_{11}' \rightarrow T_{12}' \cap \ldots \cap T_{m1}' \rightarrow T_{m2}'$ and $\Gamma \vdash_{\cap CC} v_2 : c_{11} \cap \ldots \cap c_{m1} : T_{11}' \cap \ldots \cap T_{m1}'$ and therefore $\Gamma \vdash_{\cap CC} (v_1 : r_1 \cap \ldots \cap r_m)\ (v_2 : c_{11} \cap \ldots \cap c_{m1}) : T_{12}' \cap \ldots \cap T_{m2}'$.
    Therefore, $\Gamma \vdash_{\cap CC} (v_1 : r_1 \cap \ldots \cap r_m)\ (v_2 : c_{11} \cap \ldots \cap c_{m1}) : c_{12} \cap \ldots \cap c_{m2} : T_{12} \cap \ldots \cap T_{m2}$, such that $unique\ types\ in\ [T_{12}, \ldots, T_{m2}] \subset unique\ types\ in\ [T_{12}, \ldots, T_{n2}]$.
    By rule Simulate$\cap$, $(v_1 : cv_1 \cap \ldots \cap cv_n)\ v_2 \longrightarrow_{\cap CC}\\ (v_1 : r_1 \cap \ldots \cap r_m)\ (v_2 : c_{11} \cap \ldots \cap c_{m1}) : c_{12} \cap \ldots \cap c_{m2}$, therefore it is proved.
    \item e = $v : cv_1 \cap \ldots \cap cv_n : T_1 \Rightarrow^l T_2$ and $isValue\ v : cv_1 \cap \ldots \cap cv_n$ and $v : c_1' \cap \ldots \cap c_m' = mergeIC(v : cv_1 \cap \ldots \cap cv_n : T_1 \Rightarrow^l T_2)$.
    If $\Gamma \vdash_{\cap CC} v : cv_1 \cap \ldots \cap cv_n : T_1 \Rightarrow^l T_2 : T_2$ then $\Gamma \vdash_{\cap CC} v : cv_1 \cap \ldots \cap cv_n : T_{11} \cap \ldots \cap T_{1n}$ and such that $\exists T_{1i}\ .\ T_{1i} = T_1$ and $\vdash_{\cap IC} cv_1 : T_{11}$ and $I_1 = initialType(cv_1)$ and ... and $\vdash_{\cap IC} cv_n : T_{1n}$ and $I_n = initialType(cv_n)$ and $\Gamma \vdash_{\cap CC} v : I_1 \cap \ldots \cap I_n$ and $m \leq n$.
    By the definition of mergeIC, $\vdash_{\cap IC} c_1' : T_2$ and $initialType(c_1') : I_1$ and ... and $\vdash_{\cap IC} c_m' : T_2$ and $initialType(c_m') : I_m$.
    As $\Gamma \vdash_{\cap CC} v : I_1 \cap \ldots \cap I_m$ and therefore $\Gamma \vdash_{\cap CC} v : c_1' \cap \ldots \cap c_m' : T_2 \cap \ldots \cap T_2$ and $T_2 \cap \ldots \cap T_2 =_{\cap} T_2$ and by rule MergeIC$\cap$, $v : cv_1 \cap \ldots \cap cv_n : T_1 \Rightarrow^l T_2 \longrightarrow_{\cap CC} v : c_1' \cap \ldots \cap c_m'$, then it is proved.
    \item e = $v : T_1 \Rightarrow^l T_2 : c_1 \cap \ldots \cap c_n$ and $isValue\ v : T_1 \Rightarrow^l T_2$ and $v : c_1' \cap \ldots \cap c_m' = mergeCI(v : T_1 \Rightarrow^l T_2 : c_1 \cap \ldots \cap c_n)$.
    If $\Gamma \vdash_{\cap CC} v : T_1 \Rightarrow^l T_2 : c_1 \cap \ldots \cap c_n : F_1 \cap \ldots \cap F_n$ then $\Gamma \vdash_{\cap CC} v : T_1 \Rightarrow T_2 : T_2$ and $\Gamma \vdash_{\cap CC} v : T_1$ and $\vdash_{\cap IC} c_1 : F_1$ and $initialType(c_1) = T_2$ and ... and $\vdash_{\cap IC} c_n : F_n$ and $initialType(c_n) = T_2$.
    By the definition of mergeCI, $mergeCI(v : T_1 \Rightarrow^l T_2 : c_1 \cap \ldots \cap c_n) = v : c_1' \cap \ldots \cap c_n'$, such that $\vdash_{\cap IC} c_1' : F_1$ and $initialType(c_1') : T_1$ and ... and $\vdash_{\cap IC} c_n' : F_n$ and $initialType(c_n') : T_1$.
    As $\Gamma \vdash_{\cap CC} v : c_1' \cap \ldots \cap c_n' : F_1 \cap \ldots \cap F_n$ and by rule MergeCI$\cap$, $v : T_1 \Rightarrow^l T_2 : c_1 \cap \ldots \cap c_n \longrightarrow_{\cap CC} v : c_1' \cap \ldots \cap c_n'$, then it is proved.
    \item e = $v : cv_1 \cap \ldots \cap cv_n : c_1' \cap \ldots \cap c_m'$ and $isValue\ v : cv_1 \cap \ldots \cap cv_n$ and $v : c_1'' \cap \ldots \cap c_j'' = mergeII(v : cv_1 \cap \ldots \cap cv_n : c_1' \cap \ldots \cap c_m')$.
    If $\Gamma \vdash_{\cap CC} v : cv_1 \cap \ldots \cap cv_n : c_1' \cap \ldots \cap c_m' : F_1' \cap \ldots \cap F_m'$ then $\vdash_{\cap IC} c_1' : F_1'$ and $initialType(c_1') = I_1'$ and ... and $\vdash_{\cap IC} c_m' : F_m'$ and $initialType(c_m') = I_m'$ and $\Gamma \vdash_{\cap CC} v : cv_1 \cap \ldots \cap cv_n : F_1 \cap \ldots \cap F_n$ and $\vdash_{\cap IC} cv_1 : F_1$ and $initialType(cv_1) = I_1$ and ... and $\vdash_{\cap IC} cv_n : F_n$ and $initialType(cv_n) = I_n$ and $\Gamma \vdash_{\cap CC} v : T_1 \cap \ldots \cap T_l$ such that either $T_1 \cap \ldots \cap T_l =_{\cap} I_1 \cap \ldots \cap I_n$ or $unique\ types\ in\ [I_1, \ldots, I_n] \subset unique\ types\ in\ [T_1, \ldots, T_l]$.
    There are two possibilities:
    \begin{itemize}
        \item $F_1 \cap \ldots \cap F_n =_{\cap} I_1' \cap \ldots \cap I_m'$.
        By the definition of mergeII, $\vdash_{\cap IC} c_1'' : F_1''$ and ... and $\vdash_{\cap IC} c_j'' : F_j''$ such that $F_1'' \cap \ldots \cap F_j'' =_{\cap} F_1' \cap \ldots \cap F_m'$ and $initialType(c_1'') = I_1''$ and ... and $initialType(c_j'') = I_j''$ such that $I_1'' \cap \ldots \cap I_j'' =_{\cap} I_1 \cap \ldots \cap I_n$.
        Therefore $\Gamma \vdash_{\cap CC} v : c_1'' \cap \ldots \cap c_j'' : F_1'' \cap \ldots \cap F_j''$.
        By rule MergeII$\cap$, $v : cv_1 \cap \ldots \cap cv_n : c_1' \cap \ldots \cap c_m' \longrightarrow_{\cap CC} v : c_1'' \cap \ldots \cap c_j''$.
        Therefore it is proved.
        \item $unique\ types\ in\ [I_1', \ldots, I_m'] \subset unique\ types\ in\ [F_1, \ldots, F_n]$.
        By the definition of mergeII, $\vdash_{\cap IC} c_1'' : F_1''$ and $initialType(c_1'') = I_1''$ and ... and $\vdash_{\cap IC} c_j'' : F_j''$ and $initialType(c_j'') = I_j''$ such that $unique\ types\ in\ [I_1'', \ldots, I_j''] \subset unique\ types\ in\ [T_1, \ldots, T_l]$ and $unique\ types\ in\ [F_1'', \ldots, F_j''] \subset unique\ types\ in\ [F_1', \ldots, F_m']$.
        Therefore, $\Gamma \vdash_{\cap CC} v : c_1'' \cap \ldots \cap c_j'' : F_1'' \cap \ldots \cap F_j''$.
        By rule MergeII$\cap$, $v : cv_1 \cap \ldots \cap cv_n : c_1' \cap \ldots \cap c_m' \longrightarrow_{\cap CC} v : c_1'' \cap \ldots \cap c_j''$.
        As $\Gamma \vdash_{\cap CC} v : cv_1 \cap \ldots \cap cv_n : c_1' \cap \ldots \cap c_m' : F_1'' \cap \ldots \cap F_j''$, it is proved.
    \end{itemize}
    \item e = $v : c_1 \cap \ldots \cap c_n$ and $\neg(\forall i \in 1 .. n\ .\ isCastValue\ c_i)$.
    If $\Gamma \vdash_{\cap CC} v : c_1 \cap \ldots \cap c_n : T_1 \cap \ldots \cap T_n$ then $\vdash_{\cap IC} c_1 : T_1$ and ... and $\vdash_{\cap IC} c_n : T_n$, $\Gamma \vdash_{\cap CC} v : I_1 \cap \ldots \cap I_n$, with $I_1 = initialType(c_1)$ and ... and $I_n = initialType(c_n)$.
    By rule Evaluate$\cap$, $c_1 \longrightarrow_{\cap IC} cv_1$ and ... and $c_n \longrightarrow_{\cap IC} cv_n$.
    By Lemmas \ref{subjectreductionIC} and \ref{initialtypepreservationIC}, $\vdash_{\cap IC} cv_1 : T_1$ and ... and $\vdash_{\cap IC} cv_n : T_n$ and $initialType(cv_1) = I_1$ and ... and $initialType(cv_n) = I_n$.
    Therefore $\Gamma \vdash_{\cap CC} v : cv_1 \cap \ldots \cap cv_n : T_1 \cap \ldots \cap T_n$.
    \item e = $v : \blamecast{T_1'}{T_1}{l_1}{m_1} \cap \ldots \cap \blamecast{T_n'}{T_n}{l_n}{m_n}$.
    If $\Gamma \vdash_{\cap CC} v : \blamecast{T_1'}{T_1}{l_1}{m_1} \cap \ldots \cap \blamecast{T_n'}{T_n}{l_n}{m_n} : T_1 \cap \ldots \cap T_n$ and by rule PropagateBlame$\cap$ $v : \blamecast{T_1'}{T_1}{l_1}{m_1} \cap \ldots \cap \blamecast{T_n'}{T_n}{l_n}{m_n} \longrightarrow_{\cap CC} \blame{(T_1 \cap \ldots \cap T_n)}{l_1}$, and $\Gamma \vdash_{\cap CC} \blame{(T_1 \cap \ldots \cap T_n)}{l_1} : T_1 \cap \ldots \cap T_n$, then it is proved.
    \item e = $v : \emptycast{T_1}{m_1} \cap \ldots \cap \emptycast{T_n}{m_n}$.
    If $\Gamma \vdash_{\cap CC} v : \emptycast{T_1}{m_1} \cap \ldots \cap \emptycast{T_n}{m_n} : T_1 \cap \ldots \cap T_n$, then $\vdash_{\cap IC} \emptycast{T_1}{m_1} : T_1$ and $initialType(\emptycast{T_1}{m_1}) = T_1$ and ... and $\vdash_{\cap IC} \emptycast{T_n}{m_n} : T_n$ and $initialType(\emptycast{T_n}{m_n}) = T_n$ and $\Gamma \vdash_{\cap CC} v : T_1 \cap \ldots \cap T_n$.
    By rule RemoveEmpty$\cap$, $v : \emptycast{T_1}{m_1} \cap \ldots \cap \emptycast{T_n}{m_n} \longrightarrow_{\cap CC} v$, therefore it is proved.
\end{itemize}
Induction step:
\begin{itemize}
    \item e =
\end{itemize}
\end{proof}

\begin{lemma}[Consistency reduces to equality when comparing static types]
\label{consistencytoequality}
If $T_1$ and $T_2$ are static types then $T_1 = T_2 \iff T_1 \sim T_2$.
\end{lemma}
\begin{proof}
We proceed by structural induction on $T$.\\\\
Base cases:
\begin{itemize}
    \item $T_1 = \Int$.
    \begin{itemize}
        \item If $T_1 = T_2$, then by the definition of $\sim$, $T_1 \sim T_2$.
        \item If $T_1 \sim T_2$, then by the definition of $\sim$, $T_1 = T_2$.
    \end{itemize}
    \item $T_1 = \Bool$.
    \begin{itemize}
        \item If $T_1 = T_2$, then by the definition of $\sim$, $T_1 \sim T_2$.
        \item If $T_1 \sim T_2$, then by the definition of $\sim$, $T_1 = T_2$.
    \end{itemize}
    \item $T_1 = \Dyn$.
    This case is not considered due to the assumption that $T_1$ is a static type.
\end{itemize}
Induction step:
\begin{itemize}
    \item $T_1 = T_{11} \rightarrow T_{12}$.
    \begin{itemize}
        \item If $T_1 = T_2$, then $\exists T_{21}, T_{22}\ .\ T_2 = T_{21} \rightarrow T_{22}$ and $T_{11} = T_{21}$ and $T_{12} = T_{22}$.
        By the induction hypothesis, $T_{11} \sim T_{21}$ and $T_{12} \sim T_{22}$.
        Therefore, by the definition of $\sim$, $T_1 \sim T_2$.
        \item If $T_1 \sim T_2$, then $\exists T_{21}, T_{22}\ .\ T_2 = T_{21} \rightarrow T_{22}$ and $T_{11} = T_{21}$ and $T_{12} = T_{22}$.
        By the induction hypothesis, $T_{11} = T_{21}$ and $T_{12} = T_{22}$.
        Therefore, by the definition of $=$, $T_1 = T_2$.
    \end{itemize}
    \item $T_1 = T_{11} \cap \ldots \cap T_{1n}$.
    \begin{itemize}
        \item If $T_1 = T_2$, then $\exists T_{21}, \ldots, T_{2n}\ .\ T_2 = T_{21} \cap \ldots \cap T_{2n}$ and $T_{11} = T_{21}$ and ... and $T_{1n} = T_{2n}$.
        By the induction hypothesis, $T_{11} \sim T_{21}$ and ... and $T_{1n} \sim T_{2n}$.
        Therefore, by the definition of $\sim$, $T_1 \sim T_2$.
        \item If $T_1 \sim T_2$, then $\exists T_{21}, \ldots, T_{2n}\ .\ T_2 = T_{21} \cap \ldots \cap T_{2n}$ and $T_{11} \sim T_{21}$ and ... and $T_{1n} \sim T_{2n}$.
        By the induction hypothesis, $T_{11} = T_{21}$ and ... and $T_{1n} = T_{2n}$.
        Therefore, by the definition of $=$, $T_1 = T_2$.
    \end{itemize}
\end{itemize}
\end{proof}

\begin{lemma}[Subject reduction of $\longrightarrow_{\cap IC}$]
\label{subjectreductionIC}
If $\vdash_{\cap IC} c : T$ for some $T$ and $c \longrightarrow_{\cap IC} c'$ then $\vdash_{\cap IC} c' : T$.
\end{lemma}
\begin{proof}
We proceed by induction on the length of the derivation tree of $\longrightarrow_{\cap IC}$.\\\\
Base cases:
\begin{itemize}
    \item c = $\blamecast{T_I}{T_F}{l_1}{n_1} : T_1 \Rightarrow^{l_2} T_2\ ^{n_2}$.
    $\vdash_{\cap IC} \blamecast{T_I}{T_F}{l_1}{n_1} : T_1 \Rightarrow^{l_2} T_2\ ^{n_2} : T_2$ and by rule PushBlameC, $\blamecast{T_I}{T_F}{l_1}{n_1} : T_1 \Rightarrow^{l_2} T_2\ ^{n_2} \longrightarrow_{\cap IC} \blamecast{T_I}{T_2}{l_1}{n_1}$.
    As $\vdash_{\cap IC} \blamecast{T_I}{T_2}{l_1}{n_1} : T_2$, then it is proved.
    \begin{comment}
    \item c = $\stuckcast{T_I}{T_F}{n_1} : T_1 \Rightarrow^{l} T_2\ ^{n_2}$.
    $\vdash_{\cap IC} \stuckcast{T_I}{T_F}{n_1} : T_1 \Rightarrow^{l} T_2\ ^{n_2} : T_2$ and by rule PushStuckC, $\stuckcast{T_I}{T_F}he{n_1} : T_1 \Rightarrow^{l} T_2\ ^{n_2} \longrightarrow_{\cap IC} \stuckcast{T_I}{T_2}{n_1}$.
    As $\vdash_{\cap IC} \stuckcast{T_I}{T_2}{n_1} : T_2$, then it is proved.
    \end{comment}
    \item c = $c' : T \Rightarrow^l T\ ^n$ and $isCastValue1\ c'\ \lor\ isEmptyCast\ c'$.
    If $\vdash_{\cap IC} c' : T \Rightarrow^l T\ ^n : T$, then $\vdash_{\cap IC} c' : T$.
    By rule IdentityC, $c' : T \Rightarrow^l T\ ^n \longrightarrow_{\cap IC} c'$.
    Therefore it is proved.
    \item c = $c' : G \Rightarrow^{l_1} \Dyn\ ^{n_1} : \Dyn \Rightarrow^{l_2} G\ ^{n_2}$ and $isCastValue1\ c'\ \lor\ isEmptyCast\ c'$.
    If $\vdash_{\cap IC} c' : G \Rightarrow^{l_1} \Dyn\ ^{n_1} : \Dyn \Rightarrow^{l_2} G\ ^{n_2} : G$, then $\vdash_{\cap IC} c' : G$.
    By rule SucceedC, $c' : G \Rightarrow^{l_1} \Dyn\ ^{n_1} : \Dyn \Rightarrow^{l_2} G\ ^{n_2} \longrightarrow_{\cap IC} c'$.
    Therefore it is proved.
    \item c = $c' : G_1 \Rightarrow^{l_1} \Dyn\ ^{n_1} : \Dyn \Rightarrow^{l_2} G_2\ ^{n_2}$ and $isCastValue1\ c'\ \lor\ isEmptyCast\ c'$ and $\neg(same\ ground\ G_1\ G_2)$ and $initialType(c') = T_I$.
    If $\vdash_{\cap IC} c' : G_1 \Rightarrow^{l_1} \Dyn\ ^{n_1} : \Dyn \Rightarrow^{l_2} G_2\ ^{n_2} : G_2$, and by rule FailC, $c' : G_1 \Rightarrow^{l_1} \Dyn\ ^{n_1} : \Dyn \Rightarrow^{l_2} G_2\ ^{n_2} \longrightarrow_{\cap IC} \blamecast{T_I}{G_2}{l_2}{n_1}$ and $\vdash_{\cap IC} \blamecast{T_I}{G_2}{l_2}{n_1} : G_2$, it is proved.
    \item c = $c' : T \Rightarrow^l \Dyn\ ^n$ and $isCastValue1\ c'\ \lor\ isEmptyCast\ c'$ and $G\ is\ ground\ type\ of\ T$ and $\neg(ground\ T)$.
    If $\vdash_{\cap IC} c' : T \Rightarrow^l \Dyn\ ^n : \Dyn$ then $\vdash_{\cap IC} c' : T$.
    By rule GroundC, $c' : T \Rightarrow^l \Dyn\ ^n \longrightarrow_{\cap IC} c' : T \Rightarrow^l G\ ^n : G \Rightarrow^l \Dyn\ ^n$.
    As $\vdash_{\cap IC} c' : T \Rightarrow^l G\ ^n : G \Rightarrow^l \Dyn\ ^n : \Dyn$, it is proved.
    \item c = $c' : \Dyn \Rightarrow^l T\ ^n$ and $isCastValue1\ c'\ \lor\ isEmptyCast\ c'$ and $G\ is\ ground\ type\ of\ T$ and $\neg(ground\ T)$.
    If $\vdash_{\cap IC} c' : \Dyn \Rightarrow^l T\ ^n : T$ then $\vdash_{\cap IC} c' : \Dyn$.
    By rule ExpandC, $c' : \Dyn \Rightarrow^l T\ ^n \longrightarrow_{\cap IC} c' : \Dyn \Rightarrow^l G\ ^n : G \Rightarrow^l T\ ^n$.
    As $\vdash_{\cap IC} c' : \Dyn \Rightarrow^l G\ ^n : G \Rightarrow^l T\ ^n : T$, it is proved.
    \begin{comment}
    \item c = $c' : T_1 \Rightarrow^l T_2\ ^n$ and $isCastValue1\ c'\ \lor\ isEmptyCast\ c'$ and $initialType(c) = T_I$.
    If $\vdash_{\cap IC} c' : T_1 \Rightarrow^l T_2\ ^n : T_2$, and by rule TriggerStuckC, $c' : T_1 \Rightarrow^l T_2\ ^n \longrightarrow_{\cap IC} \stuckcast{T_I}{T_2}{n}$, then $\vdash_{\cap IC} \stuckcast{T_I}{T_2}{n} : T_2$.
    \end{comment}
\end{itemize}
Induction step:
\begin{itemize}
    \item c = $c' : T_1 \Rightarrow^l T_2\ ^n$ and $\neg(isCastValue\ c)$.
    If $\vdash_{\cap IC} c' : T_1 \Rightarrow^l T_2\ ^n : T_2$ then $\vdash_{\cap IC} c' : T_1$.
    By rule EvaluateC, $c' \longrightarrow_{\cap IC} c''$.
    By the induction hypothesis, $\vdash_{\cap IC} c'' : T_1$.
    By rule EvaluateC, $c' : T_1 \Rightarrow^l T_2\ ^n \longrightarrow_{\cap IC} c'' : T_1 \Rightarrow^l T_2\ ^n$.
    As $\vdash_{\cap IC} c'' : T_1 \Rightarrow^l T_2\ ^n : T_2$ it is proved.
\end{itemize}
\end{proof}

\begin{lemma}[Initial type preservation of $\longrightarrow_{\cap IC}$]
\label{initialtypepreservationIC}
If $initialType(c) = T$ for some $T$ and $c \longrightarrow_{\cap IC} c'$ then $initialType(c') = T$.
\end{lemma}
\begin{proof}
We proceed by induction on the length of the derivation tree of $\longrightarrow_{\cap IC}$.\\\\
Base cases:
\begin{itemize}
    \item c = $\blamecast{T_I}{T_F}{l_1}{n_1} : T_1 \Rightarrow^{l_2} T_2\ ^{n_2}$.
    By the definition of $initialType$, $initialType(\blamecast{T_I}{T_F}{l_1}{n_1} : T_1 \Rightarrow^{l_2} T_2\ ^{n_2}) = T_I$.
    By rule PushBlameC, $\blamecast{T_I}{T_F}{l_1}{n_1} : T_1 \Rightarrow^{l_2} T_2\ ^{n_2} \longrightarrow_{\cap IC} \blamecast{T_I}{T_2}{l_1}{n_1}$.
    Since $initialType(\blamecast{T_I}{T_2}{l_1}{n_1}) = T_I$, it is proved.
    \begin{comment}
    \item c = $\stuckcast{T_I}{T_F}{n_1} : T_1 \Rightarrow^{l} T_2\ ^{n_2}$.
    By the definition of $initialType$, $initialType(\stuckcast{T_I}{T_F}{n_1} : T_1 \Rightarrow^{l} T_2\ ^{n_2}) = T_I$.
    By rule PushStuckC, $\stuckcast{T_I}{T_F}{n_1} : T_1 \Rightarrow^{l} T_2\ ^{n_2} \longrightarrow_{\cap IC} \stuckcast{T_I}{T_2}{n_1}$.
    Since $initialType(\stuckcast{T_I}{T_2}{n_1}) = T_I$, it is proved.
    \end{comment}
    \item c = $c' : T \Rightarrow^l T\ ^n$ and $isCastValue1\ c'\ \lor\ isEmptyCast\ c'$.
    By the definitions of $initialType$, $initialType(c' : T \Rightarrow^l T\ ^n) = initialType(c')$.
    By rule IdentityC, $c' : T \Rightarrow^l T\ ^n \longrightarrow_{\cap IC} c'$.
    Therefore it is proved.
    \item c = $c' : G \Rightarrow^{l_1} \Dyn\ ^{n_1} : \Dyn \Rightarrow^{l_2} G\ ^{n_2}$ and $isCastValue1\ c'\ \lor\ isEmptyCast\ c'$.
    By the definition of $initialType$, $initialType(c' : G \Rightarrow^{l_1} \Dyn\ ^{n_1} : \Dyn \Rightarrow^{l_2} G\ ^{n_2}) = initialType(c')$.
    By rule SucceedC, $c' : G \Rightarrow^{l_1} \Dyn\ ^{n_1} : \Dyn \Rightarrow^{l_2} G\ ^{n_2} \longrightarrow_{\cap IC} c'$.
    Therefore it is proved.
    \item c = $c' : G_1 \Rightarrow^{l_1} \Dyn\ ^{n_1} : \Dyn \Rightarrow^{l_2} G_2\ ^{n_2}$ and $isCastValue1\ c'\ \lor\ isEmptyCast\ c'$ and $\neg(same\ ground\ G_1\ G_2)$ and $initialType(c') = T_I$.
    By the definition of $initialType$, $initialType(c' : G_1 \Rightarrow^{l_1} \Dyn\ ^{n_1} : \Dyn \Rightarrow^{l_2} G_2\ ^{n_2}) = T_I$.
    By rule FailC, $c' : G_1 \Rightarrow^{l_1} \Dyn\ ^{n_1} : \Dyn \Rightarrow^{l_2} G_2\ ^{n_2} \longrightarrow_{\cap IC} \blamecast{T_I}{G_2}{l_2}{n_1}$.
    Since $initialType(\blamecast{T_I}{G_2}{l_2}{n_1}) = T_I$, it is proved.
    \item c = $c' : T \Rightarrow^l \Dyn\ ^n$ and $isCastValue1\ c'\ \lor\ isEmptyCast\ c'$ and $G\ is\ ground\ type\ of\ T$ and $\neg(ground\ T)$.
    By the definition of $initialType$, $initialType(c' : T \Rightarrow^l \Dyn\ ^n) = initialType(c')$.
    By rule GroundC, $c' : T \Rightarrow^l \Dyn\ ^n \longrightarrow_{\cap IC} c' : T \Rightarrow^l G\ ^n : G \Rightarrow^l \Dyn\ ^n$.
    Since $initialType(c' : T \Rightarrow^l G\ ^n : G \Rightarrow^l \Dyn\ ^n) = initialType(c')$, it is proved.
    \item c = $c' : \Dyn \Rightarrow^l T\ ^n$ and $isCastValue1\ c'\ \lor\ isEmptyCast\ c'$ and $G\ is\ ground\ type\ of\ T$ and $\neg(ground\ T)$.
    By the definition of $initialType$, $initialType(c' : \Dyn \Rightarrow^l T\ ^n) = initialType(c')$.
    By rule ExpandC, $c' : \Dyn \Rightarrow^l T\ ^n \longrightarrow_{\cap IC} c' : \Dyn \Rightarrow^l G\ ^n : G \Rightarrow^l T\ ^n$.
    Since $initialType(c' : \Dyn \Rightarrow^l G\ ^n : G \Rightarrow^l T\ ^n) = initialType(c')$, it is proved.
    \begin{comment}
    \item c = $c' : T_1 \Rightarrow^l T_2\ ^n$ and $isCastValue1\ c'\ \lor\ isEmptyCast\ c'$ and $initialType(c') = T_I$.
    By the definition of $initialType$, $initialType(c' : T_1 \Rightarrow^l T_2\ ^n) = T_I$.
    By rule TriggerStuckC, $c' : T_1 \Rightarrow^l T_2\ ^n \longrightarrow_{\cap IC} \stuckcast{T_I}{T_2}{n}$.
    Since $initialType(\stuckcast{T_I}{T_2}{n}) = T_I$, it is proved.
    \end{comment}
\end{itemize}
Induction step:
\begin{itemize}
    \item c = $c' : T_1 \Rightarrow^l T_2\ ^n$ and $\neg(isCastValue\ c')$.
    By the definition of $initialType$, $initialType(c' : T_1 \Rightarrow^l T_2\ ^n) = initialType(c')$.
    By rule EvaluateC, $c' \longrightarrow_{\cap IC} c''$.
    By the induction hypothesis, $initialType(c'') = initialType(c')$.
    By rule EvaluateC, $c' : T_1 \Rightarrow^l T_2\ ^n \longrightarrow_{\cap IC} c'' : T_1 \Rightarrow^l T_2\ ^n$.
    Since $initialType(c'' : T_1 \Rightarrow^l T_2\ ^n) = initialType(c'')$, it is proved.
\end{itemize}
\end{proof}

\begin{lemma}[Expressions annotated with only static types type with static types]
\label{staticexpressionstypestatic}
If $e$ is annotated with only static types then:
\begin{enumerate}
    \item $\Gamma \vdash_{\cap G} e : T$, for some static $T$.
    \item $\Gamma \vdash_{\cap CC} e \leadsto e' : T$, for some static $T$.
\end{enumerate}
\end{lemma}
\begin{proof}
(1) We proceed by induction on the length of the derivation tree of $\vdash_{\cap G}$.\\\\
Base cases:
\begin{itemize}
    \item e = $x$.
    If $\Gamma \vdash_{\cap G} x : T_i$, then there is a binding $x : T' \in \Gamma$, such that $T_i \subseteq T'$.
    Therefore, there must have been at some point in the typing derivation, the application of the rules $({\rightarrow} I)$ or $({\rightarrow} I')$. If $e$ is annotated with only static types, then both rules introduze the binding $x : T'$ in $\Gamma$, such that $T'$ is a static type. Therefore, $T_i$ is also a static type.
\end{itemize}
Induction step:
\begin{itemize}
    \item e = $\lambda x : T_1 \cap \ldots \cap T_n\ .\ e'$.
    There are two possibilities:
    \begin{itemize}
        \item Using the rule ${\rightarrow} I$.
        If $e$ is annotated with only static types, then $T_1 \cap \ldots \cap T_n$ is a static type.
        By rule $({\rightarrow} I)$, $\Gamma, x : T_1 \cap \ldots \cap T_n \vdash_{\cap G} e : T$.
        By the induction hypothesis, $T$ is a static type.
        Therefore $T_1 \cap \ldots \cap T_n \rightarrow T$ is a static type.
        \item Using the rule ${\rightarrow} I'$.
        If $e$ is annotated with only static types, then $T_1 \cap \ldots \cap T_n$ is a static type.
        By rule $({\rightarrow} I')$, $\Gamma, x : T_i \vdash_{\cap G} e : T$.
        Since $T_1 \cap \ldots \cap T_n$ is a static type, then so is $T_i$.
        By the induction hypothesis, $T$ is a static type, therefore so is $T_i \rightarrow T$.
    \end{itemize}
    \item e = $e_1\ e_2$.
    If $e$ is annotated with only static types, then so are $e_1$ and $e_2$.
    By the induction hypothesis, $PM$ is a static type.
    By the definition of $\rhd$, $T_1 \cap \ldots \cap T_n \rightarrow T$ is also a static type.
    Therefore, $T$ is a static type.
    \item e = $e$.
    If $e$ annotated with only static types, then by the induction hypothesis, $T_1 \ldots T_n$ are static types.
    Therefore $T_1 \cap \ldots \cap T_n$ is a static type.
\end{itemize}
(2) We proceed by induction on the length of the derivation tree of $\Gamma \vdash_{\cap CC} e \leadsto e : T$.\\\\
Base cases:
\begin{itemize}
    \item e = $x$.
    If $\Gamma \vdash_{\cap CC} x \leadsto x : T_i$, then there is a binding $x : T' \in \Gamma$, such that $T_i \subseteq T'$.
    Therefore, there must have been at some point in the typing derivation, the application of the rule for the term $\lambda x : T_1 \cap \ldots \cap T_n\ .\ e'$.
    If $e$ is annotated with only static types, then the rule introduzes the binding $x : T'$ in $\Gamma$, such that $T'$ is a static type.
    Therefore, $T_i$ is also a static type.
\end{itemize}
Induction step:
\begin{itemize}
    \item e = $\lambda x : T_1 \cap \ldots \cap T_n\ .\ e'$.
    If $e$ is annotated with only static types, then $T_1 \cap \ldots \cap T_n$ is a static type.
    By the induction hypothesis, $T$ is a static type.
    Therefore $T_1 \cap \ldots \cap T_n \rightarrow T$ is a static type.
    \item e = $e_1\ e_2$.
    If $e$ is annotated with only static types, then so are $e_1$ and $e_2$.
    By the induction hypothesis, $PM$ is a static type.
    By the definition of $\rhd$, $T_1 \cap \ldots \cap T_n \rightarrow T$ is also a static type.
    Therefore, $T$ is a static type.
\end{itemize}
\end{proof}

\begin{lemma}[Static program compilation only adds identity casts]
\label{staticcompilation}
Depends on Lemmas \ref{consistencytoequality} and \ref{staticexpressionstypestatic}.
If $e$ is annotated with only static types and $\Gamma \vdash_{\cap CC} e \leadsto e' : T$, then any casts $e'$ contains are identity casts.\\
By identity casts, we mean casts of the form $e : T \Rightarrow^l T$ for some $T$ and casts $e : c_1 \cap \ldots \cap c_n$ such that $c_1 = \emptycast{T_1}{0} : T_1 \Rightarrow T_1\ ^{0}$ and ... and $c_n = \emptycast{T_n}{0} : T_n \Rightarrow T_n\ ^{0}$ for some $T_1,\ \ldots,\ T_n$.
\end{lemma}
\begin{proof}
We proceed by structural induction on $e$.\\\\
Base cases:
\begin{itemize}
    \item e = $x$.
    As $\Gamma \vdash_{\cap CC} x \leadsto x : T_i$, and $x$ doesn't have any casts, then it is proved.
\end{itemize}
Induction step:
\begin{itemize}
    \item e = $\lambda x : T_1 \cap \ldots \cap T_n\ .\ e'$.
    By rule, $\Gamma \vdash_{\cap CC} e' \leadsto e'' : T$.
    By the induction hypothesis, $e''$ either doesn't contain casts or contains only identity casts.
    By rule, $\Gamma \vdash_{\cap CC} (\lambda x : T_1 \cap \ldots \cap T_n\ .\ e') \leadsto (\lambda x : T_1 \cap \ldots \cap T_n\ .\ e'') : T_1 \cap \ldots \cap T_n \rightarrow T$.
    As the rule doesn't introduze new casts, then it is proved.
    \item e = $e_1\ e_2$.
    By rule, $\Gamma \vdash_{\cap CC} e_1 \leadsto e_1' : PM$ and $\Gamma \vdash_{\cap CC} e_2 \leadsto e_2' : T_1' \cap \ldots \cap T_n'$.
    By the induction hypothesis, both $e_1'$ as well as $e_2'$ either only have identity casts or no casts at all.
    By Lemma \ref{staticexpressionstypestatic}, $PM$ and $T_1' \cap \ldots \cap T_n'$ are static types.
    Therefore, by the definition of $\rhd$, $PM = T_1' \cap \ldots \cap T_n' \rightarrow T$ and by Lemma \ref{consistencytoequality}, $T_1' \cap \ldots \cap T_n' = T_1 \cap \ldots \cap T_n$.
    Therefore by the definition of $getInstances$ and $addCasts$, only identity casts are introduzed.
\end{itemize}
\end{proof}

\begin{lemma}[Elimination of identity casts in c]
\label{eliminationidentitycastsc}
For any cast $c$, such that $\vdash_{\cap IC} c : T_F$, $initialType(c) = T_I$ and $c \longrightarrow_{\cap IC} cv$:
\begin{enumerate}
    \item $\vdash_{\cap IC} \erasec{c} : T_F$ and $initialType(\erasec{c}) = T_I$.
    \item $\erasec{c} \longrightarrow_{\cap IC} cv$.
\end{enumerate}
\end{lemma}
\begin{proof}
(1) We proceed by structural induction on $c$.\\\\
Base cases:
\begin{itemize}
    \item c = $\emptycast{T}{n}$.
    As $\vdash_{\cap IC} \emptycast{T}{n} : T$, $initialType(\emptycast{T}{n}) = T$ and $\erasec{c} = \emptycast{T}{n}$, then $\vdash_{\cap IC} \erasec{c} : T$ and $initialType(\erasec{c}) = T$.
    \item c = $\blamecast{T_I}{T_F}{l}{n}$.
    As $\vdash_{\cap IC} \blamecast{T_I}{T_F}{l}{n} : T_F$, $initialType(\blamecast{T_I}{T_F}{l}{n}) = T_I$ and $\erasec{c} = \blamecast{T_I}{T_F}{l}{n}$, then $\vdash_{\cap IC} \erasec{c} : T_F$ and $initialType(\erasec{c}) = T_I$.
    \begin{comment}
    \item c = $\stuckcast{T_I}{T_F}{n}$.
    As $\vdash_{\cap IC} \stuckcast{T_I}{T_F}{n} : T_F$, $initialType(\stuckcast{T_I}{T_F}{n}) = T_I$ and $\erasec{c} = \stuckcast{T_I}{T_F}{n}$, then $\vdash_{\cap IC} \erasec{c} : T_F$ and $initialType(\erasec{c}) = T_I$.
    \end{comment}
\end{itemize}
Induction step:
\begin{itemize}
    \item c = $c' : T_1 \Rightarrow^l T_2\ ^n$.
    There are two cases:
    \begin{itemize}
        \item $T_1 \neq T_2$.
        As $\vdash_{\cap IC} c' : T_1 \Rightarrow^l T_2\ ^n : T_2$ and $initialType(c' : T_1 \Rightarrow^l T_2\ ^n) = initialType(c')$, then $\vdash_{\cap IC} c' : T_1$.
        By the induction hypothesis, $\vdash_{\cap IC} \erasec{c'} : T_1$ and $initialType(\erasec{c'}) = initialType(c')$.
        With $\erasec{c} = \erasec{c'} : T_1 \Rightarrow^l T_2\ ^n$, $\vdash_{\cap IC} \erasec{c} : T_2$ and $initialType(\erasec{c}) = initialType(\erasec{c'}) = initialType(c') = initialType(c)$.
        \item $T_1 = T_2$.
        As $\vdash_{\cap IC} c' : T_1 \Rightarrow^l T_1\ ^n : T_1$ and $initialType(c' : T_1 \Rightarrow^l T_1\ ^n) = initialType(c')$ then $\vdash_{\cap IC} c' : T_1$.
        By the induction hypothesis, $\vdash_{\cap IC} \erasec{c'} : T_1$ and $initialType(\erasec{c'}) = initialType(c')$.
        With $\erasec{c} = \erasec{c'}$, $\vdash_{\cap IC} \erasec{c} : T_1$ and $initialType(\erasec{c}) = initialType(\erasec{c'}) = initialType(c') = initialType(c)$.
    \end{itemize}
\end{itemize}
(2) We proceed by structural induction on $c$.\\\\
Base cases:
\begin{itemize}
    \item c = $\blamecast{T_I}{T_F}{l_1}{n_1} : T_1 \Rightarrow^{l_2} T_2\ ^{n_2}$.
    There are two cases:
    \begin{itemize}
        \item $T_1 \neq T_2$.
        As $\erasec{c} = \blamecast{T_I}{T_F}{l_1}{n_1} : T_1 \Rightarrow^{l_2} T_2\ ^{n_2}$ and by rule PushBlameC, $\blamecast{T_I}{T_F}{l_1}{n_1} : T_1 \Rightarrow^{l_2} T_2\ ^{n_2} \longrightarrow_{\cap IC} \blamecast{T_I}{T_2}{l_1}{n_1}$ it is proved.
        \item $T_1 = T_2$.
        If $T_1 = T_2$, then by rules T-SingleC and T-BlameC, $T_F = T_1$.
        Therefore, $c = \blamecast{T_I}{T_1}{l_1}{n_1} : T_1 \Rightarrow^{l_2} T_1\ ^{n_2}$.
        By rule PushBlameC, $\blamecast{T_I}{T_1}{l_1}{n_1} : T_1 \Rightarrow^{l_2} T_1\ ^{n_2} \longrightarrow_{\cap IC} \blamecast{T_I}{T_1}{l_1}{n_1}$.
        Since $\erasec{c} = \blamecast{T_I}{T_1}{l_1}{n_1}$, and it is already a value, it is proved.
    \end{itemize}
    \begin{comment}
    \item c = $\stuckcast{T_I}{T_F}{n_1} : T_1 \Rightarrow^l T_2\ ^{n_2}$.
    There are two cases:
    \begin{itemize}
        \item $T_1 \neq T_2$.
        As $\erasec{c} = \stuckcast{T_I}{T_F}{n_1} : T_1 \Rightarrow^l T_2\ ^{n_2}$ and by rule PushStuckC, $\stuckcast{T_I}{T_F}{n_1} : T_1 \Rightarrow^l T_2\ ^{n_2} \longrightarrow_{\cap IC} \stuckcast{T_I}{T_2}{n_1}$ it is proved.
        \item $T_1 = T_2$.
        If $T_1 = T_2$, then by rules T-SingleC and T-StuckC, $T_F = T_1$.
        Therefore, $c = \stuckcast{T_I}{T_1}{n_1} : T_1 \Rightarrow^l T_1\ ^{n_2}$.
        By rule PushStuckC, $\stuckcast{T_I}{T_1}{n_1} : T_1 \Rightarrow^l T_1\ ^{n_2} \longrightarrow_{\cap IC} \stuckcast{T_I}{T_1}{n_1}$.
        Since $\erasec{c} = \stuckcast{T_I}{T_1}{n_1}$, and it is already a value, it is proved.
    \end{itemize}
    \end{comment}
    \item c = $c' : T \Rightarrow^l T\ ^n$ and $isCastValue1\ c'\ \lor\ isEmptyCast\ c'$.
    By rule IdentityC, $c' : T \Rightarrow^l T\ ^n \longrightarrow_{\cap IC} c'$.
    As $c'$ is a value, it doesn't contain identity casts, therefore $\erasec{c} = c'$.
    As $\erasec{c}$ is already a value, it reduces to itself, therefore it is proved.
    \item c = $c' : G \Rightarrow^{l_1} \Dyn\ ^{n_1} : \Dyn \Rightarrow^{l_2} G\ ^{n_2}$ and $isCastValue1\ c'\ \lor\ isEmptyCast\ c'$.
    By rule SucceedC, $c' : G \Rightarrow^{l_1} \Dyn\ ^{n_1} : \Dyn \Rightarrow^{l_2} G\ ^{n_2} \longrightarrow_{\cap IC} c'$.
    As $c'$ is already a value, then it doesn't contain identity casts, so $\erasec{c} = c' : G \Rightarrow^{l_1} \Dyn\ ^{n_1} : \Dyn \Rightarrow^{l_2} G\ ^{n_2}$.
    Therefore, $\erasec{c} \longrightarrow_{\cap IC} c'$.
    \item c = $c' : G_1 \Rightarrow^{l_1} \Dyn\ ^{n_1} : \Dyn \Rightarrow^{l_2} G_2\ ^{n_2}$ and $isCastValue1\ c'\ \lor\ isEmptyCast\ c'$ and $\neg(same\ ground\ G_1\ G_2)$ and $initialType(c') = T_I$.
    By rule FailC, $c' : G_1 \Rightarrow^{l_1} \Dyn\ ^{n_1} : \Dyn \Rightarrow^{l_2} G_2\ ^{n_2} \longrightarrow_{\cap IC} \blamecast{T_I}{G_2}{l_2}{n_1}$.
    As $c'$ is already a value, then it doesn't contain identity casts, so $\erasec{c} = c' : G_1 \Rightarrow^{l_1} \Dyn\ ^{n_1} : \Dyn \Rightarrow^{l_2} G_2\ ^{n_2}$.
    Therefore, $\erasec{c} \longrightarrow_{\cap IC} \blamecast{T_I}{G_2}{l_2}{n_1}$.
    \item c = $c' : T \Rightarrow^l \Dyn\ ^n$ and $isCastValue1\ c'\ \lor\ isEmptyCast\ c'$ and $G\ is\ ground\ type\ of\ T$ and $\neg(ground\ T)$.
    By rule GroundC, $c' : T \Rightarrow^l \Dyn\ ^n \longrightarrow_{\cap IC} c' : T \Rightarrow^l G : G \Rightarrow^l \Dyn\ ^n$.
    As $c'$ is a value, it doesn't contain identity casts, therefore $\erasec{c} = c' : T \Rightarrow^l \Dyn\ ^n$.
    Therefore $\erasec{c} \longrightarrow_{\cap IC} c' : T \Rightarrow^l G : G \Rightarrow^l \Dyn\ ^n$.
    \item c = $c' : \Dyn \Rightarrow^l T\ ^n$ and $isCastValue1\ c'\ \lor\ isEmptyCast\ c'$ and $G\ is\ ground\ type\ of\ T$ and $\neg(ground\ T)$.
    By rule ExpandC, $c' : \Dyn \Rightarrow^l T\ ^n \longrightarrow_{\cap IC} c' : \Dyn \Rightarrow^l G : G \Rightarrow^l T\ ^n$.
    As $c'$ is a value, it doesn't contain identity casts, therefore $\erasec{c} = c' : \Dyn \Rightarrow^l T\ ^n$.
    Therefore $\erasec{c} \longrightarrow_{\cap IC} c' : \Dyn \Rightarrow^l G : G \Rightarrow^l T\ ^n$.
    \begin{comment}
    \item c = $c' : T_1 \Rightarrow^l T_2\ ^n$ and $isCastValue1\ c'\ \lor\ isEmptyCast\ c'$ and $initialType(c') = T_I$.
    By rule TriggerStuckC, $c' : T_1 \Rightarrow^l T_2\ ^n \longrightarrow_{\cap IC} \stuckcast{T_I}{T_2}{n}$.
    As $T_1 \neq T_2$ and $c'$ is a value, then $\erasec{c} = c' : T_1 \Rightarrow^l T_2\ ^n$.
    Therefore, $\erasec{c} \longrightarrow_{\cap IC} \stuckcast{T_I}{T_2}{n}$.
    \end{comment}
\end{itemize}
Induction step:
\begin{itemize}
    \item c = $c' : T_1 \Rightarrow^l T_2\ ^{n}$ and $\neg(isCastValue c')$.
    There are two cases:
    \begin{itemize}
        \item $T_1 \neq T_2$.
        By rule EvaluateC, $c' \longrightarrow_{\cap IC} c''$.
        By the induction hypothesis, $\erasec{c'} \longrightarrow_{\cap IC} c''$.
        As $\erasec{c}$ equals $\erasec{c'} : T_1 \Rightarrow^l T_2\ ^n$, then by rule EvaluateC, $\erasec{c} \longrightarrow_{\cap IC} c'' : T_1 \Rightarrow T_2\ ^n$.
        \item $T_1 = T_2$.
        By the induction hypothesis, as $c' \longrightarrow_{\cap IC} cv'$, then $\erasec{c'} \longrightarrow_{\cap IC} cv'$.
        By rule EvaluateC, $c' : T_1 \Rightarrow^l T_1\ ^n \longrightarrow_{\cap IC} cv' : T_1 \Rightarrow^l T_1\ ^n$.
        However, as $cv' : T_1 \Rightarrow^l T_1\ ^n$ is not a value, the rule IdentityC must be applied, therefore $c' : T_1 \Rightarrow^l T_1\ ^n \longrightarrow_{\cap IC} cv'$.
        As $\erasec{c} \longrightarrow_{\cap IC} cv'$, then it is proved.
    \end{itemize}
\end{itemize}
\end{proof}

\begin{lemma}[Elimination of identity casts in e]
\label{eliminationidentitycastse}
Depends on Lemma \ref{eliminationidentitycastsc}.
For any expression $e$, such that $\Gamma \vdash_{\cap CC} e : T$, and $e \longrightarrow_{\cap CC} v$:
\begin{enumerate}
    \item $\Gamma \vdash_{\cap CC} \erasee{e} : T$.
    \item $\erasee{e} \longrightarrow_{\cap CC} v$.
\end{enumerate}
\end{lemma}
\begin{proof}
(1) We proceed by induction on the length of the derivation tree of $\Gamma \vdash_{\cap CC} e : T$.\\\\
Base cases:
\begin{itemize}
    \item e = $x$.
    As $x$ doesn't contain casts, then $\erasee{e} = x$.
    Therefore it is proved.
    \item e = $n$.
    As $n$ doesn't contain casts, then $\erasee{e} = n$.
    Therefore it is proved.
    \item e = $\true$.
    As $\true$ doesn't contain casts, then $\erasee{e} = \true$.
    Therefore it is proved.
    \item e = $\false$.
    As $\false$ doesn't contain casts, then $\erasee{e} = \false$.
    Therefore it is proved.
    \item e = $\blame{T}{l}$.
    As $blame{T}{l}$ doesn't contain casts, then $\erasee{e} = blame{T}{l}$.
    Therefore it is proved.
\end{itemize}
Induction step:
\begin{itemize}
    \item e = $\lambda x : T_1 \cap \ldots \cap T_n\ .\ e'$.
    There are two possibilities:
    \begin{itemize}
        \item Using the rule ${\rightarrow} I$.
        If $\Gamma \vdash_{\cap CC} \lambda x : T_1 \cap \ldots \cap T_n\ .\ e' : T_1 \cap \ldots \cap T_n \rightarrow T$, then $\Gamma, x : T_1 \cap \ldots \cap T_n \vdash_{\cap CC} e' : T$.
        By the induction hypothesis, $\Gamma, x : T_1 \cap \ldots \cap T_n \vdash_{\cap CC} \erasee{e'} : T$.
        As $\erasee{e} = \lambda x : T_1 \cap \ldots \cap T_n\ .\ \erasee{e'}$, then $\Gamma \vdash_{\cap CC} \erasee{e} : T_1 \cap \ldots \cap T_n \rightarrow T$.
        \item Using the rule ${\rightarrow} I'$.
        If $\Gamma \vdash_{\cap CC} \lambda x : T_1 \cap \ldots \cap T_n\ .\ e' : T_i \rightarrow T$, then $\Gamma, x : T_i \vdash_{\cap CC} e' : T$.
        By the induction hypothesis, $\Gamma, x : T_i \vdash_{\cap CC} \erasee{e'} : T$.
        As $\erasee{e} = \lambda x : T_1 \cap \ldots \cap T_n\ .\ \erasee{e'}$, then $\Gamma \vdash_{\cap CC} \erasee{e} : T_i \rightarrow T$.
    \end{itemize}
    \item e = $e_1\ e_2$.
    If $\Gamma \vdash_{\cap CC} e_1\ e_2 : T$, then $\Gamma \vdash_{\cap CC} e_1 : PM$, $PM \rhd T_1 \cap \ldots \cap T_n \rightarrow T$, $\Gamma \vdash_{\cap CC} e_2 : T_1' \cap \ldots \cap T_n'$ and $T_1' \cap \ldots \cap T_n' \sim T_1 \cap \ldots \cap T_n$.
    By the induction hypothesis, $\Gamma \vdash_{\cap CC} \erasee{e_1} : PM$ and $\Gamma \vdash_{\cap CC} \erasee{e_2} : T_1' \cap \ldots \cap T_n'$.
    As $\erasee{e} = \erasee{e_1}\ \erasee{e_2}$, therefore $\Gamma \vdash_{\cap CC} \erasee{e} : T$.
    \item e = $e$.
    If $\Gamma \vdash_{\cap CC} e : T_1 \cap \ldots \cap T_n$, then $\Gamma \vdash_{\cap CC} e : T_1$ and ... and $\Gamma \vdash_{\cap CC} e : T_n$.
    By the induction hypothesis, $\Gamma \vdash_{\cap CC} \erasee{e} : T_1$ and ... and $\Gamma \vdash_{\cap CC} \erasee{e} : T_n$.
    Therefore $\Gamma \vdash_{\cap CC} \erasee{e} : T_1 \cap \ldots \cap T_n$.
    \item e = $e' : T_1 \Rightarrow^l T_2$.
    There are two possibilities:
    \begin{itemize}
        \item $T_1 \neq T_2$.
        If $\Gamma \vdash_{\cap CC} e' : T_1 \Rightarrow^l T_2 : T_2$, then $\Gamma \vdash_{\cap CC} e' : T_1$.
        By the induction hypothesis, $\Gamma \vdash_{\cap CC} \erasee{e'} : T_1$.
        As $\erasee{e} = \erasee{e'} : T_1 \Rightarrow^l T_2$, then $\Gamma \vdash_{\cap CC} \erasee{e} : T_2$.
        \item $T_1 = T_2$.
        If $\Gamma \vdash_{\cap CC} e' : T_1 \Rightarrow^l T_1 : T_1$, then $\Gamma \vdash_{\cap CC} e' : T_1$.
        By the induction hypothesis, $\Gamma \vdash_{\cap CC} \erasee{e'} : T_1$.
        As $\erasee{e} = \erasee{e'} : T_1 \Rightarrow^l T_1$, then $\Gamma \vdash_{\cap CC} \erasee{e} : T_1$.
    \end{itemize}
    \item e = $e' : c_1 \cap \ldots \cap c_n$.
    If $\Gamma \vdash_{\cap CC} e' : c_1 \cap \ldots \cap c_n : T_1 \cap \ldots \cap T_n$, then $\Gamma \vdash_{\cap CC} e' : T$, $\vdash_{\cap IC} c_1 : T_1$ and ... and $\vdash_{\cap IC} c_n : T_n$ and $initialType(c_1) \cap \ldots \cap initialType(c_n) =_{\cap} T$.
    By the induction hypothesis, $\Gamma \vdash_{\cap CC} \erasee{e'} : T$.
    We now have 2 possibilities:
    \begin{itemize}
        \item $\neg(\forall i \in 1..n\ .\ isEmptyCast\ \erasec{c_i})$:
        For all casts $c_i$, with $i \in 1..n$, that don't contain identity casts, then $\erasec{c_i} = c_i$, therefore $\vdash_{\cap IC} \erasec{c_i} : T_i$ and $initialType(\erasec{c_i}) = initialType(c_i)$.
        For the remaining casts, by Lemma \ref{eliminationidentitycastsc}, $\vdash_{\cap IC} \erasec{c_i} : T_i$ and $initialType(\erasec{c_i}) = initialType(c_i)$.
        Therefore, with $\erasee{e} = \erasee{e'} : \erasec{c_1} \cap \ldots \cap \erasec{c_n}$, $\Gamma \vdash_{\cap CC} \erasee{e} : T_1 \cap \ldots \cap T_n$.
        \item $\forall i \in 1..n\ .\ isEmptyCast\ \erasec{c_i}$:
        As all casts are empty casts, then for all casts $\erasec{c_i}$, by Lemma \ref{eliminationidentitycastsc} and by rule T-EmptyC, $\vdash_{\cap IC} \erasec{c_i} : T_i$ and $initialType(\erasec{c_i}) = T_i$.
        Therefore $\erasee{e} = \erasee{e'}$.
        We now have two possibilities:
        \begin{itemize}
            \item If $T$ is not an intersection type, then $T_1 = \ldots = T_n = T$ and by idempotence of $\cap$, we have that $\Gamma \vdash_{\cap CC} \erasee{e} : T_1 \cap \ldots \cap T_n$.
            \item If $T$ is an intersection type, then $T = T_1 \cap \ldots \cap T_n$. Therefore $\Gamma \vdash_{\cap CC} \erasee{e} : T_1 \cap \ldots \cap T_n$.
        \end{itemize}
    \end{itemize}
\end{itemize}
(2) We proceed by induction on the length of the derivation tree of $\longrightarrow_{\cap CC}$.\\\\
Base cases:
\begin{itemize}
    \item e = $(v_1 : cv_1 \cap \ldots \cap cv_n)\ v_2$ and $isValue((v_1 : cv_1 \cap \ldots \cap cv_n)\ v_2)$ and $\exists i \in 1..n\ .\ isArrowCompatible\ cv_i$.
    As $v_1 : cv_1 \cap \ldots \cap cv_n$ and $v_2$ are values, then $e$ doesn't contain identity casts.
    Therefore $\erasee{e} = e$.
    \item e = $v : cv_1 \cap \ldots \cap cv_n : T_1 \Rightarrow^l T_2$ and $isValue\ v : cv_1 \cap \ldots \cap cv_n$ and $v : c_1' \cap \ldots \cap c_m' = mergeIC (v : cv_1 \cap \ldots \cap cv_n : T_1 \Rightarrow^l T_2)$.
    There are 2 possibilities:
    \begin{itemize}
        \item If $T_1 \neq T_2$ and as $v : cv_1 \cap \ldots \cap cv_n$ doesn't contain identity casts, then $\erasee{e} = e$, therefore it is proved.
        \item If $T_1 = T_2$ and as $v : cv_1 \cap \ldots \cap cv_n$ doesn't contain identity casts, then $\erasee{e} = v : cv_1 \cap \ldots \cap cv_n$.
        By rule MergeIC$\cap$, $v : cv_1 \cap \ldots \cap cv_n : T_1 \Rightarrow^l T_2 \longrightarrow_{\cap CC} v : c_1' \cap \ldots \cap c_m'$.
        By rule Evaluate$\cap$, $v : cv_1 : T_1 \Rightarrow^l T_2\ ^{m_1} \cap \ldots \cap cv_n : T_1 \Rightarrow^l T_2\ ^{m_n} \longrightarrow_{\cap CC} v : cv_1 \cap \ldots \cap cv_n$, with $cv_1 : T_1 \Rightarrow^l T_2\ ^{m_1} \longrightarrow_{\cap IC} cv_1$ and ... and $cv_n : T_1 \Rightarrow^l T_2\ ^{m_n} \longrightarrow_{\cap IC} cv_n$ by rule IdentityC.
        As $\erasee{e}$ is already a value, it is proved.
    \end{itemize}
    \item e = $v : T_1 \Rightarrow^l T_2 : c_1 \cap \ldots \cap c_n$ and $isValue\ v : T_1 \Rightarrow^l T_2$ and $v : c_1' \cap \ldots \cap c_n' = mergeCI (v : T_1 \Rightarrow^l T_2 : c_1 \cap \ldots \cap c_n)$.
    There are 2 possibilities:
    \begin{itemize}
        \item $v : T_1 \Rightarrow^l T_2 : c_1 \cap \ldots \cap c_n$ doesn't contain identity casts, then $\erasee{e} = e$, therefore it is proved.
        \item $v : T_1 \Rightarrow^l T_2 : c_1 \cap \ldots \cap c_n$ contain identity casts.
        By rule MergeCI$\cap$, $v : T_1 \Rightarrow^l T_2 : c_1 \cap \ldots \cap c_n \longrightarrow_{\cap CC} v : c_1' \cap \ldots \cap c_n'$.
        By rule Evaluate$\cap$, $v : T_1 \Rightarrow^l T_2 : c_1' \cap \ldots \cap c_n' \longrightarrow_{\cap CC} v : cv_1' \cap \ldots \cap cv_n'$, with $c_1' \longrightarrow_{\cap IC} cv_1'$ and ... and $c_n' \longrightarrow_{\cap IC} cv_n'$.
        For all casts $c_i$ that don't contain identity casts, then $\erasec{c_i} = c_i$, therefore for those casts, the property is proved.
        For all casts $c_i$ that contain identity casts, $mergeCI$ will generate casts $c_i'$ that will evaluate to $cv_i'$.
        By Lemma 6, casts $\erasec{c_i}$ will generate casts $c_i''$ that will evaluate to $cv_i'$, therefore it is proved.
    \end{itemize}
    \item e = $v : cv_1 \cap \ldots \cap cv_n : c_1 \cap \ldots \cap c_m$ and $isValue\ v : cv_1 \cap \ldots \cap cv_n$ and $v : c_1' \cap \ldots \cap c_j' = mergeII (v : cv_1 \cap \ldots \cap cv_n : c_1 \cap \ldots \cap c_m)$.
    There are 2 possibilities:
    \begin{itemize}
        \item $v : cv_1 \cap \ldots \cap cv_n : c_1 \cap \ldots \cap c_m$ doesn't contain identity casts, then $\erasee{e} = e$, therefore it is proved.
        \item $v : cv_1 \cap \ldots \cap cv_n : c_1 \cap \ldots \cap c_m$ contain identity casts.
        By rule MergeII$\cap$, $v : cv_1 \cap \ldots \cap cv_n : c_1 \cap \ldots \cap c_m \longrightarrow_{\cap CC} v : c_1' \cap \ldots \cap c_j'$.
        By rule Evaluate$\cap$, $v : c_1' \cap \ldots \cap c_j' \longrightarrow_{\cap CC} v' : cv_1' \cap \ldots \cap cv_j'$, with $c_1' \longrightarrow_{\cap IC} cv_1'$ and ... and $c_j' \longrightarrow_{\cap IC} cv_j'$.
        For all casts $cv_i$ and $c_i$ that will be joined into $c_i'$ by function mergeII, and that don't contain identity casts, then $\erasec{c_i'} = c_i'$, therefore for those casts, the property is proved.
        For all casts $cv_i$ and $c_i$ that will be joined into $c_i'$ by function mergeII, and that contain identity casts, $c_i' \longrightarrow_{\cap IC} cv_i'$ and by Lemma \ref{eliminationidentitycastsc}, $\erasec{c_i'} \longrightarrow_{\cap IC} cv_i'$, therefore it is proved.
    \end{itemize}
    \item e = $v : c_1 \cap \ldots \cap c_n$ and $\neg(\forall i \in 1 .. n\ .\ isCastValue\ c_i)$.
    By rule Evaluate$\cap$, $v : c_1 \cap \ldots \cap c_n \longrightarrow_{\cap CC} v : cv_1 \cap \ldots \cap cv_n$, with $c_1 \longrightarrow_{\cap IC} cv_1$ and ... and $c_n \longrightarrow_{\cap IC} cv_n$.
    With $\erasee{e} = v : \erasec{c_1} \cap \ldots \cap \erasec{c_n}$, by Lemma \ref{eliminationidentitycastsc}, $\erasec{c_1} \longrightarrow_{\cap IC} cv_1$ and ... and $\erasec{c_n} \longrightarrow_{\cap IC} cv_n$.
    Therefore, by rule Evaluate$\cap$, $v : \erasec{c_1} \cap \ldots \cap \erasec{c_n} \longrightarrow_{\cap CC} v : cv_1 \cap \ldots \cap cv_n$.
    \item e = $v : \blamecast{I_1}{F_1}{l_1}{m_1} \cap \ldots \cap \blamecast{I_n}{F_n}{l_n}{m_n}$.
    As $\erasee{e} = e$, then it is proved.
    \item e = $v : \emptycast{T_1}{m_1} \cap \ldots \cap \emptycast{T_n}{m_n}$.
    As $\erasee{e} = e$, then it is proved.
    \begin{comment}
    \item e = $v : cv_1 \cap \ldots \cap cv_n$ and $\neg(\forall i \in 1 .. n\ .\ isStuck\ c_i)$ and $\exists i \in 1 .. n\ .\ isStuck\ c_i$.
    As $\erasee{e} = e$, then it is proved.
    \end{comment}
\end{itemize}
Induction step:
\begin{itemize}
    \item e =
\end{itemize}
\end{proof}
\end{document}
