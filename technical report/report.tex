\documentclass[a4paper]{article}

\usepackage[utf8]{inputenc}
\usepackage[T1]{fontenc}


\usepackage[a4paper, total={6in, 8in}]{geometry}


\usepackage[english]{babel}

\usepackage{macros}

\usepackage{listings}
\usepackage{verbatim}
\usepackage{amsmath}
\usepackage{graphicx}
\usepackage[colorinlistoftodos]{todonotes}
\usepackage{placeins}
\usepackage{mathpartir}
\usepackage{mathtools,array}
\usepackage{amssymb}
\usepackage{enumitem}
\usepackage{amsthm}
\usepackage{float}
\usepackage{stmaryrd}


\newtheorem{theorem}{Theorem}
\newtheorem{lemma}{Lemma}

\begin{document}

\title{Gradual Intersection Types}

\author{Pedro Ângelo, Mário Florido}

\maketitle
\pagenumbering{arabic}

\section{Language Definition}
\begin{figure}[H]
Syntax
\begin{align*}
&Types\ \begin{aligned}[t] T &::=\ \Int\ |\ \Bool\ |\ \Dyn\ |\ T \rightarrow T'\ |\ T \cap \ldots \cap T\\
                          T' &::=\ \Int\ |\ \Bool\ |\ \Dyn\ |\ T' \rightarrow T' \end{aligned}\\
&Ground\ Types\ G\ ::=\ \Int\ |\ \Bool\ |\ \Dyn \rightarrow \Dyn\\
&Casts\ \begin{aligned}[t] c\ ::=&\ c : T' \Rightarrow^l T'\ ^{cl}\ |\ \blamecast{T'}{T'}{l}{cl}\ |\ \emptycast{T'}{cl}\ \iffalse |\ \stuckcast{T'}{T'}{n}\fi \end{aligned}\\
&Expressions\ \begin{aligned}[t] e\ ::=&\ x\ |\ \lambda x : T\ .\ e\ |\ e\ e\ |\ n\ |\ \true\ |\ \false\\
                                      |&\ e : T' \Rightarrow^l T'\ |\ e : c \cap \ldots \cap c\ |\ \blame{T}{l} \end{aligned}\\
&Cast\ Values\ \begin{aligned}[t] cv\ ::=&\ cv1\ |\ cv2\\
                                cv1\ ::=&\ \emptycast{T'}{cl} : G \Rightarrow^l \Dyn\ ^{cl}\\
                                       |&\ \emptycast{T'}{cl} : T'_1 \rightarrow T'_2 \Rightarrow^l T'_3 \rightarrow T'_4\ ^{cl}\\
                                       |&\ cv1 : G \Rightarrow^l \Dyn\ ^{cl}\\
                                       |&\ cv1 : T'_1 \rightarrow T'_2 \Rightarrow^l T'_3 \rightarrow T'_4\ ^{cl}\\
                                cv2\ ::=&\ \blamecast{T'}{T'}{l}{cl}\\
                                       |&\ \emptycast{T'}{cl} \end{aligned}\\
&Values\ \begin{aligned}[t] v\ ::=&\ x\ |\ \lambda x : T\ .\ e\ |\ n\ |\ \true\ |\ \false\ |\ \blame{T}{l}\\
                                 |&\ v : G \Rightarrow^l \Dyn\\
                                 |&\ v : T'_1 \rightarrow T'_2 \Rightarrow^l T'_3 \rightarrow T'_4\\
                                 |&\ v : cv_1 \cap \ldots \cap cv_n\ such\ that\\
                                 &\neg(\forall_{i\in 1..n}\ .\ cv_i = \blamecast{T'}{T'}{l}{cl})\ \land\\
                                 &\neg(\forall_{i\in 1..n}\ .\ cv_i = \emptycast{T'}{cl}) \end{aligned}
\end{align*}
\hrulefill
\caption{Gradual Intersection System}
\label{intersection_syntax}
\end{figure}

\begin{figure}[H]
\framebox[1.2\width]{$\Gamma \gits e : T$}\ Typing
\begin{mathpar}
\inferrule* [right=$Var$]
{x : T \in \Gamma}
{\Gamma \gits x : T}

\inferrule* [right=${\rightarrow} I$]
{\Gamma, x : T_1 \cap \ldots \cap T_n \gits\ e : T}
{\Gamma \gits \lambda x : T_1 \cap \ldots \cap T_n\ .\ e : T_1 \cap \ldots \cap T_n \rightarrow T}

\inferrule* [right=${\rightarrow} I'$]
{\Gamma, x : T_i \gits\ e : T}
{\Gamma \gits \lambda x : T_1 \cap \ldots \cap T_n\ .\ e : T_i \rightarrow T}

\inferrule* [right=${\rightarrow} E$]
{\Gamma \gits e_1 : PM \\
 PM \rhd T_1 \cap \ldots \cap T_n \rightarrow T \\\\
 \Gamma \gits e_2 : T_1' \cap \ldots \cap T_n' \\
 T_1' \cap \ldots \cap T_n' \sim T_1 \cap \ldots \cap T_n}
{\Gamma \gits e_1\ e_2 : T}

\inferrule* [right=${\cap} I$]
{\Gamma \gits e : T_1\ \ldots\ \Gamma \gits e : T_n}
{\Gamma \gits e : T_1 \cap \ldots \cap T_n}

\inferrule* [right=${\cap} E$]
{\Gamma \gits e : T_1 \cap \ldots \cap T_n}
{\Gamma \gits e : T_i}
\end{mathpar}

\framebox[1.2\width]{$T \sim T$}\ Consistency
\begin{mathpar}
\inferrule* []
{}
{B \sim B}

\inferrule* []
{}
{T \sim \Dyn}

\inferrule* []
{}
{\Dyn \sim T}

\inferrule* []
{T_1 \sim T_3 \\ T_2 \sim T_4}
{T_1 \rightarrow T_2 \sim T_3 \rightarrow T_4}

\inferrule* []
{T_1 \sim T'_1\ \ldots\ T_n \sim T'_n}
{T_1 \cap \ldots \cap T_n \sim T'_1 \cap \ldots \cap T'_n}

\inferrule* []
{T \sim T_1\ \ldots\ T \sim T_n}
{T \sim T_1 \cap \ldots \cap T_n}

\inferrule* []
{T_1 \sim T\ \ldots\ T_n \sim T}
{T_1 \cap \ldots \cap T_n \sim T}
\end{mathpar}

\framebox[1.2\width]{$T \rhd T$}\ Pattern Matching
\begin{mathpar}
\inferrule* []
{}
{T_1 \rightarrow T_2 \rhd T_1 \rightarrow T_2}

\inferrule* []
{}
{\Dyn \rhd \Dyn \rightarrow \Dyn}
\end{mathpar}

\framebox[1.2\width]{$T \sqsubseteq T$}\ Type Precision
\begin{mathpar}
\inferrule* []
{}
{\Dyn \sqsubseteq T}

\inferrule* []
{}
{B \sqsubseteq B}

\inferrule* []
{T_1 \sqsubseteq T_3 \\ T_2 \sqsubseteq T_4}
{T_1 \rightarrow T_2 \sqsubseteq T_3 \rightarrow T_4}

\inferrule* []
{T_1 \sqsubseteq T'_1\ \ldots\ T_n \sqsubseteq T'_n}
{T_1 \cap \ldots \cap T_n \sqsubseteq T'_1 \cap \ldots \cap T'_n}

\inferrule* []
{T \sqsubseteq T_1\ \ldots\ T \sqsubseteq T_n}
{T \sqsubseteq T_1 \cap \ldots \cap T_n}

\inferrule* []
{T_1 \sqsubseteq T\ \ldots\ T_n \sqsubseteq T}
{T_1 \cap \ldots \cap T_n \sqsubseteq T}
\end{mathpar}

\framebox[1.2\width]{$e \sqsubseteq e$}\ Term Precision
\begin{mathpar}
\inferrule* []
{}
{x \sqsubseteq x}

\inferrule* []
{T \sqsubseteq T' \\ e \sqsubseteq e'}
{\lambda x : T\ .\ e \sqsubseteq \lambda x : T'\ .\ e'}

\inferrule* []
{e_1 \sqsubseteq e_1' \\e_2 \sqsubseteq e_2'}
{e_1\ e_2 \sqsubseteq e_1'\ e_2'}
\end{mathpar}
\hrulefill
\caption{Gradual Intersection Type System ($\gits$)}
\label{intersection_type_system}
\end{figure}

\begin{figure}[H]
\framebox[1.2\width]{$\Gamma \iccts e : T$}\ Typing
\begin{mathpar}
\inferrule* []
{}
{rules\ in\ Figure\ \ref{intersection_type_system}\ and}\\

\inferrule* [right=T-App]
{\Gamma \iccts e_1 : T_{11} \rightarrow T_{12} \cap \ldots \cap T_{n1} \rightarrow T_{n2} \\
 \Gamma \iccts e_2 : T_1' \cap \ldots \cap T_n' \\
 T_{11} \sim T_1' \ldots T_{n1} \sim T_n'}
{\Gamma \iccts e_1\ e_2 : T_{12} \cap \ldots \cap T_{n2}}

\inferrule* [right=T-Cast]
{\Gamma \iccts e : T_1 \\ T_1 \sim T_2}
{\Gamma \iccts e : T_1 \Rightarrow^l T_2 : T_2}

\inferrule* [right=T-Blame]
{ }
{\Gamma \iccts \blame{T}{l} : T}

\inferrule* [right=T-IntersectionCast]
{\Gamma \iccts e : T \\ \icts c_1 : T_1\ \ldots\ \icts c_n : T_n \\\\ initialType(c_1) \cap \ldots \cap initialType(c_n) = T}
{\Gamma \iccts e : c_1 \cap \ldots \cap c_n : T_1 \cap \ldots \cap T_n}
\end{mathpar}

\framebox[1.2\width]{initialType(c) = T}
\begin{mathpar}
\inferrule* []
{}
{initialType(c : T_1 \Rightarrow^l T_2\ ^{cl}) = initialType(c)}\\

\inferrule* []
{}
{initialType(\emptycast{T}{cl}) = T}\\

\inferrule* []
{}
{initialType(\blamecast{T_I}{T_F}{l}{cl}) = T_I}
\end{mathpar}

\framebox[1.2\width]{finalType(c) = T}
\begin{mathpar}
\inferrule* []
{}
{finalType(c : T_1 \Rightarrow^l T_2\ ^{cl}) = T_2}\\

\inferrule* []
{}
{finalType(\emptycast{T}{cl}) = T}\\

\inferrule* []
{}
{finalType(\blamecast{T_I}{T_F}{l}{cl}) = T_F}
\end{mathpar}
\hrulefill
\caption{Intersection Cast Calculus ($\iccts$)}
\label{intersection_cast_calculus}
\end{figure}

\begin{figure}[H]
\framebox[1.2\width]{$\Gamma \iccts e \leadsto e : T$}\ Compilation
\begin{mathpar}
\inferrule* [right=Var]
{x : T \in \Gamma}
{\Gamma \iccts x \leadsto x : T}

\inferrule* [right=Abs]
{\Gamma, x : T_1 \cap \ldots \cap T_n \iccts e \leadsto e' : T}
{\Gamma \iccts (\lambda x : T_1 \cap \ldots \cap T_n\ .\ e) \leadsto (\lambda x : T_1 \cap \ldots \cap T_n\ .\ e') : T_1 \cap \ldots \cap T_n \rightarrow T}

\inferrule* [right=App]
{\Gamma \iccts e_1 \leadsto e_1' : PM \\ PM \rhd T_1 \cap \ldots \cap T_n \rightarrow T \\ \Gamma \iccts e_2 \leadsto e_2' : T'_1 \cap \ldots \cap T'_n \\ T'_1 \cap \ldots \cap T'_n \sim T_1 \cap \ldots \cap T_n \\ instances(PM) = S_1 \\ instances(T_1 \cap \ldots \cap T_n \rightarrow T) = S_2 \\ instances(T'_1 \cap \ldots \cap T'_n) = S_3 \\ instances(T_1 \cap \ldots \cap T_n) = S_4 \\ S_1,\ S_2,\ e_1' \hookrightarrow e_1'' \\ S_3,\ S_4,\ e_2' \hookrightarrow e_2''}
{\Gamma \iccts e_1\ e_2 \leadsto e_1''\ e_2'' : T}
\end{mathpar}

\framebox[1.2\width]{instances(T) = \{T\}}
\begin{mathpar}
\inferrule* []
{}
{instances(\Int) = \{\Int\}}\\

\inferrule* []
{}
{instances(\Bool) = \{\Bool\}}\\

\inferrule* []
{}
{instances(\Dyn) = \{\Dyn\}}\\

\inferrule* []
{instances(T_1) = \{T_{11}, \ldots, T_{1n}\}}
{instances(T_1 \rightarrow T_2) = \{T_{11} \rightarrow T_2, \ldots, T_{1n} \rightarrow T_2\}}\\

\inferrule* []
{instances(T_1) = \{T_{11}, \ldots, T_{1m}\} \ldots instances(T_n) = \{T_{n1}, \ldots, T_{nj}\}}
{instances(T_1 \cap \ldots \cap T_n) = \{T_{11}, \ldots, T_{1m}, \ldots, T_{n1}, \ldots, T_{nj}\}}
\end{mathpar}

\framebox[1.2\width]{$S,\ S,\ e \hookrightarrow e$}
\begin{mathpar}
\inferrule* []
{}
{\{T_1\},\ \{T_2\},\ e \hookrightarrow e : T_1 \Rightarrow^l T_2}\\

\inferrule* []
{}
{\{T_{11}, \ldots, T_{1n}\},\ \{T_{21}, \ldots, T_{2n}\},\ e \hookrightarrow e : (\emptycast{T_{11}}{0} : T_{11} \Rightarrow^l T_{21}\ ^{0}) \cap \ldots \cap (\emptycast{T_{1n}}{0} : T_{1n} \Rightarrow^l T_{2n}\ ^{0})}\\

\inferrule* []
{}
{\{T_{11}, \ldots, T_{1n}\},\ \{T_2\},\ e \hookrightarrow e : (\emptycast{T_{11}}{0} : T_{11} \Rightarrow^l T_2\ ^{0}) \cap \ldots \cap (\emptycast{T_{1n}}{0} : T_{1n} \Rightarrow^l T_2\ ^{0})}\\

\inferrule* []
{}
{\{T_1\},\ \{T_{21}, \ldots, T_{2n}\},\ e \hookrightarrow e : (\emptycast{T_1}{0} : T_1 \Rightarrow^l T_{21}\ ^{0}) \cap \ldots \cap (\emptycast{T_1}{0} : T_1 \Rightarrow^l T_{2n}\ ^{0})}
\end{mathpar}
\hrulefill
\caption{Compilation to the Cast Calculus}
\label{intersection_cast_insertion}
\end{figure}

\begin{figure}[H]
\framebox[1.2\width]{$e \gios e$}\ Evaluation
\begin{mathpar}
\inferrule* []
{}
{Simulate\ casts\ on\ data\ types}

\inferrule* [right=Simulate$\cap$]
{is\ value\ v_1 : cv_1 \cap \ldots \cap cv_n \\ \exists i \in 1..n\ .\ isArrowCompatible(cv_i) \\\\ ((c_{11}, c_{12}, c_1^s), \ldots, (c_{m1}, c_{m2}, c_m^s)) = simulateArrow(cv_1, \ldots, cv_n)}
{(v_1 : cv_1 \cap \ldots \cap cv_n)\ v_2 \gios\\ (v_1 : c_1^s \cap \ldots \cap c_m^s)\ (v_2 : c_{11} \cap \ldots \cap c_{m1}) : c_{12} \cap \ldots \cap c_{m2}}

\inferrule* []
{}
{Merge\ casts}\\

\inferrule* [right=MergeIC$\cap$]
{is\ value\ v : cv_1 \cap \ldots \cap cv_n \\ v : c_1' \cap \ldots \cap c_m' = \mergeIC{v : cv_1 \cap \ldots \cap cv_n : T_1 \Rightarrow^l T_2}}
{v : cv_1 \cap \ldots \cap cv_n : T_1 \Rightarrow^l T_2 \gios v : c_1' \cap \ldots \cap c_m'}

\inferrule* [right=MergeCI$\cap$]
{is\ value\ v : T_1 \Rightarrow^l T_2 \\ v : c_1' \cap \ldots \cap c_m' = \mergeCI{v : T_1 \Rightarrow^l T_2 : c_1 \cap \ldots \cap c_n}}
{v : T_1 \Rightarrow^l T_2 : c_1 \cap \ldots \cap c_n \gios v : c_1' \cap \ldots \cap c_m'}\\

\inferrule* [right=MergeII$\cap$]
{is\ value\ v : cv_1 \cap \ldots \cap cv_n \\ v : c_1'' \cap \ldots \cap c_j'' = \mergeII{v : cv_1 \cap \ldots \cap cv_n : c_1' \cap \ldots \cap c_m'}}
{v : cv_1 \cap \ldots \cap cv_n : c_1' \cap \ldots \cap c_m' \gios v : c_1'' \cap \ldots \cap c_j''}\\

\inferrule* []
{}
{Evaluate\ intersection\ casts}\\

\inferrule* [right=EvaluateCasts$\cap$]
{\neg(\forall i \in 1..n\ .\ is\ cast\ value\ c_i) \\ c_1 \icos cv_1\ \ldots\ c_n \icos cv_n}
{v : c_1 \cap \ldots \cap c_n \gios v : cv_1 \cap \ldots \cap cv_n}\\

\inferrule* []
{}
{Transition\ from\ cast\ values\ to\ values}

\inferrule* [right=PropagateBlame$\cap$]
{ }
{v : \blamecast{T_1'}{T_1}{l_1}{cl_1} \cap \ldots \cap \blamecast{T_n'}{T_n}{l_n}{cl_n}\\ \gios \blame{(T_1 \cap \ldots \cap T_n)}{l_1}}

\inferrule* [right=RemoveEmpty$\cap$]
{ }
{v : \emptycast{T_1}{cl_1} \cap \ldots \cap \emptycast{T_n}{cl_n} \gios v}
\end{mathpar}
\hrulefill
\caption{Cast Calculus Semantics ($\gios$)}
\label{casts_calculus_semantics}
\end{figure}

\begin{figure}[H]
\framebox[1.2\width]{\assignCastLabel{c}{cl} = c}
\begin{mathpar}
\inferrule* []
{}
{\assignCastLabel{c : T_1 \Rightarrow^l T_2\ ^{cl}}{cl'} = \assignCastLabel{c}{cl'} : T_1 \Rightarrow^l T_2\ ^{cl'}}\\

\inferrule* []
{}
{\assignCastLabel{\blamecast{T_I}{T_F}{l}{cl'}}{cl} = \blamecast{T_I}{T_F}{l}{cl}}\\

\inferrule* []
{}
{\assignCastLabel{\emptycast{T}{cl'}}{cl} = \emptycast{T}{cl}}\\
\end{mathpar}

\framebox[1.2\width]{isArrowCompatible(c) = Bool}
\begin{mathpar}
\inferrule* []
{}
{isArrowCompatible(c : T_{11} \rightarrow T_{12} \Rightarrow^l T_{21} \rightarrow T_{22}\ ^{cl}) = isArrowCompatible(c)}\\

\inferrule* []
{}
{isArrowCompatible(\emptycast{(T_1 \rightarrow T_2)}{cl}) = True}\\
\end{mathpar}

\framebox[1.2\width]{\separateIntersectionCast{c} = (c, c)}
\begin{mathpar}
\inferrule* []
{}
{\separateIntersectionCast{c : T_1 \Rightarrow^l T_2\ ^{cl}} = (\emptycast{T_1}{cl} : T_1 \Rightarrow^l T_2\ ^{cl}, c)}\\

\inferrule* []
{}
{\separateIntersectionCast{\emptycast{T}{cl}} = (\emptycast{T}{cl}, \emptycast{T}{cl})}\\
\end{mathpar}

\framebox[1.2\width]{\breakdownArrowType{c} = (c, c)}
\begin{mathpar}
\inferrule* []
{}
{\breakdownArrowType{\emptycast{T_{11} \rightarrow T_{12}}{cl} : T_{11} \rightarrow T_{12} \Rightarrow^l T_{21} \rightarrow T_{22}\ ^{cl}} =\\ (\emptycast{T_{21}}{cl} : T_{21} \Rightarrow^l T_{11}\ ^{cl}, \emptycast{T_{12}}{cl} : T_{12} \Rightarrow^l T_{22}\ ^{cl})}\\

\inferrule* []
{}
{\breakdownArrowType{\emptycast{T_1 \rightarrow T_2}{cl}} = (\emptycast{T_1}{cl}, \emptycast{T_2}{cl}})\\
\end{mathpar}

\framebox[1.2\width]{simulateArrow($c_1, \ldots, c_n$) = $((c_{11}, c_{12}, c_1^s), \ldots, (c_{m1}, c_{m2}, c_m^s))$}
\begin{mathpar}
\inferrule* []
{(c_1', \ldots, c_m') = filter\ isArrowCompatible\ (c_1, \ldots, c_n)\\
 ((c_1^f, c_1^s), \ldots, (c_m^f, c_m^s)) = map\ separateIntersectionCast\ (\assignCastLabel{c_1'}{0}, \ldots, \assignCastLabel{c_m'}{0})\\
 ((c_{11}, c_{12}), \ldots, (c_{m1}, c_{m2})) = map\ breakdownArrowType\ (\assignCastLabel{c_1^f}{1}, \ldots, \assignCastLabel{c_m^f}{m})}
{simulateArrow(c_1, \ldots, c_n) = ((c_{11}, c_{12}, c_1^s), \ldots, (c_{m1}, c_{m2}, c_m^s))}\\
\end{mathpar}
\hrulefill
\caption{Definitions for auxiliary semantic functions}
\label{definitions}
\end{figure}

\begin{figure}[H]
\framebox[1.2\width]{getCastLabel(c) = cl}
\begin{mathpar}
\inferrule* []
{}
{getCastLabel(c : T_1 \Rightarrow^l T_2\ ^{cl}) = cl}\\

\inferrule* []
{}
{getCastLabel(\blamecast{T_I}{T_F}{l}{cl}) = cl}\\

\inferrule* []
{}
{getCastLabel(\emptycast{T}{cl}) = cl}\\
\end{mathpar}

\framebox[1.2\width]{sameCastLabel(c, c) = Bool}
\begin{mathpar}
\inferrule* []
{}
{sameCastLabel(c_1, c_2) = getCastLabel(c_1) == 0}\\
\inferrule* []
{}
{sameCastLabel(c_1, c_2) = getCastLabel(c_2) == 0}\\

\inferrule* []
{}
{sameCastLabel(c_1, c_2) = getCastLabel(c_1) == getCastLabel(c_2)}\\
\end{mathpar}
\framebox[1.2\width]{joinCasts(c, c) = c}
\begin{mathpar}
\inferrule* []
{}
{joinCasts(c : T_1 \Rightarrow^l T_2\ ^{cl}, c') = joinCasts(c, c') : T_1 \Rightarrow^l T_2\ ^{cl}}\\

\inferrule* []
{}
{joinCasts(\blamecast{T_I}{T_F}{l}{cl}, c) = \blamecast{T_I}{T_F}{l}{cl}}\\

\inferrule* []
{}
{getCastLabel(\emptycast{T}{cl}, c) = \assignCastLabel{c}{cl}}\\
\end{mathpar}

\framebox[1.2\width]{$\mergeIC{e} = e$}
\begin{mathpar}
\inferrule* []
{(c_1', \ldots, c_m') = filter\ (\lambda x\ .\ finalType\ x == T_1)\ (c_1, \ldots, c_n)\\
 (cl_1, \ldots, cl_m) = map\ getCastLabel\ (c_1', \ldots, c_m')}
{\mergeIC{e : c_1 \cap \ldots \cap c_n : T_1 \Rightarrow^l T_2} = e : (c_1' : T_1 \Rightarrow^l T_2\ ^{cl_1}) \cap \ldots \cap (c_m' : T_1 \Rightarrow^l T_2\ ^{cl_m})}\\
\end{mathpar}

\framebox[1.2\width]{$\mergeCI{e} = e$}
\begin{mathpar}
\inferrule* []
{(c_1', \ldots, c_m') = filter\ (\lambda x\ .\ initialType\ x == T_2)\ (c_1, \ldots, c_n)\\
 (c_1'', \ldots, c_m'') = map\ (\lambda x\ .\ joinCasts\ x\ (\emptycast{T_1}{0} : T_1 \Rightarrow^l T_2\ ^0))\ (c_1', \ldots, c_m')}
{\mergeCI{e : T_1 \Rightarrow^l T_2 : c_1 \cap \ldots \cap c_n} = e : c_1'' \cap \ldots \cap c_m''}\\
\end{mathpar}

\framebox[1.2\width]{$\mergeII{e} = e$}
\begin{mathpar}
\inferrule* []
{(c_1', \ldots, c_o') = [joinCast\ y\ x\ |\ x \leftarrow (c_{11}, \ldots, c_{1m}),\ y \leftarrow (c_{21}, \ldots, c_{2n}),\\ sameCastLabel\ y\ x\ \&\&\ initialType(y) == finalType (x)]}
{\mergeII{e : c_{11} \cap \ldots \cap c_{1m} : c_{21} \cap \ldots \cap c_{2n}} = e : c_1' \cap \ldots \cap c_o'}\\
\end{mathpar}
\hrulefill
\caption{Definitions for auxiliary semantic functions}
\label{definitions2}
\end{figure}

\begin{figure}[H]
\framebox[1.2\width]{$\icts c : T$}\ Typing
\begin{mathpar}

\inferrule* [right=T-SingleC]
{\icts c : T_1 \\ T_1 \sim T_2}
{\icts (c : T_1 \Rightarrow^l T_2\ ^{cl}) : T_1}

\inferrule* [right=T-EmptyC]
{ }
{\icts \emptycast{T}{cl} : T}

\inferrule* [right=T-BlameC]
{ }
{\icts \blamecast{T_I}{T_F}{l}{cl} : T_F}
\end{mathpar}
\hrulefill
\caption{Intersection Casts Type System ($\icts$)}
\label{intersection_casts_type_system}
\end{figure}

\begin{figure}[H]
\framebox[1.2\width]{$c \icos c$}\ Evaluation
\begin{mathpar}
\inferrule* []
{}
{Push\ blame\ to\ top\ level}\\

\inferrule* [right=PushBlameC]
{ }
{\blamecast{T_I}{T_F}{l_1}{cl_1} : T_1 \Rightarrow^{l_2} T_2\ ^{cl_2} \icos \blamecast{T_I}{T_2}{l_1}{cl_1}}

\inferrule* []
{}
{Evaluate\ inside\ casts}\\

\inferrule* [right=EvaluateC]
{\neg(is\ cast\ value\ c) \\ c \icos c'}
{c : T_1 \Rightarrow^l T_2\ ^{cl} \icos c' : T_1 \Rightarrow^l T_2\ ^{cl}}

\inferrule* []
{}
{Detect\ success\ or\ failure\ of\ casts}\\

\inferrule* [right=IdentityC]
{is\ cast\ value\ 1\ c \lor is\ empty\ cast\ c}
{c : T \Rightarrow^l T\ ^{cl} \icos c}

\inferrule* [right=SucceedC]
{is\ cast\ value\ 1\ c \lor is\ empty\ cast\ c}
{c : G \Rightarrow^{l_1} \Dyn\ ^{cl_1} : \Dyn \Rightarrow^{l_2} G\ ^{cl_2} \icos c}

\inferrule* [right=FailC]
{is\ cast\ value\ 1\ c \lor is\ empty\ cast\ c \\ \neg(same\ ground\ G_1\ G_2) \\ initialType(c) = T_I}
{c : G_1 \Rightarrow^{l_1} \Dyn\ ^{cl_1} : \Dyn \Rightarrow^{l_2} G_2\ ^{cl_2} \icos \blamecast{T_I}{G_2}{l_2}{cl_1}}\\

\inferrule* []
{}
{Mediate\ the\ transition\ between\ the\ two\ disciplines}\\

\inferrule* [right=GroundC]
{is\ cast\ value\ 1\ c \lor is\ empty\ cast\ c \\ G\ is\ ground\ type\ of\ T \\ \neg(ground\ T)}
{c : T \Rightarrow^l \Dyn\ ^{cl} \icos c : T \Rightarrow^l G\ ^{cl} : G \Rightarrow^l \Dyn\ ^{cl}}

\inferrule* [right=ExpandC]
{is\ cast\ value\ 1\ c \lor is\ empty\ cast\ c \\ G\ is\ ground\ type\ of\ T \\ \neg(ground\ T)}
{c : \Dyn \Rightarrow^l T\ ^{cl} \icos c : \Dyn \Rightarrow^l G\ ^{cl} : G \Rightarrow^l T\ ^{cl}}
\end{mathpar}
\hrulefill
\caption{Intersection Casts Semantics ($\icos$)}
\label{intersection_casts_semantics}
\end{figure}

\begin{figure}[H]
\framebox[1.2\width]{\erasee{e} = e} Erase identity casts
\begin{mathpar}
\inferrule* []
{}
{\erasee{x} = x}\\

\inferrule* []
{}
{\erasee{\lambda x : T\ .\ e} = \lambda x : T\ .\ \erasee{e}}\\

\inferrule* []
{}
{\erasee{e_1\ e_2} = \erasee{e_1}\ \erasee{e_2}}\\

\inferrule* []
{}
{\erasee{n} = n}\\

\inferrule* []
{}
{\erasee{\true} = \true}\\

\inferrule* []
{}
{\erasee{\false} = \false}\\

\inferrule* []
{}
{\erasee{e : T \Rightarrow^l T} = \erasee{e}}\\

\inferrule* []
{}
{\erasee{e : T_1 \Rightarrow^l T_2} = \erasee{e} : T_1 \Rightarrow^l T_2}\\

\inferrule* []
{\erasec{c_1} = \emptycast{T_1}{cl_1}\ \ldots\ \erasec{c_n} = \emptycast{T_n}{cl_n}}
{\erasee{e : c_1 \cap \ldots \cap c_n} = \erasee{e}}\\

\inferrule* []
{\erasec{c_1} = c_1'\ \ldots\ \erasec{c_n} = c_n'}
{\erasee{e : c_1 \cap \ldots \cap c_n} = \erasee{e} : c_1' \cap \ldots \cap c_n'}\\

\end{mathpar}

\framebox[1.2\width]{\erasec{c} = c} Erase identity casts
\begin{mathpar}
\inferrule* []
{}
{\erasec{c : T \Rightarrow^l T\ ^{cl}} = \erasec{c}}\\

\inferrule* []
{}
{\erasec{c : T_1 \Rightarrow^l T_2\ ^{cl}} = \erasec{c} : T_1 \Rightarrow^l T_2\ ^{cl}}\\

\inferrule* []
{}
{\erasec{\blamecast{T_I}{T_F}{l}{cl}} = \blamecast{T_I}{T_F}{l}{cl}}\\

\inferrule* []
{}
{\erasec{\emptycast{T}{cl}} = \emptycast{T}{cl}}
\end{mathpar}
\hrulefill
\caption{Identity Cast Erasure}
\label{identity_cast_erasure}
\end{figure}

\newpage

\section{Proofs}
\begin{lemma}[Consistency reduces to equality when comparing static types]
\label{consistencytoequality}
If $T_1$ and $T_2$ are static types then $T_1 = T_2 \iff T_1 \sim T_2$.
\end{lemma}
\begin{proof}
We proceed by structural induction on $T$.\\\\
Base cases:
\begin{itemize}
    \item $T_1 = \Int$.
    \begin{itemize}
        \item If $\Int = \Int$, then by the definition of $\sim$, $\Int \sim \Int$.
        \item If $\Int \sim \Int$, then, $\Int = \Int$.
    \end{itemize}
    \item $T_1 = \Bool$.
    \begin{itemize}
        \item If $\Bool = \Bool$, then by the definition of $\sim$, $\Bool \sim \Bool$.
        \item If $\Bool \sim \Bool$, then, $\Bool = \Bool$.
    \end{itemize}
    \item $T_1 = \Dyn$.
    This case is not considered due to the assumption that $T_1$ is a static type.
\end{itemize}
Induction step:
\begin{itemize}
    \item $T_1 = T_{11} \rightarrow T_{12}$.
    \begin{itemize}
        \item If $T_{11} \rightarrow T_{12} = T_{21} \rightarrow T_{22}$, for some $T_{21}$ and $T_{22}$, then $T_{11} = T_{21}$ and $T_{12} = T_{22}$.
        By the induction hypothesis, $T_{11} \sim T_{21}$ and $T_{12} \sim T_{22}$.
        Therefore, by the definition of $\sim$, $T_{11} \rightarrow T_{12} \sim T_{21} \rightarrow T_{22}$.
        \item If $T_{11} \rightarrow T_{12} \sim T_2$, then by the definition of $\sim$, $T_2 = T_{21} \rightarrow T_{22}$ and $T_{11} \sim T_{21}$ and $T_{12} \sim T_{22}$.
        By the induction hypothesis, $T_{11} = T_{21}$ and $T_{12} = T_{22}$.
        Therefore, $T_{11} \rightarrow T_{12} = T_{21} \rightarrow T_{22}$.
    \end{itemize}
    \item $T_1 = T_{11} \cap \ldots \cap T_{1n}$.
    \begin{itemize}
        \item If $T_{11} \cap \ldots \cap T_{1n} = T_2$, then $\exists T_{21} \ldots T_{2n}\ .\ T_2 = T_{21} \cap \ldots \cap T_{2n}$ and $T_{11} = T_{21}$ and ... and $T_{1n} = T_{2n}$.
        By the induction hypothesis, $T_{11} \sim T_{21}$ and ... and $T_{1n} \sim T_{2n}$.
        Therefore, by the definition of $\sim$, $T_{11} \cap \ldots \cap T_{1n} \sim T_{21} \cap \ldots \cap T_{2n}$.
        \item If $T_{11} \cap \ldots \cap T_{1n} \sim T_2$, then either:
        \begin{itemize}
            \item $\exists T_{21} \ldots T_{2n}\ .\ T_2 = T_{21} \cap \ldots \cap T_{2n}$ and $T_{11} \sim T_{21}$ and ... and $T_{1n} \sim T_{2n}$.
            By the induction hypothesis, $T_{11} = T_{21}$ and ... and $T_{1n} = T_{2n}$.
            Therefore, $T_{11} \cap \ldots \cap T_{1n} = T_{21} \cap \ldots \cap T_{2n}$.
            \item $T_{11} \sim T_2$ and ... and $T_{1n} \sim T_2$.
            By the induction hypothesis, $T_{11} = T_2$ and ... and $T_{1n} = T_2$.
            Due to the idempotence property of intersection types, $T_2 \cap \ldots \cap T_2 = T_2$
            Therefore, $T_{11} \cap \ldots \cap T_{1n} = T_2$.
        \end{itemize}
    \end{itemize}
\end{itemize}
\end{proof}

\begin{theorem}[Conservative Extension]
\label{conservative_extension_typesystem}
Depends on Lemma \ref{consistencytoequality}.
If e is fully static and T is a static type, then $\Gamma \sits e : T \iff \Gamma \gits e : T$.
\end{theorem}
\begin{proof}
We proceed by induction on the length of the derivation tree of $\sits$ and $\gits$ for the left and right direction of the implication, respectively.\\\\
Base case:
\begin{itemize}
    \item Rule $Var$.
    \begin{itemize}
        \item If $\Gamma \sits x : T$, then $x : T \in \Gamma$.
        Therefore, $\Gamma \gits x : T$.
        \item If $\Gamma \gits x : T$, then $x : T \in \Gamma$.
        Therefore, $\Gamma \sits e : T$.
    \end{itemize}
    \item Rule $Int$.
    \begin{itemize}
        \item If $\Gamma \sits n : \Int$, then $\Gamma \gits n : \Int$.
        \item If $\Gamma \gits n : \Int$, then $\Gamma \sits n : \Int$.
    \end{itemize}
    \item Rule $True$.
    \begin{itemize}
        \item If $\Gamma \sits \true : \Bool$, then $\Gamma \gits \true : \Bool$.
        \item If $\Gamma \gits \true : \Bool$, then $\Gamma \sits \true : \Bool$.
    \end{itemize}
    \item Rule $False$.
    \begin{itemize}
        \item If $\Gamma \sits \false : \Bool$, then $\Gamma \gits \false : \Bool$.
        \item If $\Gamma \gits \false : \Bool$, then $\Gamma \sits \false : \Bool$.
    \end{itemize}
\end{itemize}
Induction step:
\begin{itemize}
    \item Rule ${\rightarrow} I$.
    \begin{itemize}
        \item If $\Gamma \sits \lambda x\ .\ T_1 \cap \ldots \cap T_n\ .\ e' : T_1 \cap \ldots \cap T_n \rightarrow T$, then $\Gamma, x : T_1 \cap \ldots \cap T_n \sits e' : T$.
        By the induction hypothesis, $\Gamma, x : T_1 \cap \ldots \cap T_n \gits e' : T$.
        Therefore, $\Gamma \gits \lambda x\ .\ T_1 \cap \ldots \cap T_n\ .\ e' : T_1 \cap \ldots \cap T_n \rightarrow T$.
        \item If $\Gamma \gits \lambda x\ .\ T_1 \cap \ldots \cap T_n\ .\ e' : T_1 \cap \ldots \cap T_n \rightarrow T$, then $\Gamma, x : T_1 \cap \ldots \cap T_n \gits e' : T$.
        By the induction hypothesis, $\Gamma, x : T_1 \cap \ldots \cap T_n \sits e' : T$.
        Therefore, $\Gamma \sits \lambda x\ .\ T_1 \cap \ldots \cap T_n\ .\ e' : T_1 \cap \ldots \cap T_n \rightarrow T$.
    \end{itemize}
    \item Rule ${\rightarrow} I'$.
    \begin{itemize}
        \item If $\Gamma \sits \lambda x\ .\ T_1 \cap \ldots \cap T_n\ .\ e' : T_i \rightarrow T$, then $\Gamma, x : T_i \sits e' : T$.
        By the induction hypothesis, $\Gamma, x : T_i \gits e' : T$.
        Therefore, $\Gamma \gits \lambda x\ .\ T_1 \cap \ldots \cap T_n\ .\ e' : T_i \rightarrow T$.
        \item If $\Gamma \gits \lambda x\ .\ T_1 \cap \ldots \cap T_n\ .\ e' : T_i \rightarrow T$, then $\Gamma, x : T_i \gits e' : T$.
        By the induction hypothesis, $\Gamma, x : T_i \sits e' : T$.
        Therefore, $\Gamma \sits \lambda x\ .\ T_1 \cap \ldots \cap T_n\ .\ e' : T_i \rightarrow T$.
    \end{itemize}
    \item Rule ${\rightarrow} E$.
    \begin{itemize}
        \item If $\Gamma \sits e_1\ e_2 : T$ then $\Gamma \sits e_1 : T_1 \cap \ldots \cap T_n \rightarrow T$ and $\Gamma \sits e_2 : T_1 \cap \ldots \cap T_n$.
        By the induction hypothesis, $\Gamma \gits e_1\ : T_1 \cap \ldots \cap T_n \rightarrow T$ and $\Gamma \gits e_2 : T_1 \cap \ldots \cap T_n$.
        By the definition of $\rhd$, $T_1 \cap \ldots \cap T_n \rightarrow T \rhd T_1 \cap \ldots \cap T_n \rightarrow T$.
        By the definition of consistency ($T \sim T$), $T_1 \cap \ldots \cap T_n \sim T_1 \cap \ldots \cap T_n$.
        Therefore, $\Gamma \gits e_1\ e_2 : T$.
        \item If $\Gamma \gits e_1\ e_2 : T$ then $\Gamma \gits e_1 : PM$, $PM \rhd T_1 \cap \ldots \cap T_n \rightarrow T$, $\Gamma \gits e_2 : T'_1 \cap \ldots \cap T'_n$ and $T'_1 \cap \ldots \cap T'_n \sim T_1 \cap \ldots \cap T_n$.
        By the definition of $\rhd$, $PM = T_1 \cap \ldots \cap T_n \rightarrow T$, therefore $\Gamma \gits e_1 : T_1 \cap \ldots \cap T_n \rightarrow T$.
        By Lemma \ref{consistencytoequality}, $T'_1 \cap \ldots \cap T'_n = T_1 \cap \ldots \cap T_n$, and therefore $\Gamma \gits e_2 : T_1 \cap \ldots \cap T_n$.
        By the induction hypothesis, $\Gamma \sits e_1\ : T_1 \cap \ldots \cap T_n \rightarrow T$ and $\Gamma \sits e_2 : T_1 \cap \ldots \cap T_n$.
        Therefore, $\Gamma \sits e_1\ e_2 : T$.
    \end{itemize}
    \item Rule ${\cap} I$.
    \begin{itemize}
        \item If $\Gamma \sits e : T_1 \cap \ldots \cap T_n$ then $\Gamma \sits e : T_1$ and ... and $\Gamma \sits e : T_n$.
        By the induction hypothesis, $\Gamma \gits e : T_1$ and ... and $\Gamma \gits e : T_n$.
        Therefore, $\Gamma \gits e : T_1 \cap \ldots \cap T_n$.
        \item If $\Gamma \gits e : T_1 \cap \ldots \cap T_n$ then $\Gamma \gits e : T_1$ and ... and $\Gamma \gits e : T_n$.
        By the induction hypothesis, $\Gamma \sits e : T_1$ and ... and $\Gamma \sits e : T_n$.
        Therefore $\Gamma \sits e : T_1 \cap \ldots \cap T_n$.
    \end{itemize}
    \item Rule ${\cap} E$.
    \begin{itemize}
        \item If $\Gamma \sits e : T_i$ then $\Gamma \sits e : T_1 \cap \ldots \cap T_n$, such that $T_i \in \{T_1, ..., T_n\}$.
        By the induction hypothesis, $\Gamma \gits e : T_1 \cap \ldots \cap T_n$.
        As $T_i \in \{T_1, ..., T_n\}$, then $\Gamma \gits e : T_i$.
        \item If $\Gamma \gits e : T_i$ then $\Gamma \gits e : T_1 \cap \ldots \cap T_n$, such that $T_i \in \{T_1, ..., T_n\}$.
        By the induction hypothesis, $\Gamma \sits e : T_1 \cap \ldots \cap T_n$.
        As $T_i \in \{T_1, ..., T_n\}$, then $\Gamma \sits e : T_i$.
    \end{itemize}
\end{itemize}
\end{proof}

\begin{theorem}[Monotonicity w.r.t. precision]
\label{monotonicity_wtr_precision}
If $\Gamma \gits e : T$ and $e' \sqsubseteq e$ then $\Gamma \gits e' : T'$ and $T' \sqsubseteq T$ for some $T'$.
\end{theorem}
\begin{proof}
We proceed by induction on the length of the derivation tree of $\Gamma \gits e : T$.\\\\
Base case:
\begin{itemize}
    \item Rule $Var$.
    If $\Gamma \gits x : T$ and $x \sqsubseteq x$, then $\Gamma \gits x : T$ and $T \sqsubseteq T$.
    \item Rule $Int$.
    If $\Gamma \gits n : \Int$ and $n \sqsubseteq n$, then $\Gamma \gits n : \Int$ and $\Int \sqsubseteq \Int$.
    \item Rule $True$.
    If $\Gamma \gits \true : \Bool$ and $\true \sqsubseteq \true$, then $\Gamma \gits \true : \Bool$ and $\Bool \sqsubseteq \Bool$.
    \item Rule $False$.
    If $\Gamma \gits \false : \Bool$ and $\false \sqsubseteq \false$, then $\Gamma \gits \false : \Bool$ and $\Bool \sqsubseteq \Bool$.
\end{itemize}
Induction step:
\begin{itemize}
    \item Rule ${\cap} I$.
    If $\Gamma \gits \lambda x : T_1 \cap \ldots \cap T_n\ .\ e : T_1 \cap \ldots \cap T_n \rightarrow T$ and $\lambda x : T_1' \cap \ldots \cap T_n'\ .\ e' \sqsubseteq \lambda x : T_1 \cap \ldots \cap T_n\ .\ e$, then $\Gamma \gits e : T$, $T_1' \cap \ldots \cap T_n' \sqsubseteq T_1 \cap \ldots \cap T_n$ and $e' \sqsubseteq e$.
    By the induction hypothesis, $\Gamma \gits e' : T'$ and $T' \sqsubseteq T$.
    As $\Gamma \gits \lambda x : T_1' \cap \ldots \cap T_n'\ .\ e' : T_1' \cap \ldots \cap T_n' \rightarrow T'$, and by the definition of $\sqsubseteq$, $T_1' \cap \ldots \cap T_n' \rightarrow T' \sqsubseteq T_1 \cap \ldots \cap T_n \rightarrow T$, then it is proved.
    \item Rule ${\cap} I'$.
    If $\Gamma \gits \lambda x : T_1 \cap \ldots \cap T_n\ .\ e : T_i \rightarrow T$ and $\lambda x : T_1' \cap \ldots \cap T_n'\ .\ e' \sqsubseteq \lambda x : T_1 \cap \ldots \cap T_n\ .\ e$, then $\Gamma \gits e : T$, $T_1' \cap \ldots \cap T_n' \sqsubseteq T_1 \cap \ldots \cap T_n$ and $e' \sqsubseteq e$.
    By the induction hypothesis, $\Gamma \gits e' : T'$ and $T' \sqsubseteq T$.
    As $\Gamma \gits \lambda x : T_1' \cap \ldots \cap T_n'\ .\ e' : T_i' \rightarrow T'$, and by the definition of $\sqsubseteq$, $T_i' \rightarrow T' \sqsubseteq T_i \rightarrow T$, then it is proved.
    \item Rule ${\rightarrow} E$.
    If $\Gamma \gits e_1\ e_2 : T$ and $e_1'\ e_2' \sqsubseteq e_1\ e_2$ then $\Gamma \gits e_1 : PM$, $PM \rhd T_{11} \cap \ldots \cap T_{1n} \rightarrow T$, $\Gamma \gits e_2 : T_{21} \cap \ldots \cap T_{2n}$, and $T_{21} \cap \ldots \cap T_{2n} \sim T_{11} \cap \ldots \cap T_{1n}$, $e_1' \sqsubseteq e_1$ and $e_2' \sqsubseteq e_2$.
    By the induction hypothesis, $\Gamma \gits e_1' : PM'\ and\ PM' \sqsubseteq PM\ and\ PM' \rhd T_{11}' \cap \ldots \cap T_{1n}' \rightarrow T'$ and $\Gamma \gits e_2' : T_{21}' \cap \ldots \cap T_{2n}'\ and\ T_{21}' \cap \ldots \cap T_{2n}' \sqsubseteq T_{21} \cap \ldots \cap T_{2n}\ and\ T_{21}' \cap \ldots \cap T_{2n}' \sim T_{11}' \cap \ldots \cap T_{1n}'$.
    By the definition of $\sqsubseteq$ and $\rhd$, $T_{11}' \cap \ldots \cap T_{1n}' \rightarrow T' \sqsubseteq T_{11} \cap \ldots \cap T_{1n} \rightarrow T$, and therefore, $T' \sqsubseteq T$.
    As $\Gamma \gits e_1'\ e_2' : T'$, it is proved.
    \item Rule ${\cap} I$.
    If $\Gamma \gits e : T_1 \cap \ldots \cap T_n$ and $e' \sqsubseteq e$, then $\Gamma \gits e : T_1$ and ... and $\Gamma \gits e : T_n$.
    By the induction hypothesis, $\Gamma \gits e' : T_1'\ and\ T_1' \sqsubseteq T_1$ and ... and $\Gamma \gits e' : T_n'\ and\ T_n' \sqsubseteq T_n$.
    Then, $\Gamma \gits e' : T_1' \cap \ldots \cap T_n'$ and by the definition of $\sqsubseteq$, $T_1' \cap \ldots \cap T_n' \sqsubseteq T_1 \cap \ldots \cap T_n$, then it is proved.
    \item Rule ${\cap} E$.
    If $\Gamma \gits e : T_i$ and $e' \sqsubseteq e$, then $\Gamma \gits e : T_1 \cap \ldots \cap T_n$ such that $T_i \in \{T_1, ..., T_n\}$.
    By the induction hypothesis, $\Gamma \gits e' : T_1' \cap \ldots \cap T_n'$ and $T_1' \cap \ldots \cap T_n' \sqsubseteq T_1 \cap \ldots \cap T_n$.
    Therefore, $\Gamma \gits e' : T_i'$ and by the definition of $\sqsubseteq$, $T_i' \sqsubseteq T_i$, then it is proved.
\end{itemize}
\end{proof}

\begin{theorem}[Type preservation of cast insertion]
\label{type_preservation_cast_insertion}
If $\Gamma \gits e : T$ then $\Gamma \iccts e \leadsto e' : T$ and $\Gamma \iccts e' : T$.
\end{theorem}

\begin{theorem}[Monotonicity of cast insertion]
\label{monotonicity_cast_insertion}
If $\Gamma \iccts e_1 \leadsto e_1' : T$ and $\Gamma \iccts e_2 \leadsto e_2' : T$ and $e_1 \sqsubseteq e_2$ then $e_1' \sqsubseteq e_2'$.
\end{theorem}

\begin{lemma}[Expressions annotated with only static types type with static types]
\label{staticexpressionstypestatic}
If $e$ is annotated with only static types then:
\begin{enumerate}
    \item $\Gamma \gits e : T$, for some static $T$.
    \item $\Gamma \iccts e \leadsto e' : T$, for some static $T$.
\end{enumerate}
\end{lemma}
\begin{proof}
(1) We proceed by induction on the length of the derivation tree of $\gits$.\\\\
Base cases:
\begin{itemize}
    \item Rule $Var$.
    If $\Gamma \gits x : T$, then $x : T \in \Gamma$.
    Therefore, there must have been at some point in the typing derivation the application of the rules $({\rightarrow} I)$ or $({\rightarrow} I')$ to type the expression $\lambda x : T . e$, for some $e$.
    Both rules introduze the binding $x : T$ in $\Gamma$, such that $T$ is a static type.
    \item Rule $Int$.
    As $\Gamma \gits n : \Int$, it is proved.
    \item Rule $True$.
    As $\Gamma \gits \true : \Bool$, it is proved.
    \item Rule $False$.
    As $\Gamma \gits \false : \Bool$, it is proved.
\end{itemize}
Induction step:
\begin{itemize}
    \item Rule ${\rightarrow} I$.
    If $\lambda x : T_1 \cap \ldots \cap T_n\ .\ e$ is annotated with only static types, then $T_1 \cap \ldots \cap T_n$ is a static type.
    By rule $({\rightarrow} I)$, $\Gamma, x : T_1 \cap \ldots \cap T_n \gits e : T$.
    By the induction hypothesis, $T$ is a static type.
    Therefore $T_1 \cap \ldots \cap T_n \rightarrow T$ is a static type.
    As $\Gamma \gits \lambda x : T_1 \cap \ldots \cap T_n\ .\ e : T_1 \cap \ldots \cap T_n \rightarrow T$, then it is proved.
    \item Rule ${\rightarrow} I'$.
    If $\lambda x : T_1 \cap \ldots \cap T_n\ .\ e$ is annotated with only static types, then $T_1 \cap \ldots \cap T_n$ is a static type.
    By rule $({\rightarrow} I')$, $\Gamma, x : T_i \gits e : T$.
    Since $T_1 \cap \ldots \cap T_n$ is a static type, then so is $T_i$.
    By the induction hypothesis, $T$ is a static type, therefore so is $T_i \rightarrow T$.
    As $\Gamma \gits \lambda x : T_1 \cap \ldots \cap T_n\ .\ e : T_i \rightarrow T$, then it is proved.
    \item Rule ${\rightarrow} E$.
    If $e_1\ e_2$ is annotated with only static types, then so are $e_1$ and $e_2$.
    By the induction hypothesis, $PM$ is a static type.
    By the definition of $\rhd$, $T_1 \cap \ldots \cap T_n \rightarrow T$ is also a static type.
    Therefore, $T$ is a static type.
    As $\Gamma \gits e_1\ e_2 : T$, then it is proved.
    \item Rule ${\cap} I$.
    If $e$ is annotated with only static types, then by the induction hypothesis, $T_1 \ldots T_n$ are static types.
    Therefore, $T_1 \cap \ldots \cap T_n$ is a static type.
    As $\Gamma \gits e : T_1 \cap \ldots \cap T_n$, then it is proved.
    \item Rule ${\cap} E$.
    If $e$ is annotated with only static types, then by the induction hypothesis, $T_1 \cap \ldots \cap T_n$ is a static type.
    Therefore, $T_i$ is a static type.
    As $\Gamma \gits e : T_i$, then it is proved.
\end{itemize}
(2) We proceed by induction on the length of the derivation tree of $\Gamma \iccts e \leadsto e : T$.\\\\
Base cases:
\begin{itemize}
    \item Rule $Var$.
    If $\Gamma \iccts x \leadsto x : T$, then there is a binding $x : T \in \Gamma$.
    Therefore, there must have been at some point in the typing derivation, the application of the rule for the term $\lambda x : T_1 \cap \ldots \cap T_n\ .\ e$ for some expressions $e$.
    If $e$ is annotated with only static types, then the rule introduzes the binding $x : T$ in $\Gamma$, such that $T$ is a static type.
\end{itemize}
Induction step:
\begin{itemize}
    \item Rule $Abs$.
    If $\lambda x : T_1 \cap \ldots \cap T_n\ .\ e$ is annotated with only static types, then $T_1 \cap \ldots \cap T_n$ is a static type.
    By the induction hypothesis, $T$ is a static type.
    Therefore $T_1 \cap \ldots \cap T_n \rightarrow T$ is a static type.
    As $\Gamma \iccts \lambda x : T_1 \cap \ldots \cap T_n\ .\ e \leadsto \lambda x : T_1 \cap \ldots \cap T_n\ .\ e' : T_1 \cap \ldots \cap T_n \rightarrow T$, then it is proved.
    \item Rule $App$.
    If $e_1\ e_2$ is annotated with only static types, then so are $e_1$ and $e_2$.
    By the induction hypothesis, $PM$ is a static type.
    By the definition of $\rhd$, $T_1 \cap \ldots \cap T_n \rightarrow T$ is also a static type.
    Therefore, $T$ is a static type.
    As $\Gamma \iccts e_1\ e_2 \leadsto e_1'\ e_2' : T$, then it is proved.
\end{itemize}
\end{proof}

\begin{lemma}[Static program compilation only adds identity casts]
\label{staticcompilation}
Depends on Lemmas \ref{consistencytoequality} and \ref{staticexpressionstypestatic}.
If $e$ is annotated with only static types and $\Gamma \iccts e \leadsto e' : T$, then any casts $e'$ contains are identity casts.\\
By identity casts, we mean casts of the form $e : T \Rightarrow^l T$ for some $T$ and casts $e : c_1 \cap \ldots \cap c_n$ such that $c_1 = \emptycast{T_1}{0} : T_1 \Rightarrow T_1\ ^{0}$ and ... and $c_n = \emptycast{T_n}{0} : T_n \Rightarrow T_n\ ^{0}$ for some $T_1,\ \ldots,\ T_n$.
\end{lemma}
\begin{proof}
We proceed by induction on the length of the derivation tree of $\Gamma \iccts e \leadsto e : T$.\\\\
Base cases:
\begin{itemize}
    \item Rule $Var$.
    As $\Gamma \iccts x \leadsto x : T$, and $x$ doesn't have any casts, then it is proved.
\end{itemize}
Induction step:
\begin{itemize}
    \item Rule $Abs$.
    If $\Gamma \iccts (\lambda x : T_1 \cap \ldots \cap T_n\ .\ e) \leadsto (\lambda x : T_1 \cap \ldots \cap T_n\ .\ e') : T_1 \cap \ldots \cap T_n \rightarrow T$, then $\Gamma, x : T_1 \cap \ldots \cap T_n \iccts e \leadsto e' : T$.
    By the induction hypothesis, $e'$ either doesn't contain casts or contains only identity casts.
    As the rule doesn't introduze new casts, then it is proved.
    \item Rule $App$.
    If $\Gamma \iccts e_1\ e_2 \leadsto e_1'\ e_2': T$, then $\Gamma \iccts e_1 \leadsto e_1' : PM$ and $PM \rhd T_1 \cap \ldots \cap T_n \rightarrow T$ and $\Gamma \iccts e_2 \leadsto e_2' : T'_1 \cap \ldots \cap T'_n$ and $T'_1 \cap \ldots \cap T'_n \sim T_1 \cap \ldots \cap T_n$ and $instances(PM) = S_1$ and $instances(T_1 \cap \ldots \cap T_n \rightarrow T) = S_2$ and $instances(T'_1 \cap \ldots \cap T'_n) = S_3$ and $instances(T_1 \cap \ldots \cap T_n) = S_4$ and $S_1,\ S_2,\ e_1' \hookrightarrow e_1''$ and $S_3,\ S_4,\ e_2' \hookrightarrow e_2''$.
    By the induction hypothesis, both $e_1'$ as well as $e_2'$ either only have identity casts or no casts at all.
    By Lemma \ref{staticexpressionstypestatic}, $PM$ and $T_1' \cap \ldots \cap T_n'$ are static types.
    Therefore, by the definition of $\rhd$, $PM = T_1 \cap \ldots \cap T_n \rightarrow T$ and by Lemma \ref{consistencytoequality}, $T_1' \cap \ldots \cap T_n' = T_1 \cap \ldots \cap T_n$.
    Therefore by the definition of $instances(T) = \{T\}$ and $S,\ S,\ e \hookrightarrow e$, only identity casts are introduzed.
\end{itemize}
\end{proof}

\begin{lemma}[Elimination of identity casts in c]
\label{eliminationidentitycastsc}
For any cast $c$, such that $\icts c : T_F$, $initialType(c) = T_I$ then:
\begin{enumerate}
    \item $\icts \erasec{c} : T_F$ and $initialType(\erasec{c}) = T_I$.
    \item $c \icos cv \iff \erasec{c} \icos cv$.
\end{enumerate}
\end{lemma}
\begin{proof}
(1) We proceed by structural induction on $c$.\\\\
Base cases:
\begin{itemize}
    \item c = $\emptycast{T}{cl}$.
    As $\icts \emptycast{T}{cl} : T$, $initialType(\emptycast{T}{cl}) = T$ and $\erasec{c} = \emptycast{T}{cl}$, then $\icts \erasec{c} : T$ and $initialType(\erasec{c}) = T$.
    \item c = $\blamecast{T_I}{T_F}{l}{cl}$.
    As $\icts \blamecast{T_I}{T_F}{l}{cl} : T_F$, $initialType(\blamecast{T_I}{T_F}{l}{cl}) = T_I$ and $\erasec{c} = \blamecast{T_I}{T_F}{l}{cl}$, then $\icts \erasec{c} : T_F$ and $initialType(\erasec{c}) = T_I$.
\end{itemize}
Induction step:
\begin{itemize}
    \item c = $c' : T_1 \Rightarrow^l T_2\ ^{cl}$.
    There are two cases:
    \begin{itemize}
        \item $T_1 \neq T_2$.
        As $\icts c' : T_1 \Rightarrow^l T_2\ ^{cl} : T_2$ and $initialType(c' : T_1 \Rightarrow^l T_2\ ^{cl}) = initialType(c')$, then $\icts c' : T_1$.
        By the induction hypothesis, $\icts \erasec{c'} : T_1$ and $initialType(\erasec{c'}) = initialType(c')$.
        With $\erasec{c} = \erasec{c'} : T_1 \Rightarrow^l T_2\ ^{cl}$, $\icts \erasec{c} : T_2$ and $initialType(\erasec{c}) = initialType(\erasec{c'}) = initialType(c') = initialType(c)$.
        \item $T_1 = T_2$.
        As $\icts c' : T_1 \Rightarrow^l T_1\ ^{cl} : T_1$ and $initialType(c' : T_1 \Rightarrow^l T_1\ ^{cl}) = initialType(c')$ then $\icts c' : T_1$.
        By the induction hypothesis, $\icts \erasec{c'} : T_1$ and $initialType(\erasec{c'}) = initialType(c')$.
        With $\erasec{c} = \erasec{c'}$, $\icts \erasec{c} : T_1$ and $initialType(\erasec{c}) = initialType(\erasec{c'}) = initialType(c') = initialType(c)$.
    \end{itemize}
\end{itemize}
(2) We proceed by induction on the length of the derivation tree of $\icos$.\\\\
Base cases:
\begin{itemize}
    \item Rule $PushBlameC$.
    \begin{itemize}
        \item There are two cases:
        \begin{itemize}
            \item $T_1 \neq T_2$.
            As $\erasec{c} = \blamecast{T_I}{T_F}{l_1}{cl_1} : T_1 \Rightarrow^{l_2} T_2\ ^{cl_2}$ and by rule PushBlameC, $\blamecast{T_I}{T_F}{l_1}{cl_1} : T_1 \Rightarrow^{l_2} T_2\ ^{cl_2} \icos \blamecast{T_I}{T_2}{l_1}{cl_1}$ it is proved.
            \item $T_1 = T_2$.
            If $T_1 = T_2$, then by rules T-SingleC and T-BlameC, $T_F = T_1$.
            Therefore, $c = \blamecast{T_I}{T_1}{l_1}{cl_1} : T_1 \Rightarrow^{l_2} T_1\ ^{cl_2}$.
            By rule PushBlameC, $\blamecast{T_I}{T_1}{l_1}{cl_1} : T_1 \Rightarrow^{l_2} T_1\ ^{cl_2} \icos \blamecast{T_I}{T_1}{l_1}{cl_1}$.
            Since $\erasec{c} = \blamecast{T_I}{T_1}{l_1}{cl_1}$, and it is already a value, it is proved.
        \end{itemize}
        \item There are two cases:
        \begin{itemize}
            \item $T_1 \neq T_2$.
            As $c$ equals $\blamecast{T_I}{T_F}{l_1}{cl_1} : T_1 \Rightarrow^{l_2} T_2\ ^{cl_2}$ or may contain adicional identity casts, then $\erasec{c} = \blamecast{T_I}{T_F}{l_1}{cl_1} : T_1 \Rightarrow^{l_2} T_2\ ^{cl_2}$.
            By the rule PushBlameC, $\blamecast{T_I}{T_F}{l_1}{cl_1} : T_1 \Rightarrow^{l_2} T_2\ ^{cl_2} \icos \blamecast{T_I}{T_2}{l_1}{cl_1}$.
            By the rules PushBlameC and IdentityC, $c \icos \blamecast{T_I}{T_2}{l_1}{cl_1}$, then it is proved.
            \item $T_1 = T_2$.
            As $c$ equals $\blamecast{T_I}{T_F}{l_1}{cl_1} : T_1 \Rightarrow^{l_2} T_1\ ^{cl_2}$ or may contain adicional identity casts, then $\erasec{c} = \blamecast{T_I}{T_F}{l_1}{cl_1}$.
            As $\blamecast{T_I}{T_F}{l_1}{cl_1}$ is already a value, it reduced to itself.
            By rules $T-SingleC$ and $T-BlameC$, $T_F = T_1$.
            By the rules PushBlameC and IdentityC, $c \icos \blamecast{T_I}{T_F}{l_1}{cl_1}$, then it is proved.
        \end{itemize}
    \end{itemize}
    \item Rule $IdentityC$.
    \begin{itemize}
        \item By rule IdentityC, $c : T \Rightarrow^l T\ ^{cl} \icos c$.
        As $c$ is a value, it doesn't contain identity casts, therefore $\erasec{c} = c$.
        As $\erasec{c}$ is already a value, it reduces to itself, therefore it is proved.
        \item As $c$ equals $c' : T \Rightarrow^l T\ ^{cl}$ or may contain adicional identity casts, then $\erasec{c} = c'$.
        As $c'$ is already a vaue, it reduced to itself.
        By rules IdentityC, $c \icos c'$, then it is proved.
    \end{itemize}
    \item Rule $SucceedC$.
    \begin{itemize}
        \item By rule SucceedC, $c : G \Rightarrow^{l_1} \Dyn\ ^{cl_1} : \Dyn \Rightarrow^{l_2} G\ ^{cl_2} \icos c$.
        As $c$ is already a value, then it doesn't contain identity casts, so $\erasec{c} = c : G \Rightarrow^{l_1} \Dyn\ ^{cl_1} : \Dyn \Rightarrow^{l_2} G\ ^{cl_2}$.
        Therefore, $\erasec{c} \icos c$.
        \item As $c$ equals $c' : G \Rightarrow^{l_1} \Dyn\ ^{cl_1} : \Dyn \Rightarrow^{l_2} G\ ^{cl_2}$ or may contain adicional identity casts, then $\erasec{c} = c' : G \Rightarrow^{l_1} \Dyn\ ^{cl_1} : \Dyn \Rightarrow^{l_2} G\ ^{cl_2}$.
        By rule SucceedC, $c' : G \Rightarrow^{l_1} \Dyn\ ^{cl_1} : \Dyn \Rightarrow^{l_2} G\ ^{cl_2} \icos c'$.
        By rules SucceedC and IdentityC, $c \icos c'$, then it is proved.
    \end{itemize}
    \item Rule $FailC$.
    \begin{itemize}
        \item By rule FailC, $c : G_1 \Rightarrow^{l_1} \Dyn\ ^{cl_1} : \Dyn \Rightarrow^{l_2} G_2\ ^{cl_2} \icos \blamecast{T_I}{G_2}{l_2}{cl_1}$.
        As $c$ is already a value, then it doesn't contain identity casts, so $\erasec{c} = c : G_1 \Rightarrow^{l_1} \Dyn\ ^{cl_1} : \Dyn \Rightarrow^{l_2} G_2\ ^{cl_2}$.
        Therefore, $\erasec{c} \icos \blamecast{T_I}{G_2}{l_2}{cl_1}$.
        \item As $c$ equals $c' : G_1 \Rightarrow^{l_1} \Dyn\ ^{cl_1} : \Dyn \Rightarrow^{l_2} G_2\ ^{cl_2}$ or may contain adicional identity casts, then $\erasec{c} = c' : G_1 \Rightarrow^{l_1} \Dyn\ ^{cl_1} : \Dyn \Rightarrow^{l_2} G_2\ ^{cl_2}$.
        By rule FailC, $c' : G_1 \Rightarrow^{l_1} \Dyn\ ^{cl_1} : \Dyn \Rightarrow^{l_2} G_2\ ^{cl_2} \icos \blamecast{T_I}{G_2}{l_2}{cl_1}$.
        By rules FailC and IdentityC, $c \icos \blamecast{T_I}{G_2}{l_2}{cl_1}$, then it is proved.
    \end{itemize}
    \item Rule $GroundC$.
    \begin{itemize}
        \item By rule GroundC, $c : T \Rightarrow^l \Dyn\ ^{cl} \icos c : T \Rightarrow^l G : G \Rightarrow^l \Dyn\ ^{cl}$.
        As $c$ is a value, it doesn't contain identity casts, therefore $\erasec{c} = c : T \Rightarrow^l \Dyn\ ^{cl}$.
        Therefore $\erasec{c} \icos c : T \Rightarrow^l G : G \Rightarrow^l \Dyn\ ^{cl}$.
        \item As $c$ equals $c' : T \Rightarrow^l \Dyn\ ^{cl}$ or may contain adicional identity casts, then $\erasec{c} = c' : T \Rightarrow^l \Dyn\ ^{cl}$.
        By rule GroundC, $c' : T \Rightarrow^l \Dyn\ ^{cl} \icos c' : T \Rightarrow^l G : G \Rightarrow^l \Dyn\ ^{cl}$.
        By rules GroundC and IdentityC, $c \icos c' : T \Rightarrow^l G : G \Rightarrow^l \Dyn\ ^{cl}$, then it is proved.
    \end{itemize}
    \item Rule $ExpandC$.
    \begin{itemize}
        \item By rule ExpandC, $c : \Dyn \Rightarrow^l T\ ^{cl} \icos c : \Dyn \Rightarrow^l G : G \Rightarrow^l T\ ^{cl}$.
        As $c$ is a value, it doesn't contain identity casts, therefore $\erasec{c} = c : \Dyn \Rightarrow^l T\ ^{cl}$.
        Therefore $\erasec{c} \icos c : \Dyn \Rightarrow^l G : G \Rightarrow^l T\ ^{cl}$.
        \item As $c$ equals $c' : \Dyn \Rightarrow^l T\ ^{cl}$ or may contain adicional identity casts, then $\erasec{c} = c' : \Dyn \Rightarrow^l T\ ^{cl}$.
        By rule ExpandC, $c' : \Dyn \Rightarrow^l T\ ^{cl} \icos c' : \Dyn \Rightarrow^l G : G \Rightarrow^l T\ ^{cl}$.
        By rules ExpandC and IdentityC, $c \icos c' : \Dyn \Rightarrow^l G : G \Rightarrow^l T\ ^{cl}$, then it is proved.
    \end{itemize}
\end{itemize}
Induction step:
\begin{itemize}
    \item Rule $EvaluateC$.
    \begin{itemize}
        \item There are two cases:
        \begin{itemize}
            \item $T_1 \neq T_2$.
            By rule EvaluateC, $c \icos c'$.
            By the induction hypothesis, $\erasec{c} \icos c'$.
            As $\erasec{c}$ equals $\erasec{c} : T_1 \Rightarrow^l T_2\ ^{cl}$, then by rule EvaluateC, $\erasec{c} \icos c' : T_1 \Rightarrow T_2\ ^{cl}$.
            \item $T_1 = T_2$.
            By the induction hypothesis, as $c \icos cv'$, then $\erasec{c} \icos cv'$.
            By rule EvaluateC, $c : T_1 \Rightarrow^l T_1\ ^{cl} \icos cv' : T_1 \Rightarrow^l T_1\ ^{cl}$.
            However, as $cv' : T_1 \Rightarrow^l T_1\ ^{cl}$ is not a value, the rule IdentityC must be applied, therefore $c : T_1 \Rightarrow^l T_1\ ^{cl} \icos cv'$.
            As $\erasec{c} \icos cv'$, then it is proved.
        \end{itemize}
        \item There are two cases:
        \begin{itemize}
            \item $T_1 \neq T_2$. As $c$ equals $c' : T_1 \Rightarrow^l T_2\ ^{cl}$ or may contain aditional identity casts, then $\erasec{c} = \erasec{c'} : T_1 \Rightarrow^l T_2\ ^{cl}$.
            By rule EvaluateC, $\erasec{c'} : T_1 \Rightarrow^l T_2\ ^{cl} \icos c'' : T_1 \Rightarrow^l T_2\ ^{cl}$.
            By rule EvaluateC, $\erasec{c'} \icos c''$, then by the induction hypothesis $c' \icos c''$.
            Therefore, by rules EvaluateC and IdentityC, $c \icos c'' : T_1 \Rightarrow^l T_2\ ^cl$, then it is proved.
            \item $T_1 = T_2$. As $c$ equals $c' : T_1 \Rightarrow^l T_1\ ^{cl}$ or may contain aditional identity casts, then $\erasec{c} = \erasec{c'}$.
            By rule EvaluateC, $\erasec{c'} \icos c''$, then by the induction hypothesis $c' \icos c''$.
            By rules EvaluateC and IdentityC, $c' : T_1 \Rightarrow^l T_1\ ^{cl} \icos c''$, then it is proved.
        \end{itemize}
    \end{itemize}
\end{itemize}
\end{proof}

\begin{lemma}[Elimination of identity casts in e]
\label{eliminationidentitycastse}
Depends on Lemma \ref{eliminationidentitycastsc}.
For any expression $e$, such that $\Gamma \iccts e : T$:
\begin{enumerate}
    \item $\Gamma \iccts \erasee{e} : T$.
    \item If $\Gamma \iccts e' \leadsto e$ for some $e'$ then $e \gios v \iff \erasee{e} \gios v$.
\end{enumerate}
\end{lemma}
\begin{proof}
(1) We proceed by induction on the length of the derivation tree of $\Gamma \iccts e : T$.\\\\
Base cases:
\begin{itemize}
    \item Rule $Var$.
    As $x$ doesn't contain casts, then $\erasee{e} = x$.
    Therefore it is proved.
    \item Rule $Int$.
    As $n$ doesn't contain casts, then $\erasee{e} = n$.
    Therefore it is proved.
    \item Rule $True$.
    As $\true$ doesn't contain casts, then $\erasee{e} = \true$.
    Therefore it is proved.
    \item Rule $False$.
    As $\false$ doesn't contain casts, then $\erasee{e} = \false$.
    Therefore it is proved.
    \item Rule $T{-}Blame$.
    As $\blame{T}{l}$ doesn't contain casts, then $\erasee{e} = \blame{T}{l}$.
    Therefore it is proved.
\end{itemize}
Induction step:
\begin{itemize}
    \item Rule ${\rightarrow} I$.
    If $\Gamma \iccts \lambda x : T_1 \cap \ldots \cap T_n\ .\ e : T_1 \cap \ldots \cap T_n \rightarrow T$, then $\Gamma, x : T_1 \cap \ldots \cap T_n \iccts e : T$.
    By the induction hypothesis, $\Gamma, x : T_1 \cap \ldots \cap T_n \iccts \erasee{e} : T$.
    As $\erasee{e} = \lambda x : T_1 \cap \ldots \cap T_n\ .\ \erasee{e}$, then $\Gamma \iccts \erasee{e} : T_1 \cap \ldots \cap T_n \rightarrow T$.
    \item Rule ${\rightarrow} I'$.
    If $\Gamma \iccts \lambda x : T_1 \cap \ldots \cap T_n\ .\ e : T_i \rightarrow T$, then $\Gamma, x : T_i \iccts e : T$.
    By the induction hypothesis, $\Gamma, x : T_i \iccts \erasee{e} : T$.
    As $\erasee{e} = \lambda x : T_1 \cap \ldots \cap T_n\ .\ \erasee{e}$, then $\Gamma \iccts \erasee{e} : T_i \rightarrow T$.
    \item Rule ${\rightarrow} E$.
    If $\Gamma \iccts e_1\ e_2 : T$, then $\Gamma \iccts e_1 : PM$, $PM \rhd T_1 \cap \ldots \cap T_n \rightarrow T$, $\Gamma \iccts e_2 : T_1' \cap \ldots \cap T_n'$ and $T_1' \cap \ldots \cap T_n' \sim T_1 \cap \ldots \cap T_n$.
    By the induction hypothesis, $\Gamma \iccts \erasee{e_1} : PM$ and $\Gamma \iccts \erasee{e_2} : T_1' \cap \ldots \cap T_n'$.
    As $\erasee{e} = \erasee{e_1}\ \erasee{e_2}$, therefore $\Gamma \iccts \erasee{e} : T$.
    \item Rule ${\cap} I$.
    If $\Gamma \iccts e : T_1 \cap \ldots \cap T_n$, then $\Gamma \iccts e : T_1$ and ... and $\Gamma \iccts e : T_n$.
    By the induction hypothesis, $\Gamma \iccts \erasee{e} : T_1$ and ... and $\Gamma \iccts \erasee{e} : T_n$.
    Therefore $\Gamma \iccts \erasee{e} : T_1 \cap \ldots \cap T_n$.
    \item Rule ${\cap} E$.
    If $\Gamma \iccts e : T_i$, then $\Gamma \iccts e : T_1 \cap \ldots \cap T_n$, such that $T_i \in \{T_1, \ldots, T_n\}$.
    By the induction hypothesis, $\Gamma \iccts \erasee{e} : T_1 \cap \ldots \cap T_n$.
    Therefore $\Gamma \iccts \erasee{e} : T_i$.
    \item Rule $T{-}App$.
    If $\Gamma \iccts e_1\ e_2 : T_{12} \cap \ldots \cap T_{n2}$, then $\Gamma \iccts e_1 : T_{11} \rightarrow T_{12} \cap \ldots \cap T_{n1} \rightarrow T_{n2}$ and $\Gamma \iccts e_2 : T_1' \cap \ldots \cap T_n'$ and $T_{11} \sim T_1' \ldots T_{n1} \sim T_n'$.
    By the induction hypothesis, $\Gamma \iccts \erasee{e_1} : T_{11} \rightarrow T_{12} \cap \ldots \cap T_{n1} \rightarrow T_{n2}$ and $\Gamma \iccts \erasee{e_2} : T_1' \cap \ldots \cap T_n'$.
    As $\erasee{e} = \erasee{e_1}\ \erasee{e_2}$, therefore $\Gamma \iccts \erasee{e} : T_{12} \cap \ldots \cap T_{n2}$.
    \item Rule $T{-}Cast$.
    There are two possibilities:
    \begin{itemize}
        \item $T_1 \neq T_2$.
        If $\Gamma \iccts e' : T_1 \Rightarrow^l T_2 : T_2$, then $\Gamma \iccts e' : T_1$.
        By the induction hypothesis, $\Gamma \iccts \erasee{e'} : T_1$.
        As $\erasee{e} = \erasee{e'} : T_1 \Rightarrow^l T_2$, then $\Gamma \iccts \erasee{e} : T_2$.
        \item $T_1 = T_2$.
        If $\Gamma \iccts e' : T_1 \Rightarrow^l T_1 : T_1$, then $\Gamma \iccts e' : T_1$.
        By the induction hypothesis, $\Gamma \iccts \erasee{e'} : T_1$.
        As $\erasee{e} = \erasee{e'}$, then $\Gamma \iccts \erasee{e} : T_1$.
    \end{itemize}
    \item Rule $T{-}IntersectionCast$.
    If $\Gamma \iccts e' : c_1 \cap \ldots \cap c_n : T_1 \cap \ldots \cap T_n$, then $\Gamma \iccts e' : T$, $\icts c_1 : T_1$ and ... and $\icts c_n : T_n$ and $initialType(c_1) \cap \ldots \cap initialType(c_n) = T$.
    By the induction hypothesis, $\Gamma \iccts \erasee{e'} : T$.
    We now have 2 possibilities:
    \begin{itemize}
        \item $\neg(\forall i \in 1..n\ .\ isEmptyCast\ \erasec{c_i})$:
        For all casts $c_i$, with $i \in 1..n$, that don't contain identity casts, then $\erasec{c_i} = c_i$, therefore $\icts \erasec{c_i} : T_i$ and $initialType(\erasec{c_i}) = initialType(c_i)$.
        For the remaining casts, by Lemma \ref{eliminationidentitycastsc}, $\icts \erasec{c_i} : T_i$ and $initialType(\erasec{c_i}) = initialType(c_i)$.
        Therefore, with $\erasee{e} = \erasee{e'} : \erasec{c_1} \cap \ldots \cap \erasec{c_n}$, $\Gamma \iccts \erasee{e} : T_1 \cap \ldots \cap T_n$.
        \item $\forall i \in 1..n\ .\ isEmptyCast\ \erasec{c_i}$:
        As all casts are empty casts, then for all casts $\erasec{c_i}$, by Lemma \ref{eliminationidentitycastsc} and by rule T-EmptyC, $\icts \erasec{c_i} : T_i$ and $initialType(\erasec{c_i}) = T_i$.
        Therefore $\erasee{e} = \erasee{e'}$.
        We now have two possibilities:
        \begin{itemize}
            \item If $T$ is not an intersection type, then $T_1 = \ldots = T_n = T$ and by idempotence of $\cap$, we have that $\Gamma \iccts \erasee{e} : T_1 \cap \ldots \cap T_n$.
            \item If $T$ is an intersection type, then $T = T_1 \cap \ldots \cap T_n$.
            Therefore $\Gamma \iccts \erasee{e} : T_1 \cap \ldots \cap T_n$.
        \end{itemize}
    \end{itemize}
\end{itemize}
(2) We proceed by induction on the length of the derivation tree of $\gios$.\\\\
Base cases:
\begin{itemize}
    \item Rule $\beta{-}reduction$.
    With $\erasee{e} = (\lambda x : T\ .\ \erasee{e'})\ v$.
    \item Rule $SimulateArrow{\cap}$.
    As $v_1 : cv_1 \cap \ldots \cap cv_n$ and $v_2$ are values, then $e$ doesn't contain identity casts.
    As $\erasee{e} = e$, then it is proved.
    \item Rule $MergeIC{\cap}$.
    This case is not considered due to the fact that cast insertion doesn't introduce such expressions.
    \item Rule $MergeCI{\cap}$.
    This case is not considered due to the fact that cast insertion doesn't introduce such expressions.
    \item Rule $MergeII{\cap}$.
    This case is not considered due to the fact that cast insertion doesn't introduce such expressions.
    \item Rule $EvaluateCasts{\cap}$.
    \begin{itemize}
        \item By rule EvaluateCasts$\cap$, $v : c_1 \cap \ldots \cap c_n \gios v : cv_1 \cap \ldots \cap cv_n$, with $c_1 \icos cv_1$ and ... and $c_n \icos cv_n$.
        With $\erasee{e} = v : \erasec{c_1} \cap \ldots \cap \erasec{c_n}$, by Lemma \ref{eliminationidentitycastsc}, $\erasec{c_1} \icos cv_1$ and ... and $\erasec{c_n} \icos cv_n$.
        Therefore, by rule EvaluateCasts$\cap$, $v : \erasec{c_1} \cap \ldots \cap \erasec{c_n} \gios v : cv_1 \cap \ldots \cap cv_n$.
        \item By rule EvaluateCasts$\cap$, $v : \erasec{c_1} \cap \ldots \cap \erasec{c_n} \gios v : cv_1 \cap \ldots \cap cv_n$, with $\erasec{c_1} \icos cv_1$ and ... and $\erasec{c_n} \icos cv_n$.
        With $\erasee{e} = v : \erasec{c_1} \cap \ldots \cap \erasec{c_n}$, by Lemma \ref{eliminationidentitycastsc}, $c_1 \icos cv_1$ and ... and $c_n \icos cv_n$.
        Therefore, by rule EvaluateCasts$\cap$, $v : c_1 \cap \ldots \cap c_n \gios v : cv_1 \cap \ldots \cap cv_n$.
    \end{itemize}
    \item Rule $PropagateBlame{\cap}$.
    This case is not considered due to the fact that cast insertion doesn't introduce such expressions.
    \item Rule $RemoveEmpty{\cap}$.
    This case is not considered due to the fact that cast insertion doesn't introduce such expressions.
\end{itemize}
Induction step:
\begin{itemize}
    \item e =
\end{itemize}
\end{proof}

\begin{theorem}[Conservative Extension]
\label{conservative_extension_operationalsemantics}
Depends on Lemmas \ref{staticcompilation} and \ref{eliminationidentitycastse}.
If e is fully static, T is a static type and $\Gamma \iccts e \leadsto e' : T$, then $e \sios v \iff e' \gios v$.
\end{theorem}
\begin{proof}
By Lemma \ref{staticcompilation}, we have that $\erasee{e'} = e$.
By Lemma \ref{eliminationidentitycastse}, we have that $e' \gios v \iff \erasee{e'} \gios v$.
As $\erasee{e'} = e$ and $e$ doesn't contain casts, then we can evaluate $e$ using just the reduction rules (of the gradual operational semantics) analogous to the static operational semantics' reduction rules.
Therefore, we have that $e' \gios v \iff e \sios v$.
\end{proof}

\begin{lemma}[Subject reduction of $\beta{-}reduction$]
\label{subjectreductionbeta}
If $e$ is a redex and $e'$ is its contractum, then $\Gamma \iccts e : T \Rightarrow \Gamma \iccts e' : T$.
\end{lemma}
\begin{proof}
Let $e = (\lambda x : T_1 \cap \ldots \cap T_n\ .\ e_1)\ e_2$.
There exists a type $T_1 \cap \ldots \cap T_n$ such that we can deduce $\Gamma \iccts e : T$ from $\Gamma \iccts \lambda x : T_1 \cap \ldots \cap T_n\ .\ e_1 : T_1 \cap \ldots \cap T_n \rightarrow T$ and $\Gamma \iccts e_2 : T_1 \cap \ldots \cap T_n$ ($x$ does not occur in $\Gamma$).
Moreover, $\Gamma \iccts \lambda x : T_1 \cap \ldots \cap T_n\ .\ e_1 : T_1 \cap \ldots \cap T_n \rightarrow T$ only if $\Gamma, x : T_1 \cap \ldots \cap T_n \iccts e_1 : T$.
By definition, $e' = [x \mapsto e_2]\ e_1$.
To obtain $\Gamma \iccts [x \mapsto e_2]\ e_1 : T$, it is sufficient to replace, in the proof of $\Gamma, x : T_1 \cap \ldots \cap T_n \iccts e_1 : T$, the statements $x : T_i$ (introduzed by the rules $Var$ and ${\cap}E$) by the deductions of $\Gamma \iccts e_2 : T_i$ for $1 \leq i \leq n$.
Proof adapted from \cite{coppo1980extension}.
\end{proof}

\begin{lemma}[Subject reduction of $\icos$]
\label{subjectreductionIC}
If $\icts c : T$ for some $T$ and $c \icos c'$ then $\icts c' : T$.
\end{lemma}
\begin{proof}
We proceed by induction on the length of the derivation tree of $\icos$.\\\\
Base cases:
\begin{itemize}
    \item Rule $PushBlameC$.
    $\icts \blamecast{T_I}{T_F}{l_1}{cl_1} : T_1 \Rightarrow^{l_2} T_2\ ^{cl_2} : T_2$ and by rule PushBlameC, $\blamecast{T_I}{T_F}{l_1}{cl_1} : T_1 \Rightarrow^{l_2} T_2\ ^{cl_2} \icos \blamecast{T_I}{T_2}{l_1}{cl_1}$.
    As $\icts \blamecast{T_I}{T_2}{l_1}{cl_1} : T_2$, then it is proved.
    \item Rule $IdentityC$.
    If $\icts c : T \Rightarrow^l T\ ^{cl} : T$, then $\icts c : T$.
    By rule IdentityC, $c : T \Rightarrow^l T\ ^{cl} \icos c$.
    Therefore it is proved.
    \item Rule $SucceedC$.
    If $\icts c : G \Rightarrow^{l_1} \Dyn\ ^{cl_1} : \Dyn \Rightarrow^{l_2} G\ ^{cl_2} : G$, then $\icts c : G$.
    By rule SucceedC, $c : G \Rightarrow^{l_1} \Dyn\ ^{cl_1} : \Dyn \Rightarrow^{l_2} G\ ^{cl_2} \icos c$.
    Therefore it is proved.
    \item Rule $FailC$.
    If $\icts c : G_1 \Rightarrow^{l_1} \Dyn\ ^{cl_1} : \Dyn \Rightarrow^{l_2} G_2\ ^{cl_2} : G_2$, and by rule FailC, $c : G_1 \Rightarrow^{l_1} \Dyn\ ^{cl_1} : \Dyn \Rightarrow^{l_2} G_2\ ^{cl_2} \icos \blamecast{T_I}{G_2}{l_2}{cl_1}$ and $\icts \blamecast{T_I}{G_2}{l_2}{cl_1} : G_2$, it is proved.
    \item Rule $GroundC$.
    If $\icts c : T \Rightarrow^l \Dyn\ ^{cl} : \Dyn$ then $\icts c : T$.
    By rule GroundC, $c : T \Rightarrow^l \Dyn\ ^{cl} \icos c : T \Rightarrow^l G\ ^{cl} : G \Rightarrow^l \Dyn\ ^{cl}$.
    As $\icts c : T \Rightarrow^l G\ ^{cl} : G \Rightarrow^l \Dyn\ ^{cl} : \Dyn$, it is proved.
    \item Rule $ExpandC$.
    If $\icts c : \Dyn \Rightarrow^l T\ ^{cl} : T$ then $\icts c : \Dyn$.
    By rule ExpandC, $c : \Dyn \Rightarrow^l T\ ^{cl} \icos c : \Dyn \Rightarrow^l G\ ^{cl} : G \Rightarrow^l T\ ^{cl}$.
    As $\icts c : \Dyn \Rightarrow^l G\ ^{cl} : G \Rightarrow^l T\ ^{cl} : T$, it is proved.
\end{itemize}
Induction step:
\begin{itemize}
    \item Rule $EvaluateC$.
    If $\icts c : T_1 \Rightarrow^l T_2\ ^{cl} : T_2$ then $\icts c : T_1$.
    By rule EvaluateC, $c \icos c'$.
    By the induction hypothesis, $\icts c' : T_1$.
    By rule EvaluateC, $c : T_1 \Rightarrow^l T_2\ ^{cl} \icos c' : T_1 \Rightarrow^l T_2\ ^{cl}$.
    As $\icts c' : T_1 \Rightarrow^l T_2\ ^{cl} : T_2$ it is proved.
\end{itemize}
\end{proof}

\begin{lemma}[Initial type preservation of $\icos$]
\label{initialtypepreservationIC}
If $initialType(c) = T$ for some $T$ and $c \icos c'$ then $initialType(c') = T$.
\end{lemma}
\begin{proof}
We proceed by induction on the length of the derivation tree of $\icos$.\\\\
Base cases:
\begin{itemize}
    \item Rule $PushBlameC$.
    By the definition of $initialType$, $initialType(\blamecast{T_I}{T_F}{l_1}{cl_1} : T_1 \Rightarrow^{l_2} T_2\ ^{cl_2}) = T_I$.
    By rule PushBlameC, $\blamecast{T_I}{T_F}{l_1}{cl_1} : T_1 \Rightarrow^{l_2} T_2\ ^{cl_2} \icos \blamecast{T_I}{T_2}{l_1}{cl_1}$.
    Since $initialType(\blamecast{T_I}{T_2}{l_1}{cl_1}) = T_I$, it is proved.
    \item Rule $IdentityC$.
    By the definitions of $initialType$, $initialType(c : T \Rightarrow^l T\ ^{cl}) = initialType(c)$.
    By rule IdentityC, $c : T \Rightarrow^l T\ ^{cl} \icos c$.
    Therefore it is proved.
    \item Rule $SucceedC$.
    By the definition of $initialType$, $initialType(c : G \Rightarrow^{l_1} \Dyn\ ^{cl_1} : \Dyn \Rightarrow^{l_2} G\ ^{cl_2}) = initialType(c)$.
    By rule SucceedC, $c : G \Rightarrow^{l_1} \Dyn\ ^{cl_1} : \Dyn \Rightarrow^{l_2} G\ ^{cl_2} \icos c$.
    Therefore it is proved.
    \item Rule $FailC$.
    By the definition of $initialType$, $initialType(c : G_1 \Rightarrow^{l_1} \Dyn\ ^{cl_1} : \Dyn \Rightarrow^{l_2} G_2\ ^{cl_2}) = T_I$.
    By rule FailC, $c : G_1 \Rightarrow^{l_1} \Dyn\ ^{cl_1} : \Dyn \Rightarrow^{l_2} G_2\ ^{cl_2} \icos \blamecast{T_I}{G_2}{l_2}{cl_1}$.
    Since $initialType(\blamecast{T_I}{G_2}{l_2}{cl_1}) = T_I$, it is proved.
    \item Rule $GroundC$.
    By the definition of $initialType$, $initialType(c : T \Rightarrow^l \Dyn\ ^{cl}) = initialType(c)$.
    By rule GroundC, $c : T \Rightarrow^l \Dyn\ ^{cl} \icos c : T \Rightarrow^l G\ ^{cl} : G \Rightarrow^l \Dyn\ ^{cl}$.
    Since $initialType(c : T \Rightarrow^l G\ ^{cl} : G \Rightarrow^l \Dyn\ ^{cl}) = initialType(c)$, it is proved.
    \item Rule $ExpandC$.
    By the definition of $initialType$, $initialType(c : \Dyn \Rightarrow^l T\ ^{cl}) = initialType(c)$.
    By rule ExpandC, $c : \Dyn \Rightarrow^l T\ ^{cl} \icos c : \Dyn \Rightarrow^l G\ ^{cl} : G \Rightarrow^l T\ ^{cl}$.
    Since $initialType(c : \Dyn \Rightarrow^l G\ ^{cl} : G \Rightarrow^l T\ ^{cl}) = initialType(c)$, it is proved.
\end{itemize}
Induction step:
\begin{itemize}
    \item Rule $EvaluateC$.
    By the definition of $initialType$, $initialType(c : T_1 \Rightarrow^l T_2\ ^{cl}) = initialType(c)$.
    By rule EvaluateC, $c \icos c'$.
    By the induction hypothesis, $initialType(c') = initialType(c)$.
    By rule EvaluateC, $c : T_1 \Rightarrow^l T_2\ ^{cl} \icos c' : T_1 \Rightarrow^l T_2\ ^{cl}$.
    Since $initialType(c' : T_1 \Rightarrow^l T_2\ ^{cl}) = initialType(c')$, it is proved.
\end{itemize}
\end{proof}

\begin{theorem}[Subject reduction of $\gios$]
\label{subjectreduction}
Depends on Lemmas \ref{subjectreductionbeta}, \ref{subjectreductionIC} and \ref{initialtypepreservationIC}.
If $\Gamma \iccts e : T$ and $e \gios e'$ then $\Gamma \iccts e' : T$.
\end{theorem}
\begin{proof}
We proceed by induction on the length of the derivation tree of $\gios$.\\\\
Base case:
\begin{itemize}
    \item Rule $\beta{-}reduction$.
    The proof of $\Gamma \iccts e' : T$ can be obtained from that of $\Gamma \iccts e : T$ by replacing any deduction of a type for $e$, by the corresponding deduction of the same type for $e'$ (by Lemma \ref{subjectreductionbeta}).
    \item Rule $Simulate{\cap}$.
    If $\Gamma \iccts (v_1 : cv_1 \cap \ldots \cap cv_n)\ v_2 : T_{12} \cap \ldots \cap T_{n2}$, then $\Gamma \iccts v_1 : cv_1 \cap \ldots \cap cv_n : T_1 \cap \ldots \cap T_n$ with $\icts cv_1 : T_1$ and ... and $\icts cv_n : T_n$, such that $\exists i \in 1 .. n\ .\ T_i = T_{i1} \rightarrow T_{i2}$ and $\Gamma \iccts v_1 : T_1' \cap \ldots \cap T_l'$ and $I_1 = initialType(cv_1)$ and ... and $I_n = initialType(cv_n)$ such that either $T_1' \cap \ldots \cap T_l' = I_1 \cap \ldots \cap I_n$ or $\{I_1, \ldots, I_n\} \subset \{T_1', \ldots, T_l'\}$ and $\Gamma \iccts v_2 : T_{11} \cap \ldots \cap T_{n1}$.
    For the sake of simplicity lets elide cast labels and blame labels.
    As $\icts cv_1' : T_{11} \rightarrow T_{12}$ and ... and $\icts cv_m' : T_{m1} \rightarrow T_{m2}$ then $cv_1' = cv_1'' : T_{11}' \rightarrow T_{12}' \Rightarrow T_{11} \rightarrow T_{12}$ and ... and $cv_m' = cv_m'' : T_{m1}' \rightarrow T_{m2}' \Rightarrow T_{m1} \rightarrow T_{m2}$.
    By the definition of simulateArrow, $c_{11} : \emptycast{T_{11}}{} : T_{11} \Rightarrow T_{11}'$ and ... and $c_{m1} = \emptycast{T_{m1}}{} : T_{m1} \Rightarrow T_{m1}'$ and $c_{12} : \emptycast{T_{12}'}{} : T_{12}' \Rightarrow T_{12}$ and ... and $c_{m2} = \emptycast{T_{m2}'}{} : T_{m2}' \Rightarrow T_{m2}$ and $initialType(r_1) = I_1$ and ... and $initialType(r_m) = I_m$ and $\icts r_1 : T_{11}' \rightarrow T_{12}'$ and ... and $\icts r_m : T_{m1}' \rightarrow T_{m2}'$.
    Therefore $\Gamma \iccts v_1 : r_1 \cap \ldots \cap r_m : T_{11}' \rightarrow T_{12}' \cap \ldots \cap T_{m1}' \rightarrow T_{m2}'$ and $\Gamma \iccts v_2 : c_{11} \cap \ldots \cap c_{m1} : T_{11}' \cap \ldots \cap T_{m1}'$ and therefore $\Gamma \iccts (v_1 : r_1 \cap \ldots \cap r_m)\ (v_2 : c_{11} \cap \ldots \cap c_{m1}) : T_{12}' \cap \ldots \cap T_{m2}'$.
    Therefore, $\Gamma \iccts (v_1 : r_1 \cap \ldots \cap r_m)\ (v_2 : c_{11} \cap \ldots \cap c_{m1}) : c_{12} \cap \ldots \cap c_{m2} : T_{12} \cap \ldots \cap T_{m2}$, such that $\{T_{12}, \ldots, T_{m2}\} \subset \{T_{12}, \ldots, T_{n2}\}$.
    By rule Simulate$\cap$, $(v_1 : cv_1 \cap \ldots \cap cv_n)\ v_2 \gios\\ (v_1 : r_1 \cap \ldots \cap r_m)\ (v_2 : c_{11} \cap \ldots \cap c_{m1}) : c_{12} \cap \ldots \cap c_{m2}$, therefore it is proved.
    \item Rule $MergeIC{\cap}$.
    If $\Gamma \iccts v : cv_1 \cap \ldots \cap cv_n : T_1 \Rightarrow^l T_2 : T_2$ then $\Gamma \iccts v : cv_1 \cap \ldots \cap cv_n : T_{11} \cap \ldots \cap T_{1n}$ and such that $\exists T_{1i}\ .\ T_{1i} = T_1$ and $\icts cv_1 : T_{11}$ and $I_1 = initialType(cv_1)$ and ... and $\icts cv_n : T_{1n}$ and $I_n = initialType(cv_n)$ and $\Gamma \iccts v : I_1 \cap \ldots \cap I_n$ and $m \leq n$.
    By the definition of mergeIC, $\icts c_1' : T_2$ and $initialType(c_1') : I_1$ and ... and $\icts c_m' : T_2$ and $initialType(c_m') : I_m$.
    As $\Gamma \iccts v : I_1 \cap \ldots \cap I_m$ and therefore $\Gamma \iccts v : c_1' \cap \ldots \cap c_m' : T_2 \cap \ldots \cap T_2$ and $T_2 \cap \ldots \cap T_2 = T_2$ and by rule MergeIC$\cap$, $v : cv_1 \cap \ldots \cap cv_n : T_1 \Rightarrow^l T_2 \gios v : c_1' \cap \ldots \cap c_m'$, then it is proved.
    \item Rule $MergeCI{\cap}$.
    If $\Gamma \iccts v : T_1 \Rightarrow^l T_2 : c_1 \cap \ldots \cap c_n : F_1 \cap \ldots \cap F_n$ then $\Gamma \iccts v : T_1 \Rightarrow T_2 : T_2$ and $\Gamma \iccts v : T_1$ and $\icts c_1 : F_1$ and $initialType(c_1) = T_2$ and ... and $\icts c_n : F_n$ and $initialType(c_n) = T_2$.
    By the definition of mergeCI, $mergeCI(v : T_1 \Rightarrow^l T_2 : c_1 \cap \ldots \cap c_n) = v : c_1' \cap \ldots \cap c_n'$, such that $\icts c_1' : F_1$ and $initialType(c_1') : T_1$ and ... and $\icts c_n' : F_n$ and $initialType(c_n') : T_1$.
    As $\Gamma \iccts v : c_1' \cap \ldots \cap c_n' : F_1 \cap \ldots \cap F_n$ and by rule MergeCI$\cap$, $v : T_1 \Rightarrow^l T_2 : c_1 \cap \ldots \cap c_n \gios v : c_1' \cap \ldots \cap c_n'$, then it is proved.
    \item Rule $MergeII{\cap}$.
    If $\Gamma \iccts v : cv_1 \cap \ldots \cap cv_n : c_1' \cap \ldots \cap c_m' : F_1' \cap \ldots \cap F_m'$ then $\icts c_1' : F_1'$ and $initialType(c_1') = I_1'$ and ... and $\icts c_m' : F_m'$ and $initialType(c_m') = I_m'$ and $\Gamma \iccts v : cv_1 \cap \ldots \cap cv_n : F_1 \cap \ldots \cap F_n$ and $\icts cv_1 : F_1$ and $initialType(cv_1) = I_1$ and ... and $\icts cv_n : F_n$ and $initialType(cv_n) = I_n$ and $\Gamma \iccts v : T_1 \cap \ldots \cap T_l$ such that either $T_1 \cap \ldots \cap T_l = I_1 \cap \ldots \cap I_n$ or $\{I_1, \ldots, I_n\} \subset \{T_1, \ldots, T_l\}$.
    There are two possibilities:
    \begin{itemize}
        \item $F_1 \cap \ldots \cap F_n = I_1' \cap \ldots \cap I_m'$.
        By the definition of mergeII, $\icts c_1'' : F_1''$ and ... and $\icts c_j'' : F_j''$ such that $F_1'' \cap \ldots \cap F_j'' = F_1' \cap \ldots \cap F_m'$ and $initialType(c_1'') = I_1''$ and ... and $initialType(c_j'') = I_j''$ such that $I_1'' \cap \ldots \cap I_j'' = I_1 \cap \ldots \cap I_n$.
        Therefore $\Gamma \iccts v : c_1'' \cap \ldots \cap c_j'' : F_1'' \cap \ldots \cap F_j''$.
        By rule MergeII$\cap$, $v : cv_1 \cap \ldots \cap cv_n : c_1' \cap \ldots \cap c_m' \gios v : c_1'' \cap \ldots \cap c_j''$.
        Therefore it is proved.
        \item $\{I_1', \ldots, I_m'\} \subset \{F_1, \ldots, F_n\}$.
        By the definition of mergeII, $\icts c_1'' : F_1''$ and $initialType(c_1'') = I_1''$ and ... and $\icts c_j'' : F_j''$ and $initialType(c_j'') = I_j''$ such that $\{I_1'', \ldots, I_j''\} \subset \{I_1, \ldots, I_n\}$ and $\{F_1'', \ldots, F_j''\} \subset \{F_1', \ldots, F_m'\}$.
        Therefore, $\Gamma \iccts v : c_1'' \cap \ldots \cap c_j'' : F_1'' \cap \ldots \cap F_j''$.
        By rule MergeII$\cap$, $v : cv_1 \cap \ldots \cap cv_n : c_1' \cap \ldots \cap c_m' \gios v : c_1'' \cap \ldots \cap c_j''$.
        Therefore, it is proved.
    \end{itemize}
    \item Rule $EvaluateCasts{\cap}$.
    If $\Gamma \iccts v : c_1 \cap \ldots \cap c_n : T_1 \cap \ldots \cap T_n$ then $\icts c_1 : T_1$ and $I_1 = initialType(c_1)$ and ... and $\icts c_n : T_n$ and $I_n = initialType(c_n)$ and $\Gamma \iccts v : I_1 \cap \ldots \cap I_n$.
    By rule EvaluateCasts$\cap$, $c_1 \icos cv_1$ and ... and $c_n \icos cv_n$.
    By Lemmas \ref{subjectreductionIC} and \ref{initialtypepreservationIC}, $\icts cv_1 : T_1$ and $initialType(cv_1) = I_1$ and ... and $\icts cv_n : T_n$ and $initialType(cv_n) = I_n$.
    Therefore $\Gamma \iccts v : cv_1 \cap \ldots \cap cv_n : T_1 \cap \ldots \cap T_n$.
    By rule EvaluateCasts$\cap$, $v : c_1 \cap \ldots \cap c_n \gios v : cv_1 \cap \ldots \cap cv_n$, then it is proved.
    \item Rule $PropagateBlame{\cap}$.
    If $\Gamma \iccts v : \blamecast{T_1'}{T_1}{l_1}{m_1} \cap \ldots \cap \blamecast{T_n'}{T_n}{l_n}{m_n} : T_1 \cap \ldots \cap T_n$ and by rule PropagateBlame$\cap$ $v : \blamecast{T_1'}{T_1}{l_1}{m_1} \cap \ldots \cap \blamecast{T_n'}{T_n}{l_n}{m_n} \gios \blame{(T_1 \cap \ldots \cap T_n)}{l_1}$, and $\Gamma \iccts \blame{(T_1 \cap \ldots \cap T_n)}{l_1} : T_1 \cap \ldots \cap T_n$, then it is proved.
    \item Rule $RemoveEmpty{\cap}$..
    If $\Gamma \iccts v : \emptycast{T_1}{m_1} \cap \ldots \cap \emptycast{T_n}{m_n} : T_1 \cap \ldots \cap T_n$, then $\icts \emptycast{T_1}{m_1} : T_1$ and $initialType(\emptycast{T_1}{m_1}) = T_1$ and ... and $\icts \emptycast{T_n}{m_n} : T_n$ and $initialType(\emptycast{T_n}{m_n}) = T_n$ and $\Gamma \iccts v : T_1 \cap \ldots \cap T_n$.
    By rule RemoveEmpty$\cap$, $v : \emptycast{T_1}{m_1} \cap \ldots \cap \emptycast{T_n}{m_n} \gios v$, therefore it is proved.
\end{itemize}
Induction step:
\begin{itemize}
    \item Rule $E{-}App1$. If $\Gamma \iccts e_1\ e_2 : T$, then either:
    \begin{itemize}
        \item $\Gamma \iccts e_1 : PM$, $PM \rhd T_1 \cap \ldots \cap T_n \rightarrow T$, $\Gamma \iccts e_2 : T_1' \cap \ldots \cap T_n'$ and $T_1' \cap \ldots \cap T_n' \sim T_1 \cap \ldots \cap T_n$ or
        \item $\Gamma \iccts e_1 : T_{11} \rightarrow T_{12} \cap \ldots \cap T_{n1} \rightarrow T_{n2}$, $\Gamma \iccts e_2 : T_1' \cap \ldots \cap T_n'$ and $T_{11} \sim T_1' \ldots T_{n1} \sim T_n'$.
    \end{itemize}
    By rule E-App1, $e_1 \icos e_1'$, so by the induction hypothesis either:
    \begin{itemize}
        \item $\Gamma \iccts e_1' : PM$ or
        \item $\Gamma \iccts e_1' : T_{11} \rightarrow T_{12} \cap \ldots \cap T_{n1} \rightarrow T_{n2}$.
    \end{itemize}
    Therefore, $\Gamma \iccts e_1'\ e_2 : T$.
    As by rule E-App1, $e_1\ e_2 \icos e_1'\ e_2$, it is proved.
    \item Rule $E{-}App2$. If $\Gamma \iccts v_1\ e_2 : T$, then either:
    \begin{itemize}
        \item $\Gamma \iccts v_1 : PM$, $PM \rhd T_1 \cap \ldots \cap T_n \rightarrow T$, $\Gamma \iccts e_2 : T_1' \cap \ldots \cap T_n'$ and $T_1' \cap \ldots \cap T_n' \sim T_1 \cap \ldots \cap T_n$ or
        \item $\Gamma \iccts v_1 : T_{11} \rightarrow T_{12} \cap \ldots \cap T_{n1} \rightarrow T_{n2}$, $\Gamma \iccts e_2 : T_1' \cap \ldots \cap T_n'$ and $T_{11} \sim T_1' \ldots T_{n1} \sim T_n'$.
    \end{itemize}
    By rule E-App2, $e_2 \icos e_2'$, so by the induction hypothesis, $\Gamma \iccts e_2' : T_1' \cap \ldots \cap T_n'$.
    Therefore, $\Gamma \iccts v_1\ e_2' : T$.
    As by rule E-App2, $v_1\ e_2 \icos v_1\ e_2'$, it is proved.
    \item Rule $E{-}EvaluateCasts$. If $\Gamma \iccts e : c_1 \cap \ldots \cap c_n : T_1 \cap \ldots \cap T_n$, then $\Gamma \iccts e : T$, $\icts c_1 : T_1$ and ... and $\icts c_n : T_n$ and $initialType(c_1) \cap \ldots \cap initialType(c_n) = T$.
    By rule E-EvaluateCasts, $e \icos e'$, so by the induction hypothesis, $\Gamma \iccts e' : T$.
    Therefore, $\Gamma \iccts e' : c_1 \cap \ldots \cap c_n : T_1 \cap \ldots \cap T_n$.
    As by rule E-EvaluateCasts, $e : c_1 \cap \ldots \cap c_n \icos e' : c_1 \cap \ldots \cap c_n$, it is proved.
\end{itemize}
\end{proof}

\begin{theorem}[Progress of $\gios$]
\label{progress}
If $\Gamma \iccts e : T$ then $e \gios v$.
\end{theorem}

\begin{theorem}[Blame Theorem]
\label{blame_theorem}
If $\Gamma \iccts e : T$ and $e \gios \blame{T}{l}$ then l is not a safe cast of $e$.
\end{theorem}

\begin{theorem}[Gradual Guarantee]
\label{gradual_guarantee}
If $\Gamma \iccts e_1 : T_1$ and $\Gamma \iccts e_2 : T_2$ and $e_1 \sqsubseteq e_2$ then:
\begin{enumerate}
    \item if $e_2 \gios e_2'$ then $e_1 \icos e_1'$ and $e_1' \sqsubseteq e_2'$.
    \item if $e_1 \gios e_1'$ then either $e_2 \gios e_2'$ and $e_1' \sqsubseteq e_2'$ or $e_2' \gios \blame{T}{l}$.
\end{enumerate}
\end{theorem}

\bibliographystyle{unsrt}
\bibliography{bibliography}

\end{document}
